% =============================================================
% preamble.tex — Преамбула монографии
% Оформление: СТУ СФУ (СТУ 7.5-07-2021)
% =============================================================

% --- Язык и шрифты ---
\usepackage[utf8]{inputenc}
\usepackage[T2A]{fontenc}
\usepackage[russian]{babel}
\usepackage{tempora}

% --- Геометрия страницы ---
\usepackage{geometry}
\geometry{
  left=3cm, right=1cm,
  top=2cm, bottom=2cm,
  nohead
}

\raggedbottom

% --- Интервалы ---
\usepackage{setspace}
\singlespacing

% --- Выравнивание ---
\tolerance=1000
\hbadness=1000
\emergencystretch=1.5em

% --- Абзацный отступ ---
\usepackage{indentfirst}
\setlength{\parindent}{1.25cm}

% --- Формулы: отступ слева ---
\setlength{\mathindent}{1.25cm}

% --- Формулы: одинаковые отступы сверху и снизу ---
\usepackage{etoolbox}

% Обнуляем встроенные отступы LaTeX
\AtBeginDocument{%
  \setlength{\abovedisplayskip}{0pt}
  \setlength{\belowdisplayskip}{0pt}
  \setlength{\abovedisplayshortskip}{0pt}
  \setlength{\belowdisplayshortskip}{0pt}
}

% Добавляем одинаковые отступы вручную
\BeforeBeginEnvironment{equation}{\par\vspace{0.5\baselineskip}}
\AfterEndEnvironment{equation}{\vspace{0.5\baselineskip}}

\BeforeBeginEnvironment{align}{\par\vspace{0.5\baselineskip}}
\AfterEndEnvironment{align}{\vspace{0.5\baselineskip}}

\BeforeBeginEnvironment{gather}{\par\vspace{0.5\baselineskip}}
\AfterEndEnvironment{gather}{\vspace{0.5\baselineskip}}

\BeforeBeginEnvironment{multline}{\par\vspace{0.5\baselineskip}}
\AfterEndEnvironment{multline}{\vspace{0.5\baselineskip}}

% --- Перечисления ---
\usepackage{enumitem}
\setlist[itemize]{
  label={-},
  leftmargin=0pt,
  itemindent=\dimexpr\parindent+\labelsep+0.5em\relax,
  labelsep=0.5em,
  listparindent=0pt,
  topsep=0pt,
  partopsep=0pt,
  parsep=0pt,
  itemsep=2pt,
}

% --- Математика ---
\usepackage{amsmath,amssymb,amsthm}
\usepackage{mathtools}

% --- Графика и рисунки ---
\usepackage{graphicx}
\graphicspath{{figures/}}
\usepackage{tikz}
\usetikzlibrary{arrows.meta,calc,patterns}

% --- Таблицы ---
\usepackage{booktabs}
\usepackage{longtable}
\usepackage{array}

% --- Подписи к рисункам и таблицам ---
\usepackage{caption}
\captionsetup{labelsep=endash, justification=centering, font={normalsize}}
\captionsetup[figure]{name=Рисунок, position=bottom}
\captionsetup[table]{name=Таблица, justification=raggedright, singlelinecheck=false, margin=0pt, position=top}

% --- Нумерация формул, рисунков, таблиц по главам ---
\usepackage{chngcntr}
\counterwithin{equation}{chapter}
\counterwithin{figure}{chapter}
\counterwithin{table}{chapter}

% --- Содержание ---
\usepackage[titles]{tocloft}
\renewcommand{\contentsname}{ОГЛАВЛЕНИЕ}

\setlength{\cftbeforetoctitleskip}{0pt}
\setlength{\cftaftertoctitleskip}{\baselineskip}

\setlength{\cftchapindent}{0pt}
\setlength{\cftsecindent}{1.25cm}
\setlength{\cftsubsecindent}{2.5cm}

\renewcommand{\cftchapleader}{\cftdotfill{\cftdotsep}}
\renewcommand{\cftsecleader}{\cftdotfill{\cftdotsep}}
\renewcommand{\cftsubsecleader}{\cftdotfill{\cftdotsep}}

\renewcommand{\cftchapfont}{}
\renewcommand{\cftchappagefont}{}
\renewcommand{\cftsecfont}{}
\renewcommand{\cftsecpagefont}{}
\renewcommand{\cftsubsecfont}{}
\renewcommand{\cftsubsecpagefont}{}

% --- Заголовки разделов ---
\usepackage{titlesec}

% Нумерованные главы (1, 2, 3...)
\titleformat{\chapter}[block]
  {\normalfont\normalsize\bfseries}
  {\thechapter}{0.5em}{}
\titlespacing*{\chapter}{1.25cm}{-20pt}{\baselineskip}

% Ненумерованные главы (ВВЕДЕНИЕ, ОГЛАВЛЕНИЕ, ЗАКЛЮЧЕНИЕ)
\titleformat{name=\chapter,numberless}[block]
  {\normalfont\normalsize\bfseries\centering}
  {}{0pt}{\MakeUppercase}
\titlespacing*{name=\chapter,numberless}{0pt}{-20pt}{\baselineskip}

% Подразделы (1.1, 1.2...)
\titleformat{\section}[block]
  {\normalfont\normalsize\bfseries}
  {\thesection}{0.5em}{}
\titlespacing*{\section}{\parindent}{\baselineskip}{\baselineskip}

% Пункты (1.1.1, 1.1.2...)
\titleformat{\subsection}[block]
  {\normalfont\normalsize\bfseries}
  {\thesubsection}{0.5em}{}
\titlespacing*{\subsection}{\parindent}{\baselineskip}{\baselineskip}

\renewcommand{\chaptername}{}
\renewcommand{\thechapter}{\arabic{chapter}}

% --- Библиография ---
\usepackage[numbers,sort&compress]{natbib}
\addto\captionsrussian{\renewcommand{\bibname}{СПИСОК ИСТОЧНИКОВ}}

% --- Гиперссылки ---
\usepackage{hyperref}
\hypersetup{
  colorlinks=true,
  linkcolor=black,
  citecolor=black,
  urlcolor=black
}

% --- Нумерация страниц ---
\usepackage{fancyhdr}
\pagestyle{fancy}
\fancyhf{}
\fancyfoot[C]{\thepage}
\renewcommand{\headrulewidth}{0pt}
\renewcommand{\footrulewidth}{0pt}

\fancypagestyle{plain}{%
  \fancyhf{}%
  \fancyfoot[C]{\thepage}%
  \renewcommand{\headrulewidth}{0pt}%
  \renewcommand{\footrulewidth}{0pt}%
}