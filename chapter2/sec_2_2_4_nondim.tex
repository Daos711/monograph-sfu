\subsection{Безразмерные параметры микрорельефа}

Для параметрического анализа удобно перейти к безразмерным параметрам текстуры.
Используются масштабы, введённые в подразделе~2.1.6
(таблица~\ref{tab:nondim_scales}): характерная толщина зазора~$h_0$
и характерные длины~$L_1$,~$L_2$ по поверхностным координатам~$x_1$,~$x_2$.
Безразмерные координаты и уравнение Рейнольдса уже заданы
в подразделе~2.1.6; здесь нормируются только параметры текстуры.

Ключевым безразмерным параметром является отношение глубины элемента
к характерной толщине зазора:
\begin{equation}\label{eq:Hp_nondim}
  H_p = \frac{h_p}{h_0}.
\end{equation}
Требование применимости уравнения Рейнольдса к текстурированной
поверхности (подраздел~2.1.5) формулируется как $H_p \ll 1$.

Безразмерные полуоси элемента определяются делением на соответствующие
масштабы длины:
\begin{equation}\label{eq:AB_nondim}
  A^{*} = \frac{a}{L_2}, \qquad B^{*} = \frac{b}{L_1},
\end{equation}
а аспектное отношение элемента --
\begin{equation}\label{eq:kappa}
  \kappa = \frac{a}{b}.
\end{equation}
При $\kappa = 1$ элемент осесимметричен (круговая лунка);
при $\kappa \neq 1$ -- эллиптический.

Безразмерные шаги раскладки:
\begin{equation}\label{eq:P_nondim}
  P_1^{*} = \frac{p_1}{L_1}, \qquad P_2^{*} = \frac{p_2}{L_2}.
\end{equation}
Условия непересечения элементов (подраздел~2.2.1) удобно проверять
через отношения $p_1/(2b)$ и $p_2/(2a)$, которые показывают,
во сколько раз шаг превышает характерный размер элемента.

Коэффициент заполнения~$\phi$ (подраздел~2.2.1) является безразмерным.
Для эллиптических элементов на регулярной решётке
\begin{equation}\label{eq:phi_nondim}
  \phi \approx \frac{\pi\, a\, b}{p_1\, p_2}
       = \frac{\pi\, A^{*}\, B^{*}}{P_1^{*}\, P_2^{*}}.
\end{equation}

Ряд параметров текстуры является безразмерным по определению:
$\rho_0 \in (0,\,1)$ -- параметр плоского дна (подраздел~2.2.2);
$\alpha$ -- угол ориентации элемента.
Для канавок вводится безразмерная длина $\ell^{*} = \ell / L_1$;
для сквозных канавок $\ell^{*} = 1$.
Для повёрнутых элементов ($\alpha \neq 0$) используются локальные
координаты из подраздела~2.2.2; переход к безразмерным величинам
выполняется делением длин на соответствующие масштабы из~2.1.6.

\begin{table}[!ht]
\centering
\caption{Безразмерные параметры микрорельефа}
\label{tab:nondim_texture}
\begin{tabular}{|l|l|l|}
\hline
\textbf{Параметр} & \textbf{Обозначение} & \textbf{Определение} \\
\hline
Безразмерная глубина             & $H_p$    & $h_p / h_0$                          \\
\hline
Безразмерная полуось по $x_2$    & $A^{*}$  & $a / L_2$                            \\
\hline
Безразмерная полуось по $x_1$    & $B^{*}$  & $b / L_1$                            \\
\hline
Отношение полуосей               & $\kappa$ & $a / b$                              \\
\hline
Безразмерный шаг по $x_1$        & $P_1^{*}$ & $p_1 / L_1$                        \\
\hline
Безразмерный шаг по $x_2$        & $P_2^{*}$ & $p_2 / L_2$                        \\
\hline
Коэффициент заполнения           & $\phi$   & $N A_{\mathrm{elem}} / A_{\mathrm{tex}}$ \\
\hline
Параметр плоского дна            & $\rho_0$ & $(0,\,1)$                            \\
\hline
Угол ориентации                  & $\alpha$ & рад                                  \\
\hline
Безразмерная длина канавки       & $\ell^{*}$ & $\ell / L_1$                       \\
\hline
\end{tabular}
\end{table}

Введённый набор безразмерных параметров
($H_p$, $A^{*}$, $B^{*}$, $\kappa$, $P_1^{*}$, $P_2^{*}$, $\phi$)
позволяет проводить параметрические исследования и сравнивать результаты
для различных типов подшипников при одинаковых безразмерных группах.
