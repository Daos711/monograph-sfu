\subsection{Безразмерная постановка}

Переход к безразмерным переменным унифицирует три частные постановки, сформулированные в подразделах~2.1.2--2.1.4, и позволяет сравнивать различные конфигурации подшипников и режимы их работы на единой шкале. Кроме того, безразмеризация сокращает число определяющих параметров задачи до нескольких безразмерных групп, что упрощает параметрический анализ. Введённый аппарат используется далее при исследовании влияния микроструктур поверхности (раздел~2.2) и при численной дискретизации (глава~3).

\subsubsection*{Характерные масштабы и безразмерные переменные}

Общая схема масштабирования повторяет подход, применённый при выводе уравнения Рейнольдса в подразделе~2.1.1, но теперь записывается в обобщённых обозначениях, пригодных для всех трёх постановок:
%
\begin{equation}
\bar{s}_1 = \frac{s_1}{L_1}, \quad
\bar{s}_2 = \frac{s_2}{L_2}, \quad
\bar{h}   = \frac{h}{h_0},   \quad
\bar{p}   = \frac{p}{p_0},   \quad
\bar{U}   = \frac{U}{U_0},   \quad
\bar{t}   = \frac{t}{t_0},
\label{eq:nondim_vars}
\end{equation}
%
где $s_1$, $s_2$ --- поверхностные координаты (конкретные для каждой постановки, см.\ таблицу~\ref{tab:nondim_scales} ниже); $h_0$ --- характерная толщина зазора (радиальный зазор $c$ или средний зазор); $U_0$ --- характерная скорость увлечения; $t_0$ --- характерное время (для нестационарных задач).

Масштаб давления выбирается так, чтобы клиновой член в правой части уравнения Рейнольдса имел порядок единицы. Из баланса~\eqref{eq:pressure_scale} следует:
%
\begin{equation}
p_0 = \frac{6\,\mu_0\,U_0\,L_s}{h_0^2},
\label{eq:nondim_p0}
\end{equation}
%
где $L_s$ --- масштаб длины вдоль направления увлечения, а $\mu_0$ --- характерное значение вязкости.

\subsubsection*{Безразмерное уравнение Рейнольдса}

Подставляя масштабы~\eqref{eq:nondim_vars},~\eqref{eq:nondim_p0} в обобщённое уравнение Рейнольдса~\eqref{eq:reynolds_full} и сокращая размерные множители, получаем единую безразмерную форму:
%
\begin{equation}
\frac{\partial}{\partial \bar{s}_1}\!\left(\bar{h}^{\,3}\,\frac{\partial \bar{p}}{\partial \bar{s}_1}\right)
+ \Lambda^2\,\frac{\partial}{\partial \bar{s}_2}\!\left(\bar{h}^{\,3}\,\frac{\partial \bar{p}}{\partial \bar{s}_2}\right)
= \bar{U}\,\frac{\partial \bar{h}}{\partial \bar{s}_1}
+ \sigma\,\frac{\partial \bar{h}}{\partial \bar{t}},
\label{eq:reynolds_nondim}
\end{equation}
%
где введены две безразмерные группы:
%
\begin{equation}
\Lambda = \frac{L_1}{L_2}
\label{eq:nondim_Lambda}
\end{equation}
%
--- параметр удлинённости (aspect ratio), характеризующий соотношение продольного и поперечного масштабов, и
%
\begin{equation}
\sigma = \frac{2\,L_s}{U_0\,t_0}
\label{eq:nondim_sigma}
\end{equation}
%
--- параметр нестационарности (squeeze number), определяющий относительный вклад выжимания плёнки.

При $\sigma = 0$ (либо в стационарной задаче) член выжимания отсутствует, и уравнение содержит только клиновой источник. Параметр $\Lambda$ определяет относительный вклад <<поперечного>> растекания: при $\Lambda \gg 1$ задача приближается к одномерной. Для цилиндрических координат, используемых в постановках~2.1.2--2.1.4, в левой части уравнения~\eqref{eq:reynolds_nondim} появляются дополнительные множители $1/\bar{r}$ и $1/\bar{r}^{\,2}$, однако общая структура уравнения сохраняется.

\subsubsection*{Масштабы для частных постановок}

Конкретные значения масштабов для трёх постановок, рассмотренных в подразделах~2.1.2--2.1.4, приведены в таблице~\ref{tab:nondim_scales}.

\begin{table}[!ht]
\centering
\caption{Характерные масштабы для безразмерной постановки}
\label{tab:nondim_scales}
\begin{tabular}{lcccccc}
\hline
Постановка & $s_1$ & $s_2$ & $h_0$ & $U_0$ & $p_0$ & $t_0$ \\
\hline
Упорная (2.1.2)
  & $R\theta$ & $r$ & $h_{\min}$ или $c$
  & $\omega R$ & $\dfrac{6\mu\omega R^2}{h_0^2}$ & ---
\\[6pt]
Радиальная (2.1.3)
  & $R\theta$ & $z$ & $c$
  & $\omega R$ & $\dfrac{6\mu\omega R^2}{c^2}$ & ---
\\[6pt]
Возвратно-пост.\ (2.1.4)
  & $z$ & $R\theta$ & $c$
  & $\max|U(t)|$ & $\dfrac{6\mu\,U_0\,L}{c^2}$ & $1/\Omega$
\\[6pt]
\hline
\end{tabular}
\end{table}

В упорной и радиальной постановках задача стационарна, поэтому временной масштаб $t_0$ не требуется (обозначено <<--->>{} в таблице). Координата $s_1$ соответствует направлению увлечения, а $s_2$ --- поперечному направлению. При подстановке масштабов из таблицы в общее уравнение~\eqref{eq:reynolds_nondim} получаются безразмерные формы, согласованные с размерными уравнениями~\eqref{eq:thrust_reynolds_num},~\eqref{eq:journal_reynolds_num},~\eqref{eq:recip_reynolds_num}.

Введённые масштабы определяют безразмерную форму уравнения Рейнольдса и базовой геометрии зазора. Параметры, характеризующие микроструктуры поверхности, --- безразмерная глубина $h_p/h_0$, размеры элементов, плотность покрытия --- вводятся отдельно в подразделе~2.2.4. Далее в работе, если не оговорено иное, используются безразмерные переменные; черта над символом опускается для краткости записи.
