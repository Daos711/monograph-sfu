\subsection{Параметризация микрорельефа}

В разделе~2.1 рассмотрены постановки задачи смазки для гладких
поверхностей. Для учёта детерминированных микроструктур необходимо
задать модификацию функции зазора~$h$, описывающую геометрию
нанесённых на поверхность элементов. Настоящий подраздел вводит общую
параметрическую модель микрорельефа, используемую далее
в~2.2.2--2.2.4. Формулы записываются для произвольных поверхностных
координат; конкретные координаты определяются постановкой
(таблица~\ref{tab:nondim_scales} в подразделе~2.1.6).

%% ===== Блок 1: Суперпозиция =====

\subsubsection*{Суперпозиция зазора}

Рабочую поверхность представляем в виде гладкой базовой поверхности
с наложенными регулярными элементами. Суммарная толщина смазочного
слоя записывается как суперпозиция базового зазора и добавки,
описывающей микрорельеф:
\begin{equation}
  h(x_1, x_2)
    = h_{\mathrm{base}}(x_1, x_2) + \Delta h(x_1, x_2),
  \label{eq:h_superposition}
\end{equation}
где $(x_1, x_2)$ -- поверхностные координаты с размерностью длины.
В зависимости от постановки:
\begin{itemize}
  \item радиальный подшипник: $x_1 = R\theta$, $x_2 = z$;
  \item возвратно-поступательная постановка: $x_1 = z$,
        $x_2 = R\theta$;
  \item упорный подшипник: $x_1 = R\theta$, $x_2 = r$,
\end{itemize}
где $R$ -- характерный радиус (подраздел~2.1.6). Для упорного
подшипника дуговая координата по окружному направлению в общем случае
равна $r\theta$; здесь используется аппроксимация $x_1 = R\theta$
с характерным (например, средним) радиусом, что удобно для
унифицированного задания микрорельефа и раскладки.

Слагаемые в~\eqref{eq:h_superposition} имеют следующий смысл:
\begin{itemize}
  \item $h_{\mathrm{base}}$ -- толщина зазора для гладкой поверхности,
        определённая в подразделах~2.1.2--2.1.4;
  \item $\Delta h$ -- локальная добавка, описывающая микрорельеф.
\end{itemize}
Знаковая конвенция: $\Delta h \leq 0$ для углублений (лунок),
$\Delta h \geq 0$ для выступов. Принцип суперпозиции проиллюстрирован
на рисунке~\ref{fig:h_superposition}.

% [РИСУНОК: Суперпозиция зазора h = h_base + Δh]
% Должен показывать два вида:
% (1) Продольный разрез: базовый клин h_base + ямка Δh → суммарный h
% (2) Вид сверху: пятно одного элемента микроструктуры на фоне сектора
\begin{figure}[!ht]
\centering
\fbox{\parbox{0.8\textwidth}{\centering
\vspace{3.5cm}
ПЛЕЙСХОЛДЕР ДЛЯ РИСУНКА\\
Суперпозиция зазора $h = h_{\mathrm{base}} + \Delta h$:\\
(a) продольный разрез -- базовый клин + локальная ямка;\\
(b) вид сверху -- пятно элемента микроструктуры на секторе
\vspace{3.5cm}
}}
\caption{Суперпозиция толщины зазора: базовый профиль и локальная
  микроструктура}
\label{fig:h_superposition}
\end{figure}

%% ===== Блок 2: Сумма вкладов =====

\subsubsection*{Представление микрорельефа}

Микрорельеф представляется набором из~$N$ локализованных элементов,
вклады которых суммируются:
\begin{equation}
  \Delta h(x_1, x_2)
    = \sum_{k=1}^{N} \Delta h_k(x_1, x_2).
  \label{eq:delta_h_sum}
\end{equation}

Каждый элемент задаётся общим шаблоном (конкретный вид профиля
определяется в подразделе~2.2.2):
\begin{equation}
  \Delta h_k(x_1, x_2)
    = \begin{cases}
        \sigma_k\, h_{p,k}\,
          f_k(x_1 - x_{1,k},\; x_2 - x_{2,k}),
          & (x_1, x_2) \in D_k, \\
        0,
          & (x_1, x_2) \notin D_k,
      \end{cases}
  \label{eq:delta_h_element}
\end{equation}
где:
\begin{itemize}
  \item $D_k$ -- область поддержки (пятно) $k$-го элемента
        на поверхности;
  \item $f_k$ -- профильная функция, задающая форму элемента;
        $f_k(0,0) = 1$;
  \item $\sigma_k = -1$ для углублений (лунок),
        $\sigma_k = +1$ для выступов;
  \item $h_{p,k}$ -- глубина (высота) элемента;
  \item $(x_{1,k},\, x_{2,k})$ -- координаты центра $k$-го элемента.
\end{itemize}

При наличии периодичности по координате $x_1$ (или $x_2$) разность
$x_1 - x_{1,k}$ понимается как минимальная по модулю с учётом
периода.

\paragraph{Эллиптические элементы.}
Для осесимметричных и эллиптических лунок удобно параметризовать
область элемента через безразмерную радиальную координату:
\begin{equation}
  \rho_k
    = \sqrt{
        \left(\frac{x_1 - x_{1,k}}{b_k}\right)^{\!2}
      + \left(\frac{x_2 - x_{2,k}}{a_k}\right)^{\!2}
      }.
  \label{eq:rho_k}
\end{equation}
Тогда $D_k = \{\rho_k \leq 1\}$ -- эллипс с полуосями $b_k$ по~$x_1$
и~$a_k$ по~$x_2$, а профильная функция зависит только от~$\rho$:
$f_k = f(\rho_k)$, причём $f(0) = 1$, $f(1) = 0$.

Для канавок и других неосесимметричных элементов область~$D_k$
и профильная функция~$f_k$ задаются иначе; конкретные формы
рассматриваются в подразделе~2.2.2.

Для однородной текстуры все элементы идентичны
($a_k = a$, $b_k = b$, $h_{p,k} = h_p$, $f_k = f$) и различаются
только положением центров.

% [РИСУНОК 2.5: Типовая микроструктура с параметрами]
\begin{figure}[!ht]
\centering
\fbox{\parbox{0.8\textwidth}{\centering
\vspace{3cm}
ПЛЕЙСХОЛДЕР ДЛЯ РИСУНКА\\
Типовой элемент микроструктуры (лунка) с обозначениями\\
$a$, $b$, $h_p$, $\rho$; профиль в разрезе
\vspace{3cm}
}}
\caption{Типовой элемент микроструктуры с основными геометрическими
  параметрами}
\label{fig:microtexture_params}
\end{figure}

%% ===== Блок 3: Параметры раскладки =====

\subsubsection*{Параметры раскладки}

Положение элементов на поверхности определяется раскладкой. Для её
описания вводятся следующие параметры:
\begin{itemize}
  \item $p_1$, $p_2$ -- шаги раскладки, т.\,е. расстояния между
        центрами соседних элементов по направлениям $x_1$ и~$x_2$;
  \item $N$ -- общее число элементов;
  \item область текстурирования -- участок поверхности, на который
        нанесена текстура (может совпадать со всей расчётной областью
        или быть её частью).
\end{itemize}

Важной интегральной характеристикой является коэффициент заполнения --
доля площади поверхности, занятая элементами:
\begin{equation}
  \phi = \frac{N \cdot A_{\mathrm{elem}}}{A_{\mathrm{tex}}},
  \label{eq:fill_fraction}
\end{equation}
где $A_{\mathrm{elem}}$ -- площадь «пятна» одного элемента (для
эллипса $A_{\mathrm{elem}} = \pi a b$),
$A_{\mathrm{tex}}$ -- площадь области текстурирования.

Конкретные типы решёток (прямоугольная, шахматная, гексагональная
и~др.) и способы задания координат центров рассматриваются
в подразделе~2.2.3.

%% ===== Блок 4: Ограничения модели =====

\subsubsection*{Ограничения модели}

Введённая параметризация корректна при выполнении следующих условий.
\begin{enumerate}
  \item Положительность зазора: $h(x_1, x_2) > 0$ на всей расчётной
        области. Формально:
        $h_{\mathrm{base}}(x_1, x_2) + \Delta h(x_1, x_2) > 0$.
        В частности, при непересекающихся одинаковых лунках достаточно
        $h_p < h_{\mathrm{base,min}}$.
  \item Непересечение элементов:
        $D_k \cap D_{k'} = \varnothing$ для всех $k \neq k'$.
        Для эллиптических элементов без поворота достаточное условие:
        $p_1 \geq 2b$, $p_2 \geq 2a$.
  \item Малая глубина: $h_p / h_0 \ll 1$ обеспечивает применимость
        уравнения Рейнольдса к текстурированной поверхности
        (см.\ подраздел~2.1.5).
  \item Гладкость профиля: непрерывность $\Delta h$ на границе
        элемента ($f(1) = 0$). Желательна также гладкая стыковка
        ($f'(1) = 0$) для уменьшения численных артефактов. Выполнение
        этих условий зависит от выбора профильной функции~$f(\rho)$
        (подраздел~2.2.2).
\end{enumerate}

%% ===== Блок 5: Таблица параметров =====

\begin{table}[!ht]
\centering
\caption{Параметры детерминированной микроструктуры}
\label{tab:texture_params}
\begin{tabular}{|l|l|l|}
\hline
\textbf{Параметр} & \textbf{Обозначение} & \textbf{Размерность} \\
\hline
Глубина (высота) элемента & $h_p$     & мкм \\
\hline
Полуось по $x_1$ (эллиптич.) & $b$    & мкм \\
\hline
Полуось по $x_2$ (эллиптич.) & $a$    & мкм \\
\hline
Шаг по $x_1$              & $p_1$     & мкм \\
\hline
Шаг по $x_2$              & $p_2$     & мкм \\
\hline
Число элементов           & $N$       & --  \\
\hline
Доля площади (заполнение) & $\phi$    & --  \\
\hline
Профильная функция        & $f(\rho)$ & --  \\
\hline
\end{tabular}
\end{table}

%% ===== Завершающий абзац =====

Введённая параметрическая модель позволяет описать широкий класс
детерминированных микроструктур. Конкретные формы профильной
функции~$f(\rho)$ рассматриваются в подразделе~2.2.2, типы
раскладок -- в подразделе~2.2.3, а переход к безразмерным
параметрам -- в подразделе~2.2.4.
