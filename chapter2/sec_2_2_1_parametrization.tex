\subsection{Параметризация микрорельефа: $h(x,z)$ и параметры}

% СОДЕРЖАНИЕ:
% - Как задается функция h(x,z)
% - Параметры: глубина h_d, радиусы a и b, шаг p, доля площади φ

Для учёта детерминированных микроструктур рабочую поверхность представляем в виде гладкой базовой поверхности с наложенными регулярными элементами. Суммарная толщина смазочного слоя записывается как суперпозиция:
\begin{equation}
h(x, z) = h_{\mathrm{base}}(x, z) + \Delta h(x, z),
\label{eq:h_superposition}
\end{equation}
где $h_{\mathrm{base}}$ -- толщина зазора для гладкой поверхности (клиновой профиль, определённый в подразделах~2.1.2--2.1.4), а $\Delta h$ -- локальная добавка, описывающая геометрию микроструктуры ($\Delta h \leq 0$ для углублений, $\Delta h \geq 0$ для выступов). Принцип суперпозиции проиллюстрирован на рисунке~\ref{fig:h_superposition}.

% [РИСУНОК: Суперпозиция зазора h = h_base + Δh]
% Должен показывать два вида:
% (1) Продольный разрез: базовый клин h_base + ямка Δh → суммарный h
% (2) Вид сверху: пятно одного элемента микроструктуры на фоне сектора
\begin{figure}[!ht]
\centering
\fbox{\parbox{0.8\textwidth}{\centering
\vspace{3.5cm}
ПЛЕЙСХОЛДЕР ДЛЯ РИСУНКА\\
Суперпозиция зазора $h = h_{\mathrm{base}} + \Delta h$:\\
(a) продольный разрез -- базовый клин + локальная ямка;\\
(b) вид сверху -- пятно элемента микроструктуры на секторе
\vspace{3.5cm}
}}
\caption{Суперпозиция толщины зазора: базовый профиль и локальная микроструктура}
\label{fig:h_superposition}
\end{figure}

% [РИСУНОК 2.5: Типовая микроструктура с параметрами]
\begin{figure}[!ht]
\centering
\fbox{\parbox{0.8\textwidth}{\centering
\vspace{3cm}
ПЛЕЙСХОЛДЕР ДЛЯ РИСУНКА 2.5\\
Типовая микроструктура (лунка/канавка)\\
Параметры: $a$, $b$, $h_d$, профиль в разрезе
\vspace{3cm}
}}
\caption{Типовая микроструктура с основными параметрами}
\label{fig:microtexture_params}
\end{figure}

% [ТАБЛИЦА 2.2: Параметры микроструктуры]
\begin{table}[h]
\centering
\caption{Параметры детерминированной микроструктуры}
\label{tab:texture_params}
\begin{tabular}{|l|l|l|}
\hline
\textbf{Параметр} & \textbf{Обозначение} & \textbf{Размерность} \\
\hline
Глубина элемента & $h_d$ & мкм \\
Радиус (полуось) & $a, b$ & мкм \\
Шаг раскладки & $p$ & мкм \\
Доля площади & $\phi$ & -- \\
% TODO: Добавить остальные & & \\
\hline
\end{tabular}
\end{table}
