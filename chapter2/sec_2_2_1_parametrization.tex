\subsection{Параметризация микрорельефа}

В разделе~2.1 рассмотрены постановки задачи смазки для гладких
поверхностей. Для учёта детерминированных микроструктур необходимо
задать модификацию функции зазора~$h$, описывающую геометрию
нанесённых на поверхность элементов. Настоящий подраздел вводит общую
параметрическую модель микрорельефа, используемую далее
в~2.2.2--2.2.4. Формулы записываются для произвольных поверхностных
координат; конкретные координаты определяются постановкой
(таблица~\ref{tab:nondim_scales} в подразделе~2.1.6).

%% ===== Блок 1: Суперпозиция =====

\subsubsection*{Суперпозиция зазора}

Рабочую поверхность представляем в виде гладкой базовой поверхности
с наложенными регулярными элементами. Суммарная толщина смазочного
слоя записывается как суперпозиция базового зазора и добавки,
описывающей микрорельеф:
\begin{equation}
  h(x_1, x_2)
    = h_{\mathrm{base}}(x_1, x_2) + \Delta h(x_1, x_2),
  \label{eq:h_superposition}
\end{equation}
где $(x_1, x_2)$ -- координаты на рабочей поверхности:
$(\theta, z)$ для радиального и возвратно-поступательного
подшипников, $(r, \theta)$ для упорного
(подразделы~2.1.2--2.1.4).

Слагаемые в~\eqref{eq:h_superposition} имеют следующий смысл:
\begin{itemize}
  \item $h_{\mathrm{base}}$ -- толщина зазора для гладкой поверхности,
        определённая в подразделах~2.1.2--2.1.4;
  \item $\Delta h$ -- локальная добавка, описывающая микрорельеф.
\end{itemize}
Знаковая конвенция: $\Delta h \leq 0$ для углублений (лунок),
$\Delta h \geq 0$ для выступов. Принцип суперпозиции проиллюстрирован
на рисунке~\ref{fig:h_superposition}.

% [РИСУНОК: Суперпозиция зазора h = h_base + Δh]
% Должен показывать два вида:
% (1) Продольный разрез: базовый клин h_base + ямка Δh → суммарный h
% (2) Вид сверху: пятно одного элемента микроструктуры на фоне сектора
\begin{figure}[!ht]
\centering
\fbox{\parbox{0.8\textwidth}{\centering
\vspace{3.5cm}
ПЛЕЙСХОЛДЕР ДЛЯ РИСУНКА\\
Суперпозиция зазора $h = h_{\mathrm{base}} + \Delta h$:\\
(a) продольный разрез -- базовый клин + локальная ямка;\\
(b) вид сверху -- пятно элемента микроструктуры на секторе
\vspace{3.5cm}
}}
\caption{Суперпозиция толщины зазора: базовый профиль и локальная
  микроструктура}
\label{fig:h_superposition}
\end{figure}

%% ===== Блок 2: Сумма вкладов =====

\subsubsection*{Представление микрорельефа}

Микрорельеф представляется набором из~$N$ локализованных элементов,
вклады которых суммируются:
\begin{equation}
  \Delta h(x_1, x_2)
    = \sum_{k=1}^{N} \Delta h_k(x_1, x_2).
  \label{eq:delta_h_sum}
\end{equation}

Каждый элемент задаётся шаблоном (конкретный вид профиля определяется
в подразделе~2.2.2):
\begin{equation}
  \Delta h_k(x_1, x_2)
    = \begin{cases}
        \pm\, h_{p,k}\, f_k(\rho_k), & \rho_k \leq 1, \\
        0,                             & \rho_k > 1,
      \end{cases}
  \label{eq:delta_h_element}
\end{equation}
где безразмерная радиальная координата~$\rho_k$ определяет положение
точки относительно центра $k$-го элемента:
\begin{equation}
  \rho_k
    = \sqrt{
        \left(\frac{x_1 - x_{1,k}}{b_k}\right)^{\!2}
      + \left(\frac{x_2 - x_{2,k}}{a_k}\right)^{\!2}
      }.
  \label{eq:rho_k}
\end{equation}

Параметры, входящие в~\eqref{eq:delta_h_element}
и~\eqref{eq:rho_k}:
\begin{itemize}
  \item $(x_{1,k},\, x_{2,k})$ -- координаты центра $k$-го элемента;
  \item $a_k$, $b_k$ -- полуоси (полуразмеры) элемента по направлениям
        $x_2$ и~$x_1$ соответственно;
  \item $h_{p,k}$ -- глубина (для лунки) или высота (для выступа);
  \item $f_k(\rho)$ -- профильная функция, удовлетворяющая условиям
        $f_k(0) = 1$, $f_k(1) = 0$; конкретные формы рассматриваются
        в подразделе~2.2.2;
  \item знак «$+$» соответствует выступам, «$-$» -- углублениям.
\end{itemize}

Для однородной текстуры все элементы идентичны
($a_k = a$, $b_k = b$, $h_{p,k} = h_p$, $f_k = f$) и различаются
только положением центров.

% [РИСУНОК 2.5: Типовая микроструктура с параметрами]
\begin{figure}[!ht]
\centering
\fbox{\parbox{0.8\textwidth}{\centering
\vspace{3cm}
ПЛЕЙСХОЛДЕР ДЛЯ РИСУНКА\\
Типовой элемент микроструктуры (лунка) с обозначениями\\
$a$, $b$, $h_p$, $\rho$; профиль в разрезе
\vspace{3cm}
}}
\caption{Типовой элемент микроструктуры с основными геометрическими
  параметрами}
\label{fig:microtexture_params}
\end{figure}

%% ===== Блок 3: Параметры раскладки =====

\subsubsection*{Параметры раскладки}

Положение элементов на поверхности определяется раскладкой. Для её
описания вводятся следующие параметры:
\begin{itemize}
  \item $p_1$, $p_2$ -- шаги раскладки, т.\,е. расстояния между
        центрами соседних элементов по направлениям $x_1$ и~$x_2$;
  \item $N$ -- общее число элементов;
  \item область текстурирования -- участок поверхности, на который
        нанесена текстура (может совпадать со всей расчётной областью
        или быть её частью).
\end{itemize}

Важной интегральной характеристикой является коэффициент заполнения --
доля площади поверхности, занятая элементами:
\begin{equation}
  \phi = \frac{N \cdot A_{\mathrm{elem}}}{A_{\mathrm{tex}}},
  \label{eq:fill_fraction}
\end{equation}
где $A_{\mathrm{elem}}$ -- площадь «пятна» одного элемента (для
эллипса $A_{\mathrm{elem}} = \pi a b$),
$A_{\mathrm{tex}}$ -- площадь области текстурирования.

Конкретные типы решёток (прямоугольная, шахматная, гексагональная
и~др.) и способы задания координат центров рассматриваются
в подразделе~2.2.3.

%% ===== Блок 4: Ограничения модели =====

\subsubsection*{Ограничения модели}

Введённая параметризация корректна при выполнении следующих условий.
\begin{enumerate}
  \item Положительность зазора: $h(x_1, x_2) > 0$ на всей расчётной
        области. Это накладывает ограничение на глубину элементов:
        $h_p < h_{\mathrm{base,min}}$.
  \item Непересечение элементов: области влияния соседних элементов
        не перекрываются. Формально: $\rho_k > 1$ для всех
        $k' \neq k$ в точке центра $k'$-го элемента (или
        эквивалентное условие на шаги $p_1$, $p_2$ относительно
        $a$,~$b$).
  \item Малая глубина: $h_p / h_0 \ll 1$ обеспечивает применимость
        уравнения Рейнольдса к текстурированной поверхности
        (см.\ подраздел~2.1.5).
  \item Гладкость профиля: непрерывность $\Delta h$ на границе
        элемента ($f(1) = 0$). Желательна также гладкая стыковка
        ($f'(1) = 0$) для уменьшения численных артефактов. Выполнение
        этих условий зависит от выбора профильной функции~$f(\rho)$
        (подраздел~2.2.2).
\end{enumerate}

%% ===== Блок 5: Таблица параметров =====

\begin{table}[!ht]
\centering
\caption{Параметры детерминированной микроструктуры}
\label{tab:texture_params}
\begin{tabular}{|l|l|l|}
\hline
\textbf{Параметр} & \textbf{Обозначение} & \textbf{Размерность} \\
\hline
Глубина (высота) элемента & $h_p$     & мкм \\
\hline
Полуось по $x_1$          & $b$       & мкм \\
\hline
Полуось по $x_2$          & $a$       & мкм \\
\hline
Шаг по $x_1$              & $p_1$     & мкм \\
\hline
Шаг по $x_2$              & $p_2$     & мкм \\
\hline
Число элементов           & $N$       & --  \\
\hline
Доля площади (заполнение) & $\phi$    & --  \\
\hline
Профильная функция        & $f(\rho)$ & --  \\
\hline
\end{tabular}
\end{table}

%% ===== Завершающий абзац =====

Введённая параметрическая модель позволяет описать широкий класс
детерминированных микроструктур. Конкретные формы профильной
функции~$f(\rho)$ рассматриваются в подразделе~2.2.2, типы
раскладок -- в подразделе~2.2.3, а переход к безразмерным
параметрам -- в подразделе~2.2.4.
