\subsection{Типы раскладки микроструктур}

Геометрия отдельного элемента микрорельефа задана
в подразделе~2.2.2. Настоящий подраздел определяет способы
расположения элементов на рабочей поверхности. Результат --
множество центров $\{(x_{1,k},\, x_{2,k})\}_{k=1}^{N}$ (и при
необходимости углов поворота~$\alpha_k$ для канавок). Раскладка
задаётся на прямоугольной области текстурирования
$\Omega_{\mathrm{tex}}
  = [x_{1,s},\, x_{1,t}] \times [x_{2,s},\, x_{2,t}]$
в поверхностных координатах~$(x_1, x_2)$. По периодическому
направлению (например, $x_1 = R\theta$ с периодом $L_1 = 2\pi R$)
координаты центров понимаются по модулю~$L_1$.

%% ===== Блок 1: Прямоугольная решётка =====

\subsubsection*{Прямоугольная решётка}

Базовый тип раскладки. Центры расположены в узлах прямоугольной
сетки с шагами~$p_1$, $p_2$ и начальными смещениями~$\delta_1$,
$\delta_2$:
\begin{equation}
  x_{1,i} = x_{1,s} + \delta_1 + i\,p_1, \qquad
  x_{2,j} = x_{2,s} + \delta_2 + j\,p_2,
  \label{eq:layout_rect}
\end{equation}
где $i = 0, 1, \ldots, I_{\max}$; $j = 0, 1, \ldots, J_{\max}$.
Обычно выбирают $0 \leq \delta_1 < p_1$,
$0 \leq \delta_2 < p_2$. Тогда
\[
  I_{\max}
    = \left\lfloor \frac{x_{1,t} - x_{1,s} - \delta_1}{p_1} \right\rfloor,
  \qquad
  J_{\max}
    = \left\lfloor \frac{x_{2,t} - x_{2,s} - \delta_2}{p_2} \right\rfloor.
\]

Площадь элементарной ячейки:
$S_{\mathrm{cell}} = p_1 \cdot p_2$.
Коэффициент заполнения (связь с формулой~\eqref{eq:fill_fraction}
из подраздела~2.2.1):
\begin{equation}
  \phi \approx \frac{A_{\mathrm{elem}}}{p_1\,p_2}.
  \label{eq:fill_rect}
\end{equation}

%% ===== Блок 2: Шахматная решётка =====

\subsubsection*{Шахматная решётка}

Чётные и нечётные ряды сдвинуты друг относительно друга на
половину шага по~$x_1$:
\begin{equation}
  \begin{aligned}
    x_{2,j}   &= x_{2,s} + \delta_2 + j\,p_2, \\
    x_{1,i,j} &= x_{1,s} + \delta_1 + i\,p_1
                  + \tfrac{1}{2}\,p_1\,(j \bmod 2).
  \end{aligned}
  \label{eq:layout_staggered}
\end{equation}
Индекс~$i$ для фиксированного~$j$ выбирается так, чтобы
$x_{1,i,j} \leq x_{1,t}$.
Шахматная раскладка повышает равномерность покрытия при тех же
шагах~$p_1$, $p_2$. Площадь элементарной ячейки та же:
$S_{\mathrm{cell}} = p_1 \cdot p_2$.

%% ===== Блок 3: Гексагональная решётка =====

\subsubsection*{Гексагональная решётка}

Является частным случаем шахматной раскладки с определённым
соотношением шагов. Задаётся одним параметром -- расстоянием между
ближайшими центрами~$p$:
\begin{equation}
  p_1 = p, \qquad p_2 = \frac{\sqrt{3}}{2}\,p.
  \label{eq:layout_hex}
\end{equation}
Формулы координат -- как для шахматной
решётки~\eqref{eq:layout_staggered} с этими шагами.
Гексагональная раскладка обеспечивает наиболее плотную упаковку
круговых элементов на плоскости. Площадь элементарной ячейки:
$S_{\mathrm{cell}} = p_1\,p_2 = \frac{\sqrt{3}}{2}\,p^2$.
Соответствующая оценка коэффициента заполнения:
\[
  \phi \approx \frac{A_{\mathrm{elem}}}{S_{\mathrm{cell}}}
    = \frac{2\,A_{\mathrm{elem}}}{\sqrt{3}\,p^2}.
\]

%% ===== Блок 4: Филлотаксис =====

\subsubsection*{Спиральная раскладка (филлотаксис)}

Центры элементов располагаются по спирали Ферма с шагом по углу,
равным золотому углу $\gamma = 2\pi(2 - \varphi)$, где
$\varphi = (1 + \sqrt{5})/2$ -- золотое сечение:
\begin{equation}
  \begin{aligned}
    r_k      &= c\,\sqrt{k}, \\
    \theta_k &= k\,\gamma,
  \end{aligned}
  \qquad k = 1, 2, \ldots, N,
  \label{eq:layout_phyllotaxis}
\end{equation}
где $c$ -- масштабный коэффициент, определяющий расстояние между
витками (связан с требуемой плотностью заполнения). Площадь,
приходящаяся на один элемент, приближённо равна $\pi c^2$, откуда
\begin{equation}
  c \approx \sqrt{\frac{A_{\mathrm{tex}}}{\pi\,N}}.
  \label{eq:phyllotaxis_c}
\end{equation}
Координаты
центров в поверхностных координатах $(x_{1,k},\, x_{2,k})$
пересчитываются из $(r_k, \theta_k)$ в зависимости от постановки.

Филлотаксис обеспечивает равномерное заполнение круговой или
кольцевой области без выраженных рядов и столбцов. Раскладка
особенно удобна для кольцевых областей (упорный подшипник).

%% ===== Блок 5: Раскладка канавок =====

\subsubsection*{Раскладка канавок}

Для канавок раскладка упрощается: центры располагаются вдоль одного
направления с постоянным шагом. Например, для канавок,
ориентированных вдоль~$x_1$ ($\alpha_k = 0$), центры задаются как
\[
  x_{2,k} = x_{2,s} + \delta_2 + k\,p_2,
  \quad k = 0, 1, \ldots, N_g - 1,
\]
а $x_{1,k} = (x_{1,s} + x_{1,t})/2$. Для сквозных канавок по всему
периодическому направлению длина $\ell = L_1$.
Для перекрёстных канавок (например, продольные + поперечные)
задаются два семейства с разными углами~$\alpha$.

%% ===== Блок 6: Рисунки =====

% [РИСУНОК: Типы решёток]
\begin{figure}[!ht]
\centering
\fbox{\parbox{0.85\textwidth}{\centering
\vspace{3.5cm}
ПЛЕЙСХОЛДЕР ДЛЯ РИСУНКА\\
Типы раскладок (вид сверху, точки -- центры, контуры -- элементы):\\
(a) прямоугольная, (b) шахматная, (c) гексагональная;\\
шаги $p_1$, $p_2$; сдвиг чётных рядов
\vspace{3.5cm}
}}
\caption{Основные типы регулярных раскладок микроструктур}
\label{fig:layout_staggered}
\end{figure}

% [РИСУНОК: Филлотаксис]
\begin{figure}[!ht]
\centering
\fbox{\parbox{0.7\textwidth}{\centering
\vspace{3cm}
ПЛЕЙСХОЛДЕР ДЛЯ РИСУНКА\\
Спиральная раскладка (филлотаксис):\\
точки-центры на спирали Ферма,\\
золотой угол $\gamma$, масштаб $c$
\vspace{3cm}
}}
\caption{Раскладка микроструктур по принципу филлотаксиса}
\label{fig:layout_phyllotaxis}
\end{figure}

%% ===== Правило включения центров =====

В настоящей работе элемент считается принадлежащим области
текстурирования, если его центр $(x_{1,k},\, x_{2,k})$ лежит
в~$\Omega_{\mathrm{tex}}$. При необходимости исключения элементов,
частично выходящих за границу, дополнительно требуют, чтобы
область~$D_k$ целиком помещалась в~$\Omega_{\mathrm{tex}}$, что
эквивалентно введению защитных отступов порядка~$a$, $b$ от границ
области.

%% ===== Завершающий абзац =====

Выбор типа раскладки определяется формой области текстурирования,
типом подшипника и целями параметрического анализа. Безразмерные
параметры, характеризующие микрорельеф в целом, вводятся
в подразделе~2.2.4.
