\subsection{Типы раскладки микроструктур}

% СОДЕРЖАНИЕ:
% - Регулярная решетка (шахматная/прямоугольная/гексагональная)
% - Кольцевая/секторальная
% - Филлотаксис

Рассмотрим варианты пространственного расположения микроструктур...

% [РИСУНОК 2.7: Шахматная раскладка]
\begin{figure}[!ht]
\centering
\fbox{\parbox{0.6\textwidth}{\centering
\vspace{2cm}
ПЛЕЙСХОЛДЕР ДЛЯ РИСУНКА 2.7\\
Шахматная раскладка микролунок\\
(вид сверху, координаты центров)
\vspace{2cm}
}}
\caption{Шахматная раскладка микроструктур}
\label{fig:layout_staggered}
\end{figure}

% [РИСУНОК 2.8: Кольцевая раскладка]
\begin{figure}[!ht]
\centering
\fbox{\parbox{0.6\textwidth}{\centering
\vspace{2cm}
ПЛЕЙСХОЛДЕР ДЛЯ РИСУНКА 2.8\\
Кольцевая раскладка (для радиального подшипника)
\vspace{2cm}
}}
\caption{Кольцевая раскладка микроструктур}
\label{fig:layout_annular}
\end{figure}

% [РИСУНОК 2.9: Филлотаксис]
\begin{figure}[!ht]
\centering
\fbox{\parbox{0.6\textwidth}{\centering
\vspace{2cm}
ПЛЕЙСХОЛДЕР ДЛЯ РИСУНКА 2.9\\
Раскладка по принципу филлотаксиса
\vspace{2cm}
}}
\caption{Раскладка микроструктур по принципу филлотаксиса}
\label{fig:layout_phyllotaxis}
\end{figure}
