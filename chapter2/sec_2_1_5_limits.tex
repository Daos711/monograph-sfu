\subsection{Область применимости модели}

Допущения, положенные в основу уравнения Рейнольдса, перечислены в подразделе~2.1.1 (п.~1--8). Ниже для каждого из них указаны условия, при которых допущение может нарушаться, и соответствующие расширения модели. Часть расширений реализована в последующих разделах и главах настоящей работы.

\subsubsection*{Геометрия тонкого слоя}

Смазочная аппроксимация предполагает малость отношения $\epsilon = H/L \ll 1$ и малые уклоны поверхностей $|\partial h/\partial x| \ll 1$, $|\partial h/\partial z| \ll 1$. При широких зазорах, очень коротких контактах или наличии резких кромок и ступенек эти условия нарушаются, и необходимо обращаться к CFD-моделированию на основе полных уравнений Навье--Стокса. При введении микроструктур с резкими кромками (раздел~2.2) уклоны локально могут быть велики; на практике это допустимо, если характерная глубина микроструктуры мала по сравнению с зазором ($h_p \ll h_0$).

\subsubsection*{Инерция}

Инерционные члены пренебрежимо малы при $\mathrm{Re}\,\epsilon \ll 1$, см.~\eqref{eq:Re_epsilon}. При высоких скоростях, низкой вязкости или увеличенном зазоре этот параметр растёт, и инерционные эффекты становятся существенными. В таких случаях применяются инерционные поправки к уравнению Рейнольдса или переход к полным уравнениям Навье--Стокса.

\subsubsection*{Ламинарность}

С ростом скорости и зазора или при снижении вязкости течение может утратить устойчивость и перейти в турбулентный режим. Для описания турбулентных смазочных плёнок используются модифицированные модели с эффективной вязкостью (подходы Hirs, Ng--Pan). В настоящей работе рассматривается ламинарный режим, характерный для подшипников скольжения при малых зазорах.

\subsubsection*{Несжимаемость}

При газовой смазке или при высоких давлениях, вызывающих заметное изменение плотности, допущение $\rho = \mathrm{const}$ нарушается. Сжимаемая форма уравнения Рейнольдса записывается через баланс массы:
\begin{equation}
\frac{\partial (\rho h)}{\partial t}
+ \frac{\partial (\rho\, q_x)}{\partial x}
+ \frac{\partial (\rho\, q_z)}{\partial z} = 0,
\label{eq:reynolds_compressible}
\end{equation}
где $q_x$ и $q_z$ -- объёмные расходы~\eqref{eq:flow_rate_x},~\eqref{eq:flow_rate_z}. При $\rho = \mathrm{const}$ уравнение~\eqref{eq:reynolds_compressible} сводится к уже полученному уравнению~\eqref{eq:reynolds_full}. В настоящей работе рассматривается несжимаемая жидкость.

\subsubsection*{Постоянная вязкость и изотермичность}

Интенсивный сдвиг в тонкой смазочной плёнке может приводить к значительному вязкостному нагреву. В этом случае вязкость становится функцией температуры, $\mu = \mu(T)$, и уравнение Рейнольдса необходимо решать совместно с уравнением энергии -- термогидродинамическая (THD) постановка. При высоких контактных давлениях существенной становится пьезовязкость $\mu = \mu(p)$, характерная для упругогидродинамической смазки (EHL). Обобщённое уравнение Рейнольдса~\eqref{eq:reynolds_full} допускает переменную вязкость $\mu = \mu(x, z)$ без изменения структуры левой части. Влияние температурной зависимости вязкости рассматривается в главе~4.

\subsubsection*{Условие прилипания}

При очень малых зазорах (порядка десятков нанометров) или при наличии специальных покрытий (гидрофобных, олеофобных) возможно проскальзывание жидкости на стенке (Navier slip). В данной работе скольжение на границе не учитывается.

\subsubsection*{Жёсткие поверхности}

Модель предполагает геометрически заданный зазор $h(x, z, t)$, не зависящий от давления. При высоких нагрузках или использовании податливых материалов упругая деформация поверхностей изменяет форму зазора, что приводит к упругогидродинамической (EHL) постановке: уравнение Рейнольдса решается совместно с уравнением упругости. В настоящей работе поверхности полагаются жёсткими.

\subsubsection*{Кавитация}

В областях расходящегося зазора давление, полученное из уравнения Рейнольдса, может становиться отрицательным, что физически соответствует разрыву сплошности смазочного слоя (кавитации). Простое ограничение $p \geq 0$ (условие Гюмбеля) нарушает баланс массы на границе кавитационной зоны. Массо-сохраняющий подход Jakobsson--Floberg--Olsson (JFO), обеспечивающий корректное описание кавитации, вводится в разделе~2.3.

В рамках настоящей работы базовые допущения расширяются в следующих направлениях:
\begin{itemize}
  \item кавитация и массо-сохранение -- раздел~2.3;
  \item микроструктуры поверхности -- раздел~2.2;
  \item температурная зависимость вязкости $\mu(T)$ -- глава~4.
\end{itemize}
Остальные допущения (ламинарность, несжимаемость, жёсткие поверхности, условие прилипания) сохраняются на протяжении всей работы.
