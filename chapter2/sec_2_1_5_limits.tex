\subsection{Допущения и область применимости модели}

% СОДЕРЖАНИЕ:
% - Ньютоновская жидкость (μ = const)
% - Тонкий слой (h << L)
% - Ламинарность (Re << 1)
% - Изотермичность / неизотермичность
% - Несжимаемость (ρ = const)
% - Отсутствие инерции
% - Где модель ломается

Допущения, принятые при выводе уравнения Рейнольдса, перечислены в подразделе~2.1.1. Здесь обсудим границы применимости полученной модели и условия, при которых отдельные допущения могут нарушаться.

В рамках смазочной аппроксимации ключевым является малый параметр $\epsilon = H/L$. При росте числа Рейнольдса или при больших уклонах поверхности точность модели снижается. Ниже перечислены основные факторы, требующие расширения модели.

% СОДЕРЖАНИЕ 2.1.5 (границы применимости):
% - При каких значениях Re·ε модель теряет точность (инерция)
% - Турбулентность: при каких Re ламинарность нарушается
% - Нагрев: когда нужна μ(T) и уравнение энергии
% - Неньютоновское поведение: полимерные/загущённые масла
% - Большие уклоны поверхности: |dh/dx| ~ 1
% - Упругое деформирование поверхностей (EHL)
% - Что из перечисленного снимается/расширяется далее в монографии

% TODO: Расписать границы применимости по пунктам выше
