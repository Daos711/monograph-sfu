\subsection{Уравнения Jakobsson-Floberg-Olsson}

% СОДЕРЖАНИЕ:
% - Полная формулировка JFO
% - Условия переключения
% - Форма, пригодная для численной реализации

Массо-сохраняющая модель кавитации, предложенная Jakobsson, Floberg и Olsson...

% TODO: Написать уравнения JFO

Принципиальное отличие модели JFO от простого обнуления давления проиллюстрировано на рисунке~\ref{fig:jfo_1d_profile}.

% [РИСУНОК: 1D-профиль давления и степени заполнения в модели JFO]
% Должен показывать:
% - Ось абсцисс: координата вдоль направления течения (θ или x)
% - Верхний график: p(θ) — давление; в зоне кавитации p = 0 (горизонтальная линия)
% - Нижний график (или наложенный): α(θ) или θ_f(θ) — степень заполнения;
%   α = 1 в зоне полного слоя, α < 1 в зоне кавитации
% - Вертикальные штриховые линии: граница rupture (разрыва) и reformation (восстановления)
% - Для сравнения: пунктиром показать "обнулённое" p < 0 (модель Гюмбеля)
\begin{figure}[!ht]
\centering
\fbox{\parbox{0.8\textwidth}{\centering
\vspace{3.5cm}
ПЛЕЙСХОЛДЕР ДЛЯ РИСУНКА\\
1D-профиль вдоль направления течения:\\
давление $p(\theta)$ и степень заполнения $\alpha(\theta)$;\\
границы разрыва (rupture) и восстановления (reformation);\\
пунктир -- решение с обнулением $p \geq 0$ (Гюмбель)
\vspace{3.5cm}
}}
\caption{Сравнение моделей кавитации: обнуление давления (Гюмбель) и массо-сохраняющая постановка (JFO)}
\label{fig:jfo_1d_profile}
\end{figure}
