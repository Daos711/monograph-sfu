\subsection{Радиальная постановка задачи}

% Радиальный подшипник (journal bearing)

Рассмотрим радиальный (опорный) подшипник скольжения -- узел, в котором цилиндрический вал (цапфа) вращается внутри неподвижной втулки (вкладыша), а тонкий слой вязкой жидкости, заполняющий зазор между валом и втулкой, воспринимает радиальную нагрузку, перпендикулярную оси вала. Несущая способность подшипника обусловлена гидродинамическим давлением, генерируемым в сходящейся части зазора при вращении вала.

Радиальные подшипники скольжения широко применяются в инженерной практике. В двигателях внутреннего сгорания они служат опорами коленчатого вала и шатунов. В паровых и газовых турбоагрегатах радиальные подшипники обеспечивают устойчивую работу ротора при высоких частотах вращения. В центробежных насосах и компрессорах они воспринимают радиальные усилия, возникающие от неуравновешенности и аэро- или гидродинамических сил. В шарошечных долотах для бурения нефтяных и газовых скважин радиальные подшипники работают в условиях высоких нагрузок и ограниченного пространства.

Цель постановки: для заданной геометрии зазора $h(\theta)$, определяемой эксцентриситетом вала, и режима движения поверхностей получить распределение давления $p(\theta, z)$ в смазочном слое. По найденному полю давления далее вычисляются интегральные характеристики, приведённые в подразделе~2.1.7. Все допущения, принятые при выводе уравнения Рейнольдса в подразделе~2.1.1, сохраняют силу: жидкость ньютоновская и несжимаемая, течение ламинарное и стационарное, давление постоянно по толщине смазочного слоя. В отличие от упорного подшипника (подраздел~2.1.2), где нагрузка осевая и расчётная область представляет собой сектор в плоскости $(r, \theta)$, в радиальном подшипнике нагрузка действует перпендикулярно оси вала, а расчётная область формируется на <<развёрнутой>> цилиндрической поверхности зазора в координатах $(\theta, z)$.


\subsubsection*{Геометрия и система координат}

Основными геометрическими параметрами радиального подшипника являются:
\begin{itemize}
\item $R$ -- радиус вала (цапфы);
\item $R_b$ -- радиус внутренней поверхности втулки (вкладыша);
\item $c = R_b - R$ -- радиальный зазор;
\item $e$ -- эксцентриситет, т.\,е. расстояние между центром втулки $O_b$ и центром вала $O_j$;
\item $\varepsilon = e/c$ -- относительный эксцентриситет ($0 \leq \varepsilon < 1$).
\end{itemize}

Параметр тонкослойности~$\epsilon = H/L$, введённый в подразделе~2.1.1, не следует путать с относительным эксцентриситетом~$\varepsilon$: первый характеризует геометрию смазочного слоя как целого, второй~-- положение вала внутри втулки.

Прямая, соединяющая центр втулки $O_b$ с центром вала $O_j$, называется линией центров. Положение вала внутри втулки характеризуется двумя величинами: эксцентриситетом~$e$ (или относительным эксцентриситетом~$\varepsilon$) и углом нагрузки $\varphi$ (attitude angle) -- углом между линией центров и направлением результирующей силы давления.

Для описания задачи используется <<развёрнутая>> цилиндрическая система координат:
\begin{itemize}
\item $\theta$ -- окружная координата, отсчитываемая от точки максимального зазора (диаметрально противоположной точке минимального зазора) в направлении вращения вала; при этом $h(\theta = 0) = h_{\max}$ и $h(\theta = \pi) = h_{\min}$;
\item $z$ -- осевая координата вдоль оси вала, $z \in [-L/2,\; L/2]$, где $L$ -- длина подшипника (ширина рабочей поверхности втулки);
\item $y$ -- координата по толщине смазочного слоя, направленная от неподвижной втулки ($y = 0$) к поверхности вала ($y = h$).
\end{itemize}

При таком выборе начала отсчёта угла~$\theta$ выражение для толщины зазора $h(\theta)$ приобретает наиболее простую форму (см.\ далее), а угол нагрузки $\varphi$ определяется как результат расчёта:
\begin{equation}
\varphi = \operatorname{atan2}(W_t,\; W_r),
\label{eq:journal_attitude}
\end{equation}
где $W_r$ и $W_t$ -- компоненты результирующей силы давления вдоль и перпендикулярно линии центров соответственно (см.~подраздел~2.1.7).

Область расчёта представляет собой прямоугольник:
\begin{equation}
\theta \in [0,\; 2\pi], \qquad z \in [-L/2,\; L/2].
\label{eq:journal_domain}
\end{equation}

В отличие от упорного подшипника (подраздел~2.1.2), здесь область по окружной координате~$\theta$ охватывает полную окружность, что обеспечивает периодичность решения. <<Третья>> координата -- осевая~$z$ (а не радиальная~$r$, как в упорном подшипнике). Поперечное сечение подшипника с основными обозначениями показано на рисунке~\ref{fig:journal_bearing}.

% [РИСУНОК: Схема радиального подшипника]
% Должен показывать:
% - Поперечное сечение: вал радиуса R, втулка радиуса R_b = R + c
% - Центры: O_b (втулки) и O_j (вала), эксцентриситет e
% - Линия центров O_b — O_j
% - Угол θ от линии центров в направлении вращения
% - h_min = c(1-ε) и h_max = c(1+ε)
% - Направление вращения ω
% - Вектор результирующей нагрузки W и attitude angle φ
\begin{figure}[!ht]
\centering
\fbox{\parbox{0.8\textwidth}{\centering
\vspace{3cm}
ПЛЕЙСХОЛДЕР ДЛЯ РИСУНКА\\
Поперечное сечение радиального подшипника:\\
вал радиуса $R$, втулка радиуса $R_b = R + c$,\\
центры $O_b$ и $O_j$, эксцентриситет $e$,\\
$h_{\min} = c(1-\varepsilon)$, $h_{\max} = c(1+\varepsilon)$,\\
направление вращения $\omega$, нагрузка $W$, угол $\varphi$
\vspace{3cm}
}}
\caption{Поперечное сечение радиального подшипника с основными обозначениями}
\label{fig:journal_bearing}
\end{figure}

Развёрнутый вид расчётной области представлен на рисунке~\ref{fig:journal_unwrapped}: зазор <<разрезан>> вдоль образующей при $\theta = 0$ и развёрнут в прямоугольник, верхняя граница которого соответствует профилю $h(\theta)$.

% [РИСУНОК: Развёрнутый вид зазора радиального подшипника]
% Должен показывать:
% - Развёрнутый вид зазора h(θ) как прямоугольник θ × z
% - Синусоидальный профиль h(θ) сверху
% - Границы z = ±L/2
% - Обозначения ГУ на границах
\begin{figure}[!ht]
\centering
\fbox{\parbox{0.8\textwidth}{\centering
\vspace{3cm}
ПЛЕЙСХОЛДЕР ДЛЯ РИСУНКА\\
Развёрнутый вид зазора радиального подшипника:\\
область $(\theta, z)$, профиль $h(\theta) = c(1+\varepsilon\cos\theta)$,\\
граничные условия на торцах $z = \pm L/2$,\\
периодичность по $\theta$
\vspace{3cm}
}}
\caption{Развёрнутый вид расчётной области радиального подшипника}
\label{fig:journal_unwrapped}
\end{figure}


\subsubsection*{Кинематика движения поверхностей}

Вал вращается с постоянной угловой скоростью~$\omega$ внутри неподвижной втулки. Окружная скорость поверхности вала составляет:
\begin{equation}
U = \omega\,R.
\label{eq:journal_U}
\end{equation}

Осевые скорости обеих поверхностей равны нулю. В стационарном режиме положение вала фиксировано: $\varepsilon = \mathrm{const}$, и нормальные скорости поверхностей обращаются в нуль, что исключает эффект выжимания ($\partial h/\partial t = 0$). При нестационарном анализе (динамика ротора) эксцентриситет $\varepsilon$ и угол нагрузки $\varphi$ зависят от времени, и в правой части уравнения Рейнольдса появляется squeeze-term $\partial h/\partial t$; такой случай выходит за рамки настоящего раздела.

Динамическая вязкость смазки полагается постоянной: $\mu = \mathrm{const}$ (допущение~6 из подраздела~2.1.1). Влияние температурной зависимости вязкости $\mu(T)$ рассматривается в главе~4.

Вращение вала создаёт увлекающий (куэттовский) поток смазки в окружном направлении. В зоне сходящегося зазора ($0 < \theta < \pi$, где толщина уменьшается от $h_{\max}$ к $h_{\min}$) условие неразрывности приводит к росту давления. В зоне расходящегося зазора ($\pi < \theta < 2\pi$) давление стремится к отрицательным значениям, однако физически отрицательное давление не реализуется: жидкость разрывается и образуется кавитационная зона. Именно сочетание клинового эффекта и кавитации определяет распределение давления и несущую способность радиального подшипника.


\subsubsection*{Закон толщины смазочного слоя}

Толщина смазочного слоя определяется геометрией двух эксцентричных цилиндров -- вала и втулки. При условии тонкого зазора ($c/R \ll 1$, что согласуется с допущением~3 из подраздела~2.1.1) и пренебрежении членами порядка $(c/R)^2$ и выше, толщина зазора записывается в виде:
\begin{equation}
h(\theta) = c\,(1 + \varepsilon\,\cos\theta),
\label{eq:journal_h}
\end{equation}
где $c = R_b - R$ -- радиальный зазор, $\varepsilon = e/c$ -- относительный эксцентриситет.

Из формулы~\eqref{eq:journal_h} следует:
\begin{itemize}
\item максимальная толщина зазора $h_{\max} = c(1 + \varepsilon)$ достигается при $\theta = 0$ (точка, диаметрально противоположная минимальному зазору);
\item минимальная толщина зазора $h_{\min} = c(1 - \varepsilon)$ достигается при $\theta = \pi$ (точка наибольшего сближения вала и втулки).
\end{itemize}

Толщина зазора~\eqref{eq:journal_h} не зависит от осевой координаты~$z$, что соответствует осевой однородности геометрии (вал и втулка имеют цилиндрическую форму с параллельными осями). Зависимость $h$ от $z$ может возникать при перекосе вала (misalignment), однако этот случай в настоящей работе не рассматривается.

В общем случае, при наличии детерминированных микроструктур на рабочей поверхности, толщина смазочного слоя записывается как суперпозиция:
\begin{equation}
h(\theta, z) = c\,(1 + \varepsilon\,\cos\theta) + \Delta h(\theta, z),
\label{eq:journal_h_textured}
\end{equation}
где $\Delta h(\theta, z)$ -- добавка, описывающая геометрию микроструктуры (раздел~2.2).

Для характеристики геометрии радиального подшипника вводятся следующие безразмерные параметры:
\begin{itemize}
\item $\varepsilon = e/c$ -- относительный эксцентриситет, являющийся основным параметром нагружения ($\varepsilon = 0$ соответствует концентричному расположению, $\varepsilon \to 1$ -- контакту поверхностей);
\item $\psi = c/R$ -- относительный зазор (для типичных подшипников $\psi \sim 10^{-3}$);
\item $L/D$ -- отношение длины подшипника к диаметру ($D = 2R$), определяющее степень влияния осевых утечек: при $L/D > 1$ подшипник считается длинным, при $L/D < 0{,}5$ -- коротким, промежуточные значения соответствуют подшипнику конечной длины.
\end{itemize}


\subsubsection*{Уравнение Рейнольдса для радиального подшипника}

Обобщённое уравнение Рейнольдса~\eqref{eq:reynolds_full}, полученное в подразделе~2.1.1 в декартовых координатах, необходимо переписать для цилиндрической поверхности зазора радиального подшипника. Введём дуговую координату $x = R\theta$ (расстояние вдоль окружности вала), вдоль которой производная принимает вид $\partial/\partial x = (1/R)\,\partial/\partial\theta$. Осевая координата~$z$ остаётся без изменений.

В рассматриваемой задаче втулка неподвижна ($U_1 = 0$), вал движется с окружной скоростью $U_2 = \omega R = U$; осевые скорости обеих поверхностей равны нулю ($W_1 = W_2 = 0$). В стационарном режиме ($\partial h/\partial t = 0$) правая часть обобщённого уравнения Рейнольдса содержит единственный член:
\begin{equation}
\frac{U_1 + U_2}{2}\,\frac{1}{R}\,\frac{\partial h}{\partial \theta} = \frac{U}{2R}\,\frac{\partial h}{\partial \theta} = \frac{\omega}{2}\,\frac{\partial h}{\partial \theta}.
\label{eq:journal_rhs_derivation}
\end{equation}

Стационарное уравнение Рейнольдса для несжимаемой ньютоновской жидкости в координатах $(\theta, z)$ принимает вид:
\begin{equation}
\frac{1}{R^2}\frac{\partial}{\partial \theta}\!\left[\frac{h^3}{12\mu}\,\frac{\partial p}{\partial \theta}\right] + \frac{\partial}{\partial z}\!\left[\frac{h^3}{12\mu}\,\frac{\partial p}{\partial z}\right] = \frac{\omega}{2}\,\frac{\partial h}{\partial \theta}.
\label{eq:journal_reynolds}
\end{equation}

Множитель $1/R^2$ перед первым слагаемым в левой части обусловлен двукратным применением оператора $\partial/\partial x = (1/R)\,\partial/\partial\theta$ при переходе от декартовой координаты~$x$ к угловой~$\theta$.

Умножая обе части уравнения~\eqref{eq:journal_reynolds} на~$12\mu$ (что допустимо при $\mu = \mathrm{const}$), получаем эквивалентную форму, удобную для численной реализации:
\begin{equation}
\frac{1}{R^2}\frac{\partial}{\partial \theta}\!\left[h^3\,\frac{\partial p}{\partial \theta}\right] + \frac{\partial}{\partial z}\!\left[h^3\,\frac{\partial p}{\partial z}\right] = 6\mu\omega\,\frac{\partial h}{\partial \theta}.
\label{eq:journal_reynolds_num}
\end{equation}

Уравнение~\eqref{eq:journal_reynolds_num} является основным расчётным уравнением для радиального подшипника в настоящей работе.

Подставляя конкретное выражение для толщины зазора~\eqref{eq:journal_h}, получаем производную в правой части:
\begin{equation}
\frac{\partial h}{\partial \theta} = -c\,\varepsilon\,\sin\theta.
\label{eq:journal_dh_dtheta}
\end{equation}

Правая часть уравнения~\eqref{eq:journal_reynolds_num} принимает вид $-6\mu\omega\,c\,\varepsilon\,\sin\theta$. Источниковый член в правой части пропорционален $-\sin\theta$, поэтому максимум генерации давления приходится на $\theta \approx \pi/2$, где скорость сужения зазора наибольшая. В зоне $0 < \theta < \pi$ выполняется $\partial h/\partial\theta < 0$ (сходящийся клин): зазор уменьшается по направлению вращения, и давление генерируется. В зоне $\pi < \theta < 2\pi$ выполняется $\partial h/\partial\theta > 0$ (расходящийся клин): зазор увеличивается, давление стремится к отрицательным значениям, и вступает в действие кавитация.

Левая часть уравнений~\eqref{eq:journal_reynolds}--\eqref{eq:journal_reynolds_num} представляет собой эллиптический оператор $\nabla \cdot (h^3 \nabla p)$, записанный в координатах $(\theta, z)$ на цилиндрической поверхности. По структуре он аналогичен оператору в уравнении для упорного подшипника~\eqref{eq:thrust_reynolds_num}, однако здесь вместо координат $(r, \theta)$ используются $(\theta, z)$. Правая часть -- источниковый член, пропорциональный $\sin\theta$: максимум генерации давления приходится на $\theta = \pi/2$. При $\varepsilon = 0$ (концентричное расположение вала и втулки) $\partial h/\partial \theta = 0$, правая часть обращается в нуль и давление не генерируется -- подшипник не несёт нагрузки.

Уравнение~\eqref{eq:journal_reynolds_num} допускает два классических предельных случая, для которых известны аналитические решения.

\textit{Бесконечно длинный подшипник ($L/D \to \infty$).} В этом пределе давление не зависит от осевой координаты: $\partial p/\partial z = 0$, и уравнение~\eqref{eq:journal_reynolds_num} вырождается в обыкновенное дифференциальное уравнение по~$\theta$. Полное аналитическое решение этого ОДУ с периодическими граничными условиями и без учёта кавитации известно как решение Зоммерфельда (Sommerfeld, 1904). Учёт кавитации приводит к половинному решению (half-Sommerfeld), в котором давление обнуляется в зоне расходящегося зазора.

\textit{Бесконечно короткий подшипник ($L/D \to 0$).} В этом пределе осевой градиент давления значительно превышает окружной: $\partial^2 p/\partial z^2 \gg (1/R^2)\,\partial^2 p/\partial \theta^2$, и окружной член в левой части уравнения~\eqref{eq:journal_reynolds_num} может быть опущен. Уравнение вырождается в ОДУ по~$z$ при каждом фиксированном~$\theta$, что допускает аналитическое решение. Это приближение известно как приближение Дюбуа--Оквирка (Dubois--Ocvirk, short-bearing approximation) и даёт удовлетворительные результаты при $L/D < 0{,}5$.

Для подшипника конечной длины ($0{,}5 \lesssim L/D \lesssim 2$) ни один из предельных случаев не обеспечивает достаточной точности, и уравнение~\eqref{eq:journal_reynolds_num} решается численно (глава~3).


\subsubsection*{Граничные условия}

Для замыкания уравнения Рейнольдса~\eqref{eq:journal_reynolds_num} необходимо задать граничные условия на давление по всему контуру расчётной области~\eqref{eq:journal_domain}.

\textit{По окружной координате~$\theta$.} Область по~$\theta$ охватывает полную окружность, поэтому естественным условием является периодичность:
\begin{equation}
p(0, z) = p(2\pi, z), \qquad \frac{\partial p}{\partial \theta}\bigg|_{\theta=0} = \frac{\partial p}{\partial \theta}\bigg|_{\theta=2\pi}.
\label{eq:journal_bc_periodic}
\end{equation}

\textit{По осевой координате~$z$.} На торцах подшипника давление принимается равным атмосферному:
\begin{equation}
p(\theta,\,-L/2) = 0, \qquad p(\theta,\,L/2) = 0,
\label{eq:journal_bc_axial}
\end{equation}
где давление отсчитывается от атмосферного ($p = p_{\mathrm{abs}} - p_{\mathrm{atm}}$), как и в подразделе~2.1.2. При наличии принудительной подачи смазки через осевую канавку условие на соответствующем угле $\theta = \theta_{\mathrm{groove}}$ модифицируется: $p(\theta_{\mathrm{groove}}, z) = p_{\mathrm{sup}}$, где $p_{\mathrm{sup}}$ -- давление подачи. Помимо условий Дирихле на торцах, в специальных конструкциях могут применяться условия непротекания $q_z = 0$, эквивалентные $\partial p/\partial z = 0$ (при наличии уплотнений на торцах подшипника), или смешанные условия, когда на части окружности задаётся давление подачи, а на остальной -- атмосферное давление.

\textit{Кавитация.} Принципиальной особенностью радиального подшипника является неизбежное возникновение кавитации в зоне расходящегося зазора ($\pi < \theta < 2\pi$), где расчётное давление стремится к отрицательным значениям. В простейшей постановке применяют ограничение $p \geq 0$ (условие Гюмбеля), при котором отрицательные давления обнуляются после решения. Более физичным является условие Рейнольдса (Swift--Stieber), требующее одновременного выполнения $p = 0$ и $\partial p/\partial \theta = 0$ на границе кавитации. Оба подхода, однако, не обеспечивают точного сохранения массы. Корректная массо-сохраняющая постановка, основанная на модели Jakobsson--Floberg--Olsson (JFO), вводится в разделе~2.3.

Граница кавитационной зоны является внутренней границей задачи, положение которой заранее неизвестно и определяется в процессе решения. Периодичность задаётся как граничное условие на контуре области~\eqref{eq:journal_domain}, а кавитационная зона является внутренней частью решения. Само поле давления $p(\theta, z)$ при этом остаётся непрерывным и периодическим, хотя и состоит из двух участков: зоны полного смазочного слоя ($p > 0$) и зоны кавитации ($p = 0$). Качественное распределение давления $p(\theta)$ в среднем сечении подшипника ($z = 0$) показано на рисунке~\ref{fig:journal_pressure_profile}.

% [РИСУНОК: Качественный профиль давления в радиальном подшипнике]
% Должен показывать:
% - Ось абсцисс: θ от 0 до 2π
% - p(θ) при z=0: рост в зоне 0 < θ < π, максимум вблизи θ = π/2
% - Зона кавитации: p = 0 при π < θ < 2π (приблизительно)
% - Границы разрыва и восстановления
% - Пунктир: полное решение Зоммерфельда (без кавитации, отрицательные давления)
\begin{figure}[!ht]
\centering
\fbox{\parbox{0.8\textwidth}{\centering
\vspace{3cm}
ПЛЕЙСХОЛДЕР ДЛЯ РИСУНКА\\
Качественное распределение давления $p(\theta)$ при $z=0$:\\
зона генерации давления ($0 < \theta < \pi$),\\
зона кавитации ($p=0$),\\
границы разрыва и восстановления,\\
пунктир -- полное решение Зоммерфельда (без кавитации)
\vspace{3cm}
}}
\caption{Качественное распределение давления в радиальном подшипнике и зона кавитации}
\label{fig:journal_pressure_profile}
\end{figure}


\subsubsection*{Объёмные расходы}

Объёмные расходы жидкости через единицу длины определяются путём интегрирования профилей скорости по толщине смазочного слоя, аналогично подразделу~2.1.2.

Окружной расход (на единицу длины по осевой координате~$z$):
\begin{equation}
q_\theta(\theta, z) = \frac{h\,U}{2} - \frac{h^3}{12\mu\,R}\,\frac{\partial p}{\partial \theta} = \frac{h\,\omega R}{2} - \frac{h^3}{12\mu\,R}\,\frac{\partial p}{\partial \theta}.
\label{eq:journal_q_theta}
\end{equation}

Первое слагаемое представляет куэттовскую составляющую расхода, обусловленную увлечением жидкости вращающимся валом; второе -- пуазейлевскую составляющую, определяемую окружным градиентом давления. Множитель $1/R$ во втором слагаемом обусловлен переходом от производной по дуге $\partial/\partial(R\theta)$ к производной по углу $\partial/\partial\theta$.

Осевой расход (на единицу длины по окружной координате):
\begin{equation}
q_z(\theta, z) = -\frac{h^3}{12\mu}\,\frac{\partial p}{\partial z}.
\label{eq:journal_q_z}
\end{equation}

Поскольку осевые скорости обеих поверхностей равны нулю, осевой расход полностью определяется градиентом давления (пуазейлевская составляющая). Именно осевой расход обусловливает боковую утечку смазки через торцы подшипника.

Суммарная утечка через торцы подшипника определяется интегрированием осевого расхода по окружности:
\begin{equation}
Q_{\mathrm{side}} = \int_0^{2\pi} \bigl[ q_z(\theta,\,L/2) - q_z(\theta,\,-L/2) \bigr]\,R\,\mathrm{d}\theta.
\label{eq:journal_Q_side}
\end{equation}

Множитель $R\,\mathrm{d}\theta$ соответствует элементу длины дуги на окружности вала. Здесь $q_z(\theta, L/2) > 0$ соответствует потоку, вытекающему через торец $z = L/2$ (по направлению внешней нормали $+\mathbf{e}_z$), а $q_z(\theta, -L/2) < 0$ -- потоку, вытекающему через торец $z = -L/2$ (нормаль $-\mathbf{e}_z$). Обозначая утечки через отдельные торцы как $Q_+ = \int_0^{2\pi} q_z(\theta, L/2)\,R\,\mathrm{d}\theta$ и $Q_- = -\int_0^{2\pi} q_z(\theta, -L/2)\,R\,\mathrm{d}\theta$, получаем $Q_{\mathrm{side}} = Q_+ + Q_-$.

При симметрии давления относительно средней плоскости подшипника ($p(\theta, z) = p(\theta, -z)$) утечки через оба торца одинаковы, и выражение~\eqref{eq:journal_Q_side} упрощается:
\begin{equation}
Q_{\mathrm{side}} = 2\int_0^{2\pi} q_z(\theta,\,L/2)\,R\,\mathrm{d}\theta.
\label{eq:journal_Q_side_sym}
\end{equation}

Боковая утечка является основным механизмом потери смазки в радиальном подшипнике. При уменьшении отношения $L/D$ доля давления, <<разгружаемая>> через торцы, возрастает, что снижает несущую способность. В предельном случае бесконечно короткого подшипника ($L/D \to 0$) осевые утечки полностью доминируют, и именно это обстоятельство позволяет пренебречь окружным членом в уравнении Рейнольдса (приближение Дюбуа--Оквирка).

Баланс массы в стационарном режиме без выжимания записывается в дифференциальной форме (аналогично уравнению~\eqref{eq:thrust_mass_balance} для упорного подшипника):
\begin{equation}
\frac{1}{R}\,\frac{\partial q_\theta}{\partial \theta} + \frac{\partial q_z}{\partial z} = 0.
\label{eq:journal_mass_balance}
\end{equation}

Подстановка выражений~\eqref{eq:journal_q_theta} и~\eqref{eq:journal_q_z} в уравнение~\eqref{eq:journal_mass_balance} приводит к уравнению Рейнольдса~\eqref{eq:journal_reynolds}.


Решение уравнения Рейнольдса даёт поле давления~$p$ в расчётной области. По найденному давлению вычисляются интегральные характеристики узла: несущая способность, расходы смазки, силы и моменты трения, минимальная толщина плёнки. Унифицированные формулы для всех выходных величин приведены в подразделе~2.1.7.

В практической постановке задачи внешняя радиальная нагрузка $\mathbf{W}_{\mathrm{ext}}$ и параметры режима ($\omega$, $\mu$, $c$, $R$, $L$) считаются заданными, тогда как положение вала $(\varepsilon, \varphi)$ является неизвестным. Равновесное положение определяется из условия статического баланса сил: $\mathbf{W}(\varepsilon, \varphi) = -\mathbf{W}_{\mathrm{ext}}$, т.\,е. из системы двух нелинейных уравнений относительно $\varepsilon$ и $\varphi$. Численное решение этой системы методом итерационного поиска описывается в главе~3.

Сформулированная задача сводится к решению эллиптического уравнения Рейнольдса~\eqref{eq:journal_reynolds_num} в прямоугольной области~\eqref{eq:journal_domain} с периодическими условиями по окружной координате~$\theta$ и условиями Дирихле по осевой координате~$z$. Принципиальной особенностью радиального подшипника, отличающей его от упорного (подраздел~2.1.2), является неизбежное возникновение кавитации в зоне расходящегося зазора: при любом ненулевом эксцентриситете $\varepsilon > 0$ правая часть уравнения Рейнольдса меняет знак, и в отсутствие ограничений на давление решение содержит области с $p < 0$.

Численная дискретизация уравнения~\eqref{eq:journal_reynolds_num} методом конечных разностей и алгоритм учёта кавитации на основе модели JFO рассматриваются в главе~3. При наличии микроструктур на рабочей поверхности функция зазора~\eqref{eq:journal_h_textured} приобретает быстрые локальные вариации, параметризация которых описывается в разделе~2.2.
