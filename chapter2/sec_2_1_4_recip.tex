\subsection{Возвратно-поступательная постановка задачи}

Рассмотрим узел трения с осевым возвратно-поступательным относительным движением поверхностей: плунжер (поршень, шток) совершает возвратно-поступательное перемещение внутри цилиндра (втулки). Тонкий слой вязкой жидкости заполняет кольцевой зазор между сопряжёнными цилиндрическими поверхностями и выполняет функцию смазки и уплотнения.

Узлы данного типа широко распространены в машиностроении. В плунжерных насосах высокого давления зазор между плунжером и гильзой одновременно уплотняет рабочую жидкость и обеспечивает смазку. В гидроцилиндрах поршень перемещается внутри гильзы под действием давления рабочей жидкости. В поршневых компрессорах и двигателях внутреннего сгорания поршень с кольцами образует трибологическую пару скольжения с гильзой цилиндра. В направляющих скольжения металлорежущих станков возвратно-поступательное движение суппорта или стола сопровождается формированием смазочного слоя.

Ключевое отличие рассматриваемой постановки от подразделов~2.1.2--2.1.3 состоит в следующем. Скорость увлечения $U(t)$ меняет знак при реверсе хода, вследствие чего клиновой член в правой части уравнения Рейнольдса меняет знак: зона повышенного давления <<переезжает>> от одного конца рабочей зоны к другому. При каждом реверсе возможны разрыв и последующее восстановление сплошности смазочного слоя. Второе существенное отличие -- нестационарный squeeze-член $\partial h/\partial t$ зачастую играет определяющую роль: колебания эксцентриситета, вибрации и перекос плунжера приводят к изменению зазора во времени, что генерирует давление даже при отсутствии клинового эффекта. Все общие допущения, принятые при выводе уравнения Рейнольдса, наследуются из подраздела~2.1.1.


\subsubsection*{Геометрия и система координат}

Основными геометрическими параметрами узла являются:
\begin{itemize}
\item $R$ -- радиус плунжера;
\item $R_b$ -- радиус внутренней поверхности цилиндра (втулки);
\item $c = R_b - R$ -- радиальный зазор;
\item $L$ -- длина рабочей зоны контакта (длина втулки);
\item $e(t)$ -- эксцентриситет (расстояние между осями плунжера и цилиндра), в общем случае зависящий от времени;
\item $\varepsilon(t) = e(t)/c$ -- относительный эксцентриситет;
\item $\varphi(t)$ -- угол положения линии центров.
\end{itemize}

Для описания задачи используется цилиндрическая система координат $(\theta, z, y)$:
\begin{itemize}
\item $z$ -- ось движения плунжера (вдоль хода), $z \in [0,\; L]$;
\item $\theta$ -- окружная координата, $\theta \in [0,\; 2\pi]$;
\item $y$ -- координата по толщине смазочного слоя, направленная от поверхности цилиндра ($y = 0$) к поверхности плунжера ($y = h$).
\end{itemize}

Область расчёта представляет собой прямоугольник:
\begin{equation}
\theta \in [0,\; 2\pi], \qquad z \in [0,\; L].
\label{eq:recip_domain}
\end{equation}

В отличие от радиального подшипника (подраздел~2.1.3), где основное течение является окружным ($\theta$-направление), вызванным вращением вала, а утечка -- осевой ($z$), в возвратно-поступательном узле картина обратная: основное течение направлено вдоль оси $z$ и вызвано поступательным движением плунжера, а окружное течение является вторичным, обусловленным неоднородностью давления по~$\theta$.

% [РИСУНОК: Схема возвратно-поступательного узла]
\begin{figure}[!ht]
\centering
\fbox{\parbox{0.8\textwidth}{\centering
\vspace{3cm}
ПЛЕЙСХОЛДЕР ДЛЯ РИСУНКА\\
Схема возвратно-поступательного узла (плунжер--втулка):\\
продольный разрез, координаты $z$ (вдоль хода), $\theta$ (по окружности),\\
$y$ (по толщине зазора), скорость $U(t)$ с двойной стрелкой,\\
длина рабочей зоны $L$, зазор $h(\theta, z, t)$
\vspace{3cm}
}}
\caption{Схема возвратно-поступательного узла: плунжер внутри цилиндрической втулки}
\label{fig:recip_scheme}
\end{figure}


\subsubsection*{Кинематика движения}

Закон хода плунжера задаётся функцией положения $s(t)$; скорость поступательного движения определяется как
\begin{equation}
U(t) = \frac{ds}{dt}.
\label{eq:recip_U}
\end{equation}

Типичный гармонический закон хода, удобный для верификации численных схем:
\begin{equation}
s(t) = \frac{S}{2}\bigl(1 - \cos(\Omega t)\bigr), \qquad
U(t) = \frac{S\,\Omega}{2}\sin(\Omega t),
\label{eq:recip_harmonic}
\end{equation}
где $S$ -- полный ход (stroke), $\Omega$ -- угловая частота возвратно-поступательного движения. В реальных механизмах закон $s(t)$ может определяться кинематикой кривошипно-шатунного механизма, однако для модели достаточно задать $U(t)$ произвольной функцией времени.


\subsubsection*{Закон толщины смазочного слоя}

Базовая геометрия описывается двумя эксцентричными цилиндрами -- плунжером и втулкой, аналогично радиальному подшипнику (подраздел~2.1.3):
\begin{equation}
h(\theta, t) = c\bigl(1 + \varepsilon(t)\cos(\theta - \varphi(t))\bigr).
\label{eq:recip_h_base}
\end{equation}

Для возникновения клинового эффекта вдоль оси~$z$ необходима неоднородность $\partial h/\partial z \neq 0$. Источниками такой неоднородности являются: перекос (tilt) плунжера относительно цилиндра, конусность или микропрофиль поверхности, а также специально введённый рельеф (микротекстура, см.\ раздел~2.2).

Общий вид толщины зазора с учётом осевого наклона:
\begin{equation}
h(\theta, z, t) = c\bigl(1 + \varepsilon(t)\cos(\theta - \varphi(t))\bigr) + \kappa\Bigl(z - \frac{L}{2}\Bigr),
\label{eq:recip_h_tilt}
\end{equation}
где $\kappa$ -- параметр малого осевого наклона плунжера. При $\kappa = 0$ клин вдоль $z$ отсутствует; давление может генерироваться только через squeeze-член ($\partial h/\partial t \neq 0$).

При наличии детерминированных микроструктур на рабочей поверхности толщина зазора записывается как суперпозиция:
\begin{equation}
h(\theta, z, t) = h_{\mathrm{base}}(\theta, z, t) + \Delta h(\theta, z),
\label{eq:recip_h_textured}
\end{equation}
где $\Delta h(\theta, z)$ -- добавка, описывающая геометрию микроструктуры (подробно в разделе~2.2).


\subsubsection*{Уравнение Рейнольдса}

На цилиндрической поверхности зазора в координатах $(\theta, z)$ при осевом увлечении со скоростью $U(t)$ и отсутствии окружного увлечения уравнение Рейнольдса для несжимаемой ньютоновской жидкости принимает вид:
\begin{equation}
\frac{1}{R^2}\frac{\partial}{\partial\theta}
  \biggl(\frac{h^3}{12\mu}\frac{\partial p}{\partial\theta}\biggr)
+ \frac{\partial}{\partial z}
  \biggl(\frac{h^3}{12\mu}\frac{\partial p}{\partial z}\biggr)
= \frac{U(t)}{2}\frac{\partial h}{\partial z}
+ \frac{\partial h}{\partial t}.
\label{eq:recip_reynolds}
\end{equation}

Умножая обе части на~$12\mu$, получаем эквивалентную форму, удобную для численной реализации:
\begin{equation}
\frac{1}{R^2}\frac{\partial}{\partial\theta}
  \biggl(h^3\frac{\partial p}{\partial\theta}\biggr)
+ \frac{\partial}{\partial z}
  \biggl(h^3\frac{\partial p}{\partial z}\biggr)
= 6\mu\,U(t)\frac{\partial h}{\partial z}
+ 12\mu\frac{\partial h}{\partial t}.
\label{eq:recip_reynolds_num}
\end{equation}

Правая часть уравнения~\eqref{eq:recip_reynolds_num} содержит два слагаемых различной физической природы.

Первый член $6\mu\,U(t)\,\partial h/\partial z$ описывает клиновой эффект: давление генерируется при наличии сужения зазора вдоль направления движения ($\partial h/\partial z \neq 0$). При реверсе хода ($U < 0$) знак этого члена меняется на противоположный, и область повышенного давления <<переезжает>> от одного конца рабочей зоны к другому. Именно этот механизм определяет несущую способность клинового зазора.

% [РИСУНОК: Реверс зоны давления при смене знака U(t)]
\begin{figure}[!ht]
\centering
\fbox{\parbox{0.8\textwidth}{\centering
\vspace{3.5cm}
ПЛЕЙСХОЛДЕР ДЛЯ РИСУНКА\\
Реверс зоны давления при смене знака скорости:\\
(a)~прямой ход ($U > 0$): область повышенного давления\\
\hphantom{(a)~}у правого конца рабочей зоны;\\
(b)~обратный ход ($U < 0$): область повышенного давления\\
\hphantom{(b)~}у левого конца рабочей зоны.\\
Клин вдоль~$z$ обусловлен наклоном~$\kappa$
\vspace{3.5cm}
}}
\caption{Смена направления скорости $U(t)$ приводит к перемещению зоны повышенного давления: (a)~прямой ход ($U>0$); (b)~обратный ход ($U<0$). Клин вдоль~$z$ обусловлен наклоном~$\kappa$}
\label{fig:recip_reversal}
\end{figure}

Второй член $12\mu\,\partial h/\partial t$ описывает squeeze-эффект (эффект выжимания): давление генерируется при сближении поверхностей ($\partial h/\partial t < 0$), даже если клиновой зазор вдоль $z$ отсутствует. Этот механизм важен при колебаниях эксцентриситета, вибрациях и перекосе плунжера. При удалении поверхностей ($\partial h/\partial t > 0$) данный член стремится понизить давление, что способствует возникновению кавитации.

По структуре уравнение~\eqref{eq:recip_reynolds_num} аналогично уравнениям для упорного~\eqref{eq:thrust_reynolds_num} и радиального~\eqref{eq:journal_reynolds_num} подшипников: левая часть содержит эллиптический оператор $\nabla \cdot (h^3 \nabla p)$, а правая -- источниковые члены. Отличие состоит в направлении увлечения (осевое вместо окружного) и в явной нестационарности правой части.


\subsubsection*{Граничные условия}

\textit{По окружной координате~$\theta$.} Область по~$\theta$ охватывает полную окружность, поэтому естественным условием является периодичность:
\begin{equation}
p(0, z, t) = p(2\pi, z, t), \qquad
\frac{\partial p}{\partial\theta}\bigg|_{\theta=0}
= \frac{\partial p}{\partial\theta}\bigg|_{\theta=2\pi}.
\label{eq:recip_bc_periodic}
\end{equation}

\textit{По осевой координате~$z$.} Рассматриваются три варианта.

\textit{(a) Открытые торцы (атмосферные кромки, базовый вариант):}
\begin{equation}
p(\theta, 0, t) = 0, \qquad p(\theta, L, t) = 0.
\label{eq:recip_bc_open}
\end{equation}

Давление отсчитывается от атмосферного ($p = p_{\mathrm{abs}} - p_{\mathrm{atm}}$), как и в подразделах~2.1.2--2.1.3. Данный вариант соответствует свободным кромкам втулки, открытым в окружающую среду.

\textit{(b) Давление подачи/слива (гидросистема):}
\begin{equation}
p(\theta, 0, t) = p_{\mathrm{in}}(t), \qquad p(\theta, L, t) = p_{\mathrm{out}}(t).
\label{eq:recip_bc_pressure}
\end{equation}

Такая постановка типична для гидроцилиндров, где полости с обеих сторон поршня находятся под рабочим давлением.

\textit{(c) Уплотнённые торцы (отсутствие протекания):}
\begin{equation}
\frac{\partial p}{\partial z}\bigg|_{z=0} = 0, \qquad
\frac{\partial p}{\partial z}\bigg|_{z=L} = 0.
\label{eq:recip_bc_sealed}
\end{equation}

Условие непротекания $q_z = 0$ на торцах эквивалентно $\partial p/\partial z = 0$ и применяется при наличии уплотнительных элементов на кромках втулки.

\textit{Кавитация.} При реверсе скорости и в расходящихся участках зазора давление стремится стать отрицательным. Простое ограничение $p \geq 0$ (условие Гюмбеля) нарушает баланс массы на границе кавитационной зоны. Корректный массо-сохраняющий учёт кавитации обеспечивается моделью JFO, вводимой в разделе~2.3.


\subsubsection*{Объёмные расходы}

Удельные потоки (расходы на единицу длины) определяются путём интегрирования профилей скорости по толщине смазочного слоя, аналогично подразделам~2.1.2--2.1.3.

Осевой расход (на единицу длины по окружной координате):
\begin{equation}
q_z(\theta, z, t) = \frac{h\,U(t)}{2}
  - \frac{h^3}{12\mu}\frac{\partial p}{\partial z}.
\label{eq:recip_q_z}
\end{equation}

Первое слагаемое представляет куэттовскую составляющую, обусловленную движением плунжера; второе -- пуазейлевскую составляющую, определяемую осевым градиентом давления.

Окружной расход (на единицу длины по осевой координате):
\begin{equation}
q_\theta(\theta, z, t) = -\frac{h^3}{12\mu R}\frac{\partial p}{\partial\theta}.
\label{eq:recip_q_theta}
\end{equation}

Поскольку окружные скорости обеих поверхностей равны нулю (плунжер совершает только поступательное движение), куэттовская составляющая в окружном направлении отсутствует. Окружной расход полностью определяется градиентом давления.

Полная утечка через торцы рабочей зоны:
\begin{equation}
Q_{\mathrm{out}}(t) = \int_0^{2\pi} q_z(\theta,\,L,\,t)\,R\,\mathrm{d}\theta, \qquad
Q_{\mathrm{in}}(t) = \int_0^{2\pi} q_z(\theta,\,0,\,t)\,R\,\mathrm{d}\theta.
\label{eq:recip_Q_leakage}
\end{equation}

Знаки определяются направлением внешней нормали, как и в подразделах~2.1.2--2.1.3: положительный $Q_{\mathrm{out}}$ соответствует вытеканию смазки из рабочей зоны через торец $z = L$.

Решение уравнения Рейнольдса даёт поле давления $p(\theta, z, t)$. По найденному давлению вычисляются интегральные характеристики узла: несущая способность, расходы смазки, силы и моменты трения, минимальная толщина плёнки. Унифицированные формулы для всех выходных величин приведены в подразделе~2.1.7.

Для возвратно-поступательных узлов микротекстурирование поверхностей рассматривается как способ повышения несущей способности и снижения утечек; параметризация микрорельефа вводится в разделе~2.2.
