\subsection{Типы геометрии микроструктур}

В подразделе~2.2.1 введён общий шаблон элемента микрорельефа:
$\Delta h_k = \sigma_k\, h_{p}\, f_k(\ldots)$ на области~$D_k$.
Цель настоящего подраздела -- задать конкретные профильные
функции~$f$ и формы областей~$D_k$ для всех типов элементов,
используемых в работе. Общие требования к профильной функции
формулируются следующим образом:
\begin{itemize}
  \item $f(0,0) = 1$ (или $f(0) = 1$ для осесимметричных элементов);
  \item $f = 0$ на границе~$D_k$;
  \item желательна непрерывность~$f$ и гладкая стыковка на границе:
        для профилей вида $f(\rho)$ -- $df/d\rho|_{\rho=1} = 0$,
        в общем случае -- $\partial f/\partial n = 0$ на~$\partial D_k$
        (где $n$ -- внешняя нормаль);
  \item $0 \leq f \leq 1$ внутри~$D_k$.
\end{itemize}

%% ===== Блок 1: Локальные координаты =====

\subsubsection*{Локальные координаты элемента}

Для описания элементов произвольной ориентации вводятся локальные
координаты $(\xi, \eta)$, связанные с глобальными $(x_1, x_2)$
поворотом на угол~$\alpha_k$:
\begin{equation}
  \begin{aligned}
    \xi  &= \phantom{-}(x_1 - x_{1,k})\cos\alpha_k
            + (x_2 - x_{2,k})\sin\alpha_k, \\
    \eta &= -(x_1 - x_{1,k})\sin\alpha_k
            + (x_2 - x_{2,k})\cos\alpha_k.
  \end{aligned}
  \label{eq:local_coords}
\end{equation}
При $\alpha_k = 0$ локальные координаты совпадают с глобальными.

Безразмерная радиальная координата для эллиптических элементов
(введённая в подразделе~2.2.1) записывается через локальные
координаты:
\[
  \rho_k = \sqrt{\left(\frac{\xi}{b_k}\right)^{\!2}
              +\left(\frac{\eta\vphantom{\xi}}{a_k}\right)^{\!2}}.
\]
Область элемента: $D_k = \{\rho_k \leq 1\}$. Для канавок
и прямоугольных элементов область~$D_k$ задаётся иначе (см.\ ниже).

% [РИСУНОК: Локальные координаты элемента]
\begin{figure}[!ht]
\centering
\fbox{\parbox{0.8\textwidth}{\centering
\vspace{3cm}
ПЛЕЙСХОЛДЕР ДЛЯ РИСУНКА\\
Элемент микроструктуры: вид сверху\\
Локальные оси $\xi$, $\eta$; глобальные $x_1$, $x_2$;\\
угол поворота $\alpha$; граница $\rho = 1$
\vspace{3cm}
}}
\caption{Локальные координаты элемента микроструктуры и угол поворота}
\label{fig:local_coords}
\end{figure}

%% ===== Блок 2: Осесимметричные и эллиптические профили =====

\subsubsection*{Осесимметричные и эллиптические элементы}

Для элементов с эллиптической областью $D_k = \{\rho_k \leq 1\}$
профильная функция зависит только от~$\rho_k$ (далее для краткости
$\rho$). Ниже приведены основные типы профилей.

\paragraph{A) Эллипсоидальный (полусфероидальная шапка).}
\[
  f(\rho) = \sqrt{1 - \rho^2}.
\]
Свойства: $f(1) = 0$, но $f'(1) \to -\infty$ (вертикальный обрыв
на границе). При $a = b$ профиль соответствует полусфере. Широко
используется в литературе; на практике разрыв производной обычно
не вызывает проблем при достаточном разрешении сетки.

\paragraph{B) Параболический.}
\[
  f(\rho) = 1 - \rho^2.
\]
Свойства: $f(1) = 0$, $f'(1) = -2 \neq 0$. Простейший гладкий
профиль. При $a = b$ -- параболоид вращения.

\paragraph{C) Конический (линейный).}
\[
  f(\rho) = 1 - \rho.
\]
Свойства: $f(1) = 0$, $df/d\rho|_{\rho=1} = -1$. Как функция
от~$(\xi, \eta)$ через~$\rho$ имеет недифференцируемость в центре
($\rho = 0$), что может приводить к численным артефактам на грубых
сетках.

\paragraph{D) Косинусный.}
\[
  f(\rho) = \tfrac{1}{2}\bigl(1 + \cos(\pi\rho)\bigr).
\]
Свойства: $f(1) = 0$, $f'(1) = 0$. Обеспечивает гладкую стыковку
на границе элемента. Предпочтителен для численных расчётов.

\paragraph{E) Полиномиальный (smoothstep).}
\[
  f(\rho) = 1 - 3\rho^2 + 2\rho^3.
\]
Свойства: $f(1) = 0$, $f'(1) = 0$, $f'(0) = 0$. Двусторонняя
гладкость -- и в центре, и на границе.

\paragraph{F) Цилиндрический (плоское дно).}
Вводится параметр $\rho_0$ ($0 < \rho_0 < 1$) -- радиус плоского
участка:
\begin{equation}
  f(\rho) =
    \begin{cases}
      1,
        & 0 \leq \rho \leq \rho_0, \\[4pt]
      g\!\left(\dfrac{\rho - \rho_0}{1 - \rho_0}\right),
        & \rho_0 < \rho \leq 1,
    \end{cases}
  \label{eq:profile_plateau}
\end{equation}
где $g(t)$ -- функция перехода от~1 к~0 на отрезке $t \in [0,\,1]$.
Общие требования: $g(0) = 1$, $g(1) = 0$; для гладкой стыковки
на внешней границе дополнительно $g'(1) = 0$ (и желательно
$g'(0) = 0$, что обеспечивает плавный переход от плоского участка
к наклонной стенке).
Примеры: $g(t) = 1 - t^2$ (параболический переход, без гладкой
стыковки по наклону),
$g(t) = \frac{1}{2}(1 + \cos\pi t)$ (гладкий косинусный переход).

При $\rho_0 \to 0$ профиль вырождается в соответствующий «обычный»
тип; при $\rho_0 \to 1$ -- приближается к цилиндрическому с резкими
стенками. Параметр~$\rho_0$ позволяет моделировать лунки с плоским
дном, характерные для лазерного текстурирования. При $a = b$ и без
функции перехода ($g$ -- ступенька) элемент представляет собой
круговой цилиндр; при $a \neq b$ -- эллиптический цилиндр.

%% ===== Блок 3: Канавки =====

\subsubsection*{Канавки}

Для канавок область~$D_k$ -- прямоугольник в локальных координатах:
\begin{equation}
  D_k = \bigl\{|\xi| \leq \ell/2,\; |\eta| \leq a\bigr\},
  \label{eq:groove_Dk}
\end{equation}
где $\ell$ -- длина канавки (вдоль~$\xi$), $a$ -- полуширина
(по~$\eta$).

Профильная функция задаётся по поперечному направлению~$\eta$
(вдоль~$\xi$ профиль постоянен):
\[
  f_k(\xi, \eta) = f_\perp\!\bigl(|\eta|/a\bigr),
\]
где $f_\perp(t)$, $t \in [0,\,1]$ -- поперечный профиль.

При необходимости гладкого замыкания канавки по концам вводится
продольное окно:
\[
  f_k(\xi, \eta)
    = f_\parallel\!\bigl(|\xi|/(\ell/2)\bigr)
      \cdot f_\perp\!\bigl(|\eta|/a\bigr),
\]
где $f_\parallel(s)$, $s \in [0,\,1]$ -- продольный профиль
с условиями $f_\parallel(0) = 1$, $f_\parallel(1) = 0$. Например,
$f_\parallel(s) = \frac{1}{2}(1 + \cos\pi s)$. Для канавок
с резкими торцами полагают $f_\parallel \equiv 1$.

Примеры поперечного профиля:
параболический $f_\perp(t) = 1 - t^2$,
косинусный $f_\perp(t) = \frac{1}{2}(1 + \cos\pi t)$,
прямоугольный $f_\perp(t) = 1$.
Прямоугольный профиль $f_\perp(t) = 1$ соответствует ступенчатому
разрыву на границе элемента и используется для моделирования канавок
с резкими стенками; при необходимости гладкой стыковки следует
выбирать $f_\perp(1) = 0$.

Ориентация канавки задаётся углом~$\alpha_k$ через формулы локальных
координат~\eqref{eq:local_coords}. При $\alpha_k = 0$ канавка
ориентирована вдоль~$x_1$ (продольная), при $\alpha_k = \pi/2$ --
вдоль~$x_2$ (поперечная). Дополнительным параметром канавки является
её длина~$\ell$.

%% ===== Блок 4: Сводная таблица =====

\subsubsection*{Сводка типов профилей}

Основные типы профильных функций и их свойства приведены
в таблице~\ref{tab:profile_types}.

\begin{table}[!ht]
\centering
\caption{Типы профильных функций элементов микрорельефа}
\label{tab:profile_types}
\resizebox{\textwidth}{!}{%
\begin{tabular}{|l|l|c|c|l|}
\hline
\textbf{Тип} & \textbf{Профильная функция}
  & \textbf{$f(1){=}0$} & \textbf{$df/d\rho|_{1}{=}0$}
  & \textbf{Комментарий} \\
\hline
Эллипсоидальный & $\sqrt{1-\rho^2}$
  & да & нет & $f' \to -\infty$ на границе \\
\hline
Параболический & $1-\rho^2$
  & да & нет & простейший гладкий \\
\hline
Конический & $1-\rho$
  & да & нет & излом в центре \\
\hline
Косинусный & $\frac{1}{2}(1+\cos\pi\rho)$
  & да & да & гладкая стыковка \\
\hline
Smoothstep & $1-3\rho^2+2\rho^3$
  & да & да & гладкость с обеих сторон \\
\hline
Плоское дно & кусочная (см.\ текст)
  & да & зависит от $g$ & лазерные лунки \\
\hline
Канавка & $f_\perp(|\eta|/a)$
  & -- & -- & профиль по $\eta$ \\
\hline
\end{tabular}%
}
\end{table}

%% ===== Блок 5: Рисунки =====

% [РИСУНОК: Сравнение профилей f(ρ)]
\begin{figure}[!ht]
\centering
\fbox{\parbox{0.8\textwidth}{\centering
\vspace{3.5cm}
ПЛЕЙСХОЛДЕР ДЛЯ РИСУНКА\\
Сравнение профильных функций $f(\rho)$:\\
эллипсоидальный, параболический, конический, косинусный, smoothstep\\
(одинаковые $a$, $b$, $h_p$; разрез через центр элемента)
\vspace{3.5cm}
}}
\caption{Сравнение профильных функций элементов микрорельефа}
\label{fig:profile_comparison}
\end{figure}

% [РИСУНОК: Типы микроструктур (обновлённый)]
\begin{figure}[!ht]
\centering
\fbox{\parbox{0.8\textwidth}{\centering
\vspace{3cm}
ПЛЕЙСХОЛДЕР ДЛЯ РИСУНКА\\
Типы микроструктур:\\
(a) эллиптическая лунка, (b) лунка с плоским дном,\\
(c) продольная канавка, (d) поперечная канавка
\vspace{3cm}
}}
\caption{Основные типы элементов микрорельефа}
\label{fig:texture_geometries}
\end{figure}

%% ===== Завершающий абзац =====

Приведённый набор профильных функций охватывает основные типы
элементов, используемых в практике текстурирования подшипников
скольжения. Выбор конкретного профиля определяется технологией
изготовления и требованиями к гладкости численного решения.
Пространственное расположение элементов на поверхности
рассматривается в подразделе~2.2.3.
