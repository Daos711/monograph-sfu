\subsection{Ключевые допущения и вывод уравнения Рейнольдса}

Рассмотрим движение вязкой несжимаемой ньютоновской жидкости в тонком зазоре между двумя твёрдыми поверхностями. Введём декартову систему координат $(x, y, z)$, где плоскость $xz$ совпадает с нижней поверхностью, а ось $y$ направлена перпендикулярно поверхностям (рисунок~\ref{fig:lubrication_scheme}). Толщина смазочного слоя обозначается как $h(x, z, t)$, где $t$ -- время.
Давление~$p$ во всей главе понимается как избыточное: $p = p_{\mathrm{abs}} - p_{\mathrm{atm}}$, так что $p = 0$ соответствует атмосферному давлению.

% [РИСУНОК 2.1: Схема смазочного слоя]
% Должен показывать:
% - Две поверхности (верхняя и нижняя)
% - Координатные оси x, y, z
% - Толщина слоя h(x,z,t)
% - Скорости поверхностей U₁, U₂
% - Профиль скорости u(y) внутри слоя
% - Давление p(x,z,t)
\begin{figure}[!ht]
\centering
\fbox{\parbox{0.8\textwidth}{\centering
\vspace{3cm}
ПЛЕЙСХОЛДЕР ДЛЯ РИСУНКА 2.1\\
Схема смазочного слоя между двумя поверхностями\\
(найти рисунок "Reynolds lubrication schematic")
\vspace{3cm}
}}
\caption{Схема смазочного слоя между двумя поверхностями}
\label{fig:lubrication_scheme}
\end{figure}

\subsubsection*{Основные допущения}

Вывод уравнения Рейнольдса основан на следующих физических допущениях:

\begin{enumerate}
\item Жидкость ньютоновская: касательные напряжения пропорциональны скоростям деформации, в частности $\tau_{xy} = \mu\,\partial u/\partial y$, $\tau_{yz} = \mu\,\partial w/\partial y$.

\item Жидкость несжимаемая: $\rho = \mathrm{const}$.

\item Тонкий слой: $H \ll L$, где $H$ -- характерный (порядковый) масштаб толщины смазочного слоя $h(x,z,t)$; малые уклоны поверхностей $|\partial h/\partial x| \ll 1$, $|\partial h/\partial z| \ll 1$.

\item Течение ламинарное, инерционные эффекты пренебрежимо малы: $\mathrm{Re}\cdot\epsilon \ll 1$, где $\epsilon = H/L$.

\item Давление постоянно по толщине слоя: $\partial p/\partial y \approx 0$ (следствие смазочной аппроксимации).

\item Динамическая вязкость постоянна: $\mu = \mathrm{const}$ (температурная зависимость рассматривается отдельно при необходимости).

\item На границах жидкости и твёрдых поверхностей выполняется условие прилипания (no-slip).

\item Объёмные силы (гравитация и др.) малы по сравнению с градиентом давления и не учитываются.
\end{enumerate}

\subsubsection*{Исходные уравнения}

Движение жидкости описывается уравнениями Навье--Стокса для несжимаемой жидкости:

\begin{equation}
\rho\left(\frac{\partial \mathbf{v}}{\partial t} + (\mathbf{v} \cdot \nabla)\mathbf{v}\right) = -\nabla p + \mu \nabla^2 \mathbf{v},
\label{eq:navier_stokes}
\end{equation}

и уравнением неразрывности:

\begin{equation}
\nabla \cdot \mathbf{v} = 0,
\label{eq:continuity}
\end{equation}

\noindent где $\mathbf{v} = (u, v, w)$ -- вектор скорости, $\rho$ -- плотность жидкости, $p$ -- давление, $\mu$ -- динамическая вязкость.

В компонентной форме уравнения движения приведены в приложении~А. Уравнение неразрывности~\eqref{eq:continuity} в компонентной записи имеет вид:

\begin{equation}
\frac{\partial u}{\partial x} + \frac{\partial v}{\partial y} + \frac{\partial w}{\partial z} = 0.
\label{eq:continuity_comp}
\end{equation}

Далее, на основе анализа порядков величин, будет получена смазочная аппроксимация и выведено уравнение Рейнольдса.

\subsubsection*{Масштабирование и оценка порядков величин}

Для анализа уравнений введём характерные масштабы задачи:

$L$ -- характерный размер в направлениях $x$ и $z$ (длина подшипника);

$H$ -- характерная толщина смазочного слоя;

$U$ -- характерная скорость движения поверхностей;

$P$ -- характерное давление.

Безразмерные переменные вводятся следующим образом:

\begin{equation}
\bar{x} = \frac{x}{L}, \quad \bar{y} = \frac{y}{H}, \quad \bar{z} = \frac{z}{L}, \quad \bar{u} = \frac{u}{U}, \quad \bar{v} = \frac{v}{V_0}, \quad \bar{w} = \frac{w}{U}, \quad \bar{p} = \frac{p}{P},
\end{equation}
где $V_0$ -- характерная скорость в направлении $y$, которую нужно определить из уравнения неразрывности.

Из уравнения неразрывности следует оценка:

\begin{equation}
\frac{U}{L} \sim \frac{V_0}{H} \quad \Rightarrow \quad V_0 \sim U\frac{H}{L} = U\,\epsilon.
\end{equation}

Ключевым параметром в теории смазки является отношение:

\begin{equation}
\epsilon = \frac{H}{L} \ll 1,
\label{eq:epsilon}
\end{equation}
которое характеризует малость толщины слоя по сравнению с его протяжённостью.

Подставляя безразмерные переменные в компонентные уравнения движения, получаем оценки для различных членов. Из $x$-компоненты уравнения движения при смазочной аппроксимации доминирующий баланс имеет вид:

\begin{equation}
\frac{\partial p}{\partial x} \sim \mu\,\frac{\partial^2 u}{\partial y^2} \quad \Rightarrow \quad \frac{P}{L} \sim \mu\,\frac{U}{H^2} \quad \Rightarrow \quad P \sim \mu\,U\,\frac{L}{H^2}.
\label{eq:pressure_scale}
\end{equation}

Характерное давление определяется балансом вязкого сопротивления потоку в тонком зазоре и продольного градиента давления.

Вводим число Рейнольдса для смазочного слоя:

\begin{equation}
\mathrm{Re} = \frac{\rho U H}{\mu}.
\end{equation}

Отношение инерционных членов к вязким оценивается как:

\begin{equation}
\frac{\rho U^2/L}{P/L} \sim \frac{\rho U^2}{\mu U L/H^2} = \underbrace{\frac{\rho U H}{\mu}}_{\mathrm{Re}}\;\underbrace{\frac{H}{L}}_{\epsilon} = \mathrm{Re}\,\epsilon.
\label{eq:Re_epsilon}
\end{equation}

При условии $\mathrm{Re}\,\epsilon \ll 1$ (что выполняется в большинстве практических случаев) инерционные члены в уравнениях движения пренебрежимо малы по сравнению с вязкими и градиентом давления. Малость этого параметра означает, что инерция жидкости подавлена тонкостью смазочного слоя ($\epsilon = H/L \ll 1$), и течение полностью определяется балансом вязких сил и градиента давления.

Анализируя порядки производных с учётом $\epsilon \ll 1$, получаем:

\begin{equation}
\frac{\partial^2}{\partial y^2} \gg \frac{\partial^2}{\partial x^2}, \quad \frac{\partial^2}{\partial y^2} \gg \frac{\partial^2}{\partial z^2}.
\end{equation}

Таким образом, в $x$- и $z$-компонентах уравнения движения доминируют члены:

\begin{equation}
\frac{\partial p}{\partial x} \sim \mu\frac{\partial^2 u}{\partial y^2}, \quad \frac{\partial p}{\partial z} \sim \mu\frac{\partial^2 w}{\partial y^2}.
\end{equation}

Из $y$-компоненты уравнения движения следует, что величина $\partial p/\partial y$ мала по сравнению с $\partial p/\partial x$ и $\partial p/\partial z$ и имеет порядок $O(\epsilon^2)$, откуда:

\begin{equation}
\frac{\partial p}{\partial y} \approx 0.
\label{eq:dp_dy_zero}
\end{equation}

Это означает, что давление в смазочном слое не зависит от координаты $y$: $p = p(x, z, t)$.

\subsubsection*{Упрощённые уравнения движения}

С учётом сделанных допущений уравнения движения принимают вид:

\begin{align}
\frac{\partial p}{\partial x} &= \mu\frac{\partial^2 u}{\partial y^2}, \label{eq:simplified_x} \\
\frac{\partial p}{\partial z} &= \mu\frac{\partial^2 w}{\partial y^2}. \label{eq:simplified_z}
\end{align}

Уравнение неразрывности сохраняет форму~\eqref{eq:continuity_comp}.

\subsubsection*{Граничные условия}

На поверхностях выполняются условия прилипания (no-slip):

\begin{equation}
\begin{aligned}
&y = 0: \quad u = U_1(x, z, t), \quad w = W_1(x, z, t), \quad v = V_1(x, z, t), \\
&y = h: \quad u = U_2(x, z, t), \quad w = W_2(x, z, t), \quad v = V_2(x, z, t),
\end{aligned}
\label{eq:boundary_conditions}
\end{equation}
где $U_i$, $W_i$, $V_i$ -- компоненты скорости соответствующих поверхностей. В общем случае скорости поверхностей могут зависеть от координат и времени. Далее при выводе уравнения Рейнольдса величины $U_i$ и $W_i$ полагаются постоянными вдоль соответствующей поверхности, что корректно для основных типов подшипников, рассматриваемых в настоящей работе.

\subsubsection*{Профиль скорости}

Интегрируя уравнение~\eqref{eq:simplified_x} дважды по $y$ и используя граничные условия~\eqref{eq:boundary_conditions}, получаем профиль скорости в направлении $x$:

\begin{equation}
u(x, y, z, t) = U_1 + \frac{y}{h}(U_2 - U_1) + \frac{1}{2\mu}\frac{\partial p}{\partial x}y(y - h).
\label{eq:velocity_profile_u}
\end{equation}

Аналогично для направления $z$:

\begin{equation}
w(x, y, z, t) = W_1 + \frac{y}{h}(W_2 - W_1) + \frac{1}{2\mu}\frac{\partial p}{\partial z}y(y - h).
\label{eq:velocity_profile_w}
\end{equation}

Профиль скорости~\eqref{eq:velocity_profile_u} состоит из двух частей: линейной (куэттовский поток, обусловленный движением поверхностей) и параболической (пуазейлевский поток, обусловленный градиентом давления).

\subsubsection*{Объёмный расход}

Объёмный расход жидкости через единицу длины перпендикулярно направлению $x$ определяется интегрированием скорости по толщине слоя:

\begin{equation}
q_x = \int_0^h u \,\mathrm{d}y.
\end{equation}

Подставляя выражение~\eqref{eq:velocity_profile_u}:

\begin{equation}
q_x = \int_0^h \left[U_1 + \frac{y}{h}(U_2 - U_1) + \frac{1}{2\mu}\frac{\partial p}{\partial x}y(y - h)\right] dy.
\end{equation}

Вычисляем интегралы:

\begin{equation}
\begin{aligned}
\int_0^h U_1 \,\mathrm{d}y &= U_1 h, \\
\int_0^h \frac{y}{h}(U_2 - U_1) \,\mathrm{d}y &= \frac{U_2 - U_1}{h} \cdot \frac{h^2}{2} = \frac{h}{2}(U_2 - U_1), \\
\int_0^h \frac{1}{2\mu}\frac{\partial p}{\partial x}y(y - h) \,\mathrm{d}y &= \frac{1}{2\mu}\frac{\partial p}{\partial x}\left(\frac{h^3}{3} - \frac{h^3}{2}\right) = -\frac{h^3}{12\mu}\frac{\partial p}{\partial x}.
\end{aligned}
\end{equation}

Окончательно для расхода в направлении $x$:

\begin{equation}
q_x = \frac{h}{2}(U_1 + U_2) - \frac{h^3}{12\mu}\frac{\partial p}{\partial x}.
\label{eq:flow_rate_x}
\end{equation}

Аналогично для направления $z$:

\begin{equation}
q_z = \frac{h}{2}(W_1 + W_2) - \frac{h^3}{12\mu}\frac{\partial p}{\partial z}.
\label{eq:flow_rate_z}
\end{equation}

\subsubsection*{Вывод уравнения Рейнольдса}

Из уравнения неразрывности~\eqref{eq:continuity}, интегрируя по толщине слоя от $y = 0$ до $y = h$:

\begin{equation}
\int_0^h \frac{\partial u}{\partial x} \,\mathrm{d}y + \int_0^h \frac{\partial v}{\partial y} \,\mathrm{d}y + \int_0^h \frac{\partial w}{\partial z} \,\mathrm{d}y = 0.
\end{equation}

Используя правило Лейбница для дифференцирования интеграла:

\begin{equation}
\frac{\partial}{\partial x}\int_0^h u \,\mathrm{d}y = \int_0^h \frac{\partial u}{\partial x} \,\mathrm{d}y + u|_{y=h}\frac{\partial h}{\partial x},
\end{equation}
и аналогично для $z$. Таким образом, интегралы производных в уравнении неразрывности выражаются как:

\begin{equation}
\int_0^h \frac{\partial u}{\partial x}\,\mathrm{d}y = \frac{\partial}{\partial x}\!\int_0^h u\,\mathrm{d}y
  - u\big|_{y=h}\,\frac{\partial h}{\partial x}, \qquad
\int_0^h \frac{\partial w}{\partial z}\,\mathrm{d}y = \frac{\partial}{\partial z}\!\int_0^h w\,\mathrm{d}y
  - w\big|_{y=h}\,\frac{\partial h}{\partial z}.
\label{eq:leibniz_applied}
\end{equation}

Для второго интеграла:

\begin{equation}
\int_0^h \frac{\partial v}{\partial y} \,\mathrm{d}y = v|_{y=h} - v|_{y=0} = V_2 - V_1.
\end{equation}

Кинематическое условие на верхней поверхности $y = h(x,z,t)$:

\begin{equation}
V_2 = \frac{\partial h}{\partial t} + U_2\frac{\partial h}{\partial x} + W_2\frac{\partial h}{\partial z}.
\label{eq:kinematic_upper}
\end{equation}

Нижняя поверхность полагается плоской и неподвижной в направлении $y$, откуда $V_1 = 0$. Здесь использовано условие прилипания на верхней поверхности: $u(h) = U_2$, $w(h) = W_2$, $v(h) = V_2$. Подставляя~\eqref{eq:leibniz_applied} и кинематическое условие~\eqref{eq:kinematic_upper} в интегрированное уравнение неразрывности, после сокращения членов $U_2\,\partial h/\partial x$ и $W_2\,\partial h/\partial z$, приходим к уравнению:

Используя определения расходов~\eqref{eq:flow_rate_x} и~\eqref{eq:flow_rate_z}:

\begin{equation}
\frac{\partial q_x}{\partial x} + \frac{\partial q_z}{\partial z} + \frac{\partial h}{\partial t} = 0.
\label{eq:continuity_integrated}
\end{equation}

Уравнение~\eqref{eq:continuity_integrated} допускает наглядную интерпретацию в терминах контрольного объёма (рисунок~\ref{fig:control_volume}): изменение толщины слоя $\partial h/\partial t$ компенсируется потоками $q_x$ и $q_z$ через боковые грани элемента $dx \times dz$.

% [РИСУНОК: Контрольный объём смазочного слоя]
% Должен показывать:
% - Элемент dx × dz, толщина h
% - Потоки q_x и q_x + ∂q_x/∂x·dx через левую и правую грани
% - Потоки q_z и q_z + ∂q_z/∂z·dz через переднюю и заднюю грани
% - Стрелка ∂h/∂t сверху (выжимание)
% - Формула баланса: ∂q_x/∂x + ∂q_z/∂z + ∂h/∂t = 0
\begin{figure}[!ht]
\centering
\fbox{\parbox{0.8\textwidth}{\centering
\vspace{3cm}
ПЛЕЙСХОЛДЕР ДЛЯ РИСУНКА\\
Контрольный объём смазочного слоя:\\
элемент $dx \times dz$, толщина $h$,\\
потоки $q_x$, $q_z$ через боковые грани,\\
выжимание $\partial h/\partial t$
\vspace{3cm}
}}
\caption{Контрольный объём смазочного слоя и баланс расходов}
\label{fig:control_volume}
\end{figure}

Подставляя выражения для $q_x$ и $q_z$:

\begin{equation}
\frac{\partial}{\partial x}\left[\frac{h}{2}(U_1 + U_2) - \frac{h^3}{12\mu}\frac{\partial p}{\partial x}\right] + \frac{\partial}{\partial z}\left[\frac{h}{2}(W_1 + W_2) - \frac{h^3}{12\mu}\frac{\partial p}{\partial z}\right] + \frac{\partial h}{\partial t} = 0.
\end{equation}

Раскрывая производные:

\begin{equation}
\frac{\partial}{\partial x}\left(\frac{h^3}{12\mu}\frac{\partial p}{\partial x}\right) + \frac{\partial}{\partial z}\left(\frac{h^3}{12\mu}\frac{\partial p}{\partial z}\right) = \frac{\partial}{\partial x}\left[\frac{h}{2}(U_1 + U_2)\right] + \frac{\partial}{\partial z}\left[\frac{h}{2}(W_1 + W_2)\right] + \frac{\partial h}{\partial t}.
\label{eq:reynolds_full}
\end{equation}

Уравнение~\eqref{eq:reynolds_full} является \textbf{обобщённым уравнением Рейнольдса} для случая нестационарного течения с подвижными границами произвольной формы.
Вязкость~$\mu$ записана под знаком производных, что обеспечивает корректность формулы и при переменной вязкости $\mu = \mu(x, z)$ (например, при учёте температурной зависимости в главе~4).

Для стационарного случая ($\partial h/\partial t = 0$) и при условии $W_1 = W_2 = 0$ (движение только в направлении $x$) уравнение упрощается:

\begin{equation}
\frac{\partial}{\partial x}\left(\frac{h^3}{12\mu}\frac{\partial p}{\partial x}\right) + \frac{\partial}{\partial z}\left(\frac{h^3}{12\mu}\frac{\partial p}{\partial z}\right) = \frac{U_1 + U_2}{2}\frac{\partial h}{\partial x}.
\label{eq:reynolds_steady}
\end{equation}

Уравнение~\eqref{eq:reynolds_steady} представляет собой \textbf{классическое уравнение Рейнольдса} для гидродинамической смазки.

В случае одной движущейся поверхности (например, $U_1 = U$, $U_2 = 0$) и постоянной вязкости $\mu = \mathrm{const}$ обе части уравнения можно умножить на~$12\mu$, что даёт эквивалентную форму, удобную для численной реализации:

\begin{equation}
\frac{\partial}{\partial x}\left(\frac{h^3}{\mu}\frac{\partial p}{\partial x}\right) + \frac{\partial}{\partial z}\left(\frac{h^3}{\mu}\frac{\partial p}{\partial z}\right) = 6U\frac{\partial h}{\partial x}.
\label{eq:reynolds_one_surface}
\end{equation}

Полученное уравнение Рейнольдса~\eqref{eq:reynolds_one_surface} является основным уравнением теории гидродинамической смазки и связывает распределение давления $p(x, z)$ в смазочном слое с геометрией зазора $h(x, z)$ и скоростью движения поверхности $U$.

Следует отметить, что при решении уравнения Рейнольдса в областях расходящегося зазора могут возникать зоны отрицательного давления, не имеющие физического смысла. Для корректного описания таких областей необходима массо-сохраняющая модель кавитации, которая рассматривается в разделе~2.3 на основе подхода Jakobsson--Floberg--Olsson (JFO).
