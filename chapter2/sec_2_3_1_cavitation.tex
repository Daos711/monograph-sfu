\subsection{Физика разрыва плёнки и ограничения модели с обнулением давления}

%% Блок 1: Постановка проблемы — откуда p < 0
Уравнение Рейнольдса для полностью заполненного зазора
(подразделы~2.1.2--2.1.4) в ряде случаев даёт области
с~$p < 0$ -- как правило, в зоне расходящегося клина, за максимумом
давления.  Математически это следствие граничных условий и допущения
сплошного заполнения зазора жидкостью.  Физически отрицательное
давление означает, что жидкость находится в состоянии растяжения,
что для смазочных жидкостей допустимо лишь в узком диапазоне.
Отрицательное давление в классическом решении -- индикатор того,
что допущение полного заполнения зазора жидкостью перестаёт быть
корректным.

%% Блок 2: Физическая картина — разрыв плёнки и кавитация
При снижении давления ниже некоторого порога~$p_{\mathrm{cav}}$ (кавитационного давления, обычно близкого к давлению насыщенных паров или окружающей среды)
происходит выделение растворённого газа из смазочной жидкости,
образование парогазовых полостей (каверн) и формирование двухфазной
области (жидкость~+ газ/пар), в которой давление приблизительно
постоянно и равно~$p_{\mathrm{cav}}$.  В результате расчётная область
разделяется на характерные зоны: зона полного заполнения
($p > p_{\mathrm{cav}}$), зона кавитации (частичное заполнение,
$p \approx p_{\mathrm{cav}}$), граница разрыва (переход от полного
заполнения к кавитации) и граница восстановления (реформация --
обратный переход).  Между границами разрыва и восстановления зазор
заполнен смесью жидкости и газа, причём объёмная доля жидкости
убывает по мере удаления от границы разрыва.  В инженерной модели
обычно принимают~$p_{\mathrm{cav}}$ постоянным; при работе
с избыточными давлениями часто полагают $p_{\mathrm{cav}} = 0$.

%% Блок 7: Рисунок (после физической картины, до обнуления давления)
\begin{figure}[!ht]
\centering
\fbox{\parbox{0.85\textwidth}{\centering
\vspace{4cm}
ПЛЕЙСХОЛДЕР ДЛЯ РИСУНКА\\
Кавитация в смазочном слое:\\
(a) продольный разрез: клин $h(x)$, график $p(x)$\\
с максимумом и зоной $p < p_{\mathrm{cav}}$ (штрих),\\
границы разрыва и восстановления;\\
(b) вид сверху: область полного заполнения\\
($p > p_{\mathrm{cav}}$, $\vartheta = 1$) и область кавитации\\
($p = p_{\mathrm{cav}}$, $0 < \vartheta < 1$)
\vspace{4cm}
}}
\caption{Схема кавитации в смазочном слое: продольный разрез~(a)
  и вид сверху~(b)}
\label{fig:cavitation_scheme}
\end{figure}

%% Блок 3: Простой приём — обнуление давления
Наиболее распространённый инженерный подход к устранению нефизичных
отрицательных давлений состоит в их замене на~$p_{\mathrm{cav}}$:
\[
  p := \max(p,\; p_{\mathrm{cav}}).
\]
Этот приём (условие Гюмбеля, half-Sommerfeld) позволяет получить
приемлемые оценки несущей способности и формы поля давления.
Однако он является неполной физической моделью, поскольку
кавитация -- не просто $p = p_{\mathrm{cav}}$, а ещё и частичное
заполнение зазора.

%% Блок 4: Главный недостаток — нарушение баланса массы
Уравнение Рейнольдса выведено из уравнения неразрывности при
допущении полного заполнения зазора ($\vartheta = 1$, где $\vartheta$ --
степень заполнения).  Типовой удельный расход (объёмный поток) в плёнке имеет вид
\[
  \mathbf{q} = -\frac{h^3}{12\mu}\,\nabla p
    + \frac{h}{2}\,\mathbf{U},
\]
где $\mathbf{U}$ -- скорость относительного скольжения поверхностей.
Если «отрезать» отрицательные давления, то меняется градиент
давления~$\nabla p$ и, следовательно, поток~$\mathbf{q}$, но при этом
не вводится переменная, описывающая частичное заполнение.  В результате
на границах кавитационной зоны возникают нефизичные источники или стоки
массы.  Эффект особенно заметен при расчётах расхода смазки
и в нестационарных задачах (подвижные микроструктуры, переходные
режимы).  Таким образом, корректная модель должна одновременно
описывать (а)~давление и (б)~степень заполнения в кавитационной
области.

%% Блок 5: Требования к корректной модели кавитации
Сформулируем минимальные требования к модели кавитации:
\begin{enumerate}
  \item $p \geq p_{\mathrm{cav}}$ во всей расчётной области.
  \item В зоне полного заполнения ($\vartheta = 1$) действует
        классическое уравнение Рейнольдса.
  \item В кавитационной зоне ($\vartheta < 1$) давление фиксировано
        $p = p_{\mathrm{cav}}$, а состояние смеси описывается
        дополнительной переменной $\vartheta \in (0,\, 1]$.
  \item Сохранение массы через границу зон (непрерывность нормальной компоненты потока).
\end{enumerate}

%% Блок 6: Подводка к JFO
Перечисленным требованиям удовлетворяет массосохраняющая постановка
Якобссона--Флоберга--Ольссона (JFO), в которой кавитационная зона
описывается не только условием $p = p_{\mathrm{cav}}$, но и переменной
степенью заполнения~$\vartheta$.  Соответствующие уравнения формулируются
в подразделе~2.3.2.
