\subsection{Физика разрыва пленки и ограничения модели с обнулением давления}

% СОДЕРЖАНИЕ:
% - Что происходит при отрицательном давлении
% - Почему простое p=0 неверно (нарушение массового баланса)
% - Физическая картина кавитации

При решении уравнения Рейнольдса в ряде случаев получаются 
отрицательные значения давления...

% [РИСУНОК 2.10: Схема кавитации]
\begin{figure}[!ht]
\centering
\fbox{\parbox{0.8\textwidth}{\centering
\vspace{3cm}
ПЛЕЙСХОЛДЕР ДЛЯ РИСУНКА 2.10\\
Схема кавитации в смазочном слое:\\
зона полного заполнения (p > 0),\\
зона кавитации (p = 0, частичное заполнение)
\vspace{3cm}
}}
\caption{Схематическое представление кавитации в смазочном слое}
\label{fig:cavitation_scheme}
\end{figure}
