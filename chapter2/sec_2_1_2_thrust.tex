\subsection{Упорная постановка задачи}

% Упорный подшипник (thrust bearing)
% Полная постановка задачи: геометрия, кинематика, уравнение Рейнольдса,
% граничные условия, расходы, целевые функционалы

Рассмотрим упорный (осевой) подшипник скольжения -- узел, в котором осевая нагрузка воспринимается тонким слоем вязкой жидкости, заключённым между двумя поверхностями: вращающимся диском (ротором) и неподвижной опорной поверхностью (колодкой, или сектором). Несущая способность подшипника обусловлена гидродинамическим давлением, которое генерируется в сходящемся зазоре при относительном движении поверхностей.

Упорные подшипники скольжения находят широкое применение в инженерной практике. В турбинах гидро- и теплоэнергетических установок они воспринимают осевое усилие, создаваемое рабочим телом на рабочем колесе. В центробежных насосах и компрессорах упорные подшипники фиксируют осевое положение ротора, предотвращая контакт вращающихся и неподвижных элементов. В тяжёлых прокатных станах и вертикальных гидрогенераторах упорные подшипники несут вес ротора и всего рабочего оборудования.

Цель постановки: для заданной геометрии зазора $h(r,\theta)$ и режима движения поверхностей получить распределение давления $p(r,\theta)$ в смазочном слое. По найденному полю давления далее вычисляются интегральные характеристики, приведённые в подразделе~2.1.7. Все допущения, принятые при выводе уравнения Рейнольдса в подразделе~2.1.1, сохраняют силу: жидкость ньютоновская и несжимаемая, течение ламинарное и стационарное, давление постоянно по толщине смазочного слоя. Далее рассматривается один сектор (колодка) подшипника; полный подшипник состоит из $N$ одинаковых секторов, расположенных равномерно по окружности.


\subsubsection*{Геометрия и система координат}

Для описания упорного подшипника естественно использовать цилиндрические координаты $(r, \theta, y)$:
\begin{itemize}
\item $r$ -- радиальная координата в плоскости подшипника;
\item $\theta$ -- окружная координата, отсчитываемая по направлению вращения;
\item $y$ -- координата по толщине смазочного слоя, направленная перпендикулярно рабочим поверхностям (от $y = 0$ на неподвижной колодке до $y = h$ на вращающемся диске).
\end{itemize}

Область расчёта для одного сектора определяется следующим образом:
\begin{equation}
r \in [R_{\mathrm{in}},\; R_{\mathrm{out}}], \qquad \theta \in [\theta_0,\; \theta_1],
\label{eq:thrust_domain}
\end{equation}
где $R_{\mathrm{in}}$ и $R_{\mathrm{out}}$ -- внутренний и наружный радиусы рабочей зоны подшипника, $\theta_0$ и $\theta_1$ -- угловые границы сектора. Угловая протяжённость сектора обозначается
\begin{equation}
\beta = \theta_1 - \theta_0.
\label{eq:thrust_beta}
\end{equation}

Граница $\theta_0$ соответствует входной кромке (leading edge), а $\theta_1$ -- выходной кромке (trailing edge) по направлению вращения диска. Толщина смазочного слоя в каждой точке расчётной области составляет $y \in [0,\; h(r,\theta)]$.

Для подшипника из $N$ одинаковых секторов, расположенных равномерно по окружности, угловая протяжённость одного сектора определяется как $\beta = 2\pi/N$ за вычетом углового зазора между соседними колодками. Расчётная область одного сектора схематически изображена на рисунке~\ref{fig:thrust_bearing}.

% [РИСУНОК: Схема упорного подшипника — вид сверху на сектор]
% Должен показывать:
% - Вид сверху на один сектор подшипника
% - Оси r, θ, направление вращения ω
% - Границы R_in, R_out, θ_0, θ_1
% - Угловую протяжённость β
\begin{figure}[!ht]
\centering
\fbox{\parbox{0.8\textwidth}{\centering
\vspace{3cm}
ПЛЕЙСХОЛДЕР ДЛЯ РИСУНКА\\
Схема упорного подшипника -- вид сверху на сектор\\
Оси $r$, $\theta$, направление вращения $\omega$,\\
границы $R_{\mathrm{in}}$, $R_{\mathrm{out}}$, $\theta_0$, $\theta_1$
\vspace{3cm}
}}
\caption{Схема упорного подшипника: вид сверху на один сектор}
\label{fig:thrust_bearing}
\end{figure}


\subsubsection*{Кинематика движения поверхностей}

Нижняя поверхность ($y = 0$) представляет собой неподвижную колодку, верхняя поверхность ($y = h$) -- вращающийся диск с постоянной угловой скоростью $\omega$. Тангенциальная (окружная) скорость скольжения вращающегося диска в точке с радиальной координатой $r$ составляет:
\begin{equation}
U_\theta(r) = \omega\,r.
\label{eq:thrust_U_theta}
\end{equation}

Радиальные скорости обеих поверхностей равны нулю: $U_{r,1} = U_{r,2} = 0$. В стационарном режиме нормальные скорости поверхностей также обращаются в нуль: $V_1 = V_2 = 0$, что исключает эффект выжимания ($\partial h/\partial t = 0$).

Динамическая вязкость смазки полагается постоянной: $\mu = \mathrm{const}$ (допущение~6 из подраздела~2.1.1). Влияние температурной зависимости вязкости $\mu(T)$ на характеристики подшипника рассматривается отдельно в главе~4.

Вращение диска создаёт тангенциальный увлекающий поток (куэттовскую составляющую течения): жидкость, увлекаемая подвижной поверхностью за счёт условия прилипания, движется в окружном направлении. Когда такой поток проходит через участок зазора с уменьшающейся толщиной, условие неразрывности (сохранения массы) приводит к росту давления в смазочном слое -- именно этот механизм, называемый клиновым эффектом, обеспечивает несущую способность подшипника.


\subsubsection*{Закон толщины смазочного слоя}

В общем случае толщина смазочного слоя является заданной функцией координат:
\begin{equation}
h = h(r, \theta),
\label{eq:thrust_h_general}
\end{equation}
определяющей профиль зазора между колодкой и вращающимся диском. В разделе~2.2 функция $h(r,\theta)$ будет расширена для учёта детерминированных микроструктур рабочей поверхности.

Для тестирования численных методов и верификации результатов широко используется базовая геометрия -- линейный клин, в котором толщина зазора линейно изменяется по окружной координате:
\begin{equation}
h(r,\theta) = h_{\mathrm{in}} + (h_{\mathrm{out}} - h_{\mathrm{in}})\,\frac{\theta - \theta_0}{\theta_1 - \theta_0},
\label{eq:thrust_h_wedge}
\end{equation}
где $h_{\mathrm{in}} = h(\theta_0)$ -- толщина смазочного слоя на входной кромке сектора, $h_{\mathrm{out}} = h(\theta_1)$ -- толщина на выходной кромке. Необходимым условием генерации давления является сходящийся характер зазора по направлению движения, т.\,е.
\begin{equation}
h_{\mathrm{in}} > h_{\mathrm{out}}.
\label{eq:thrust_convergence_condition}
\end{equation}

В частном случае $h_{\mathrm{in}} = h_{\mathrm{out}}$ (плоскопараллельный зазор) правая часть уравнения Рейнольдса обращается в нуль и давление не генерируется.

В формуле~\eqref{eq:thrust_h_wedge} величины $h_{\mathrm{in}}$ и $h_{\mathrm{out}}$ приняты постоянными по радиальной координате $r$, что соответствует плоскому клину. Минимальная толщина смазочного слоя обозначается
\begin{equation}
h_{\min} = \min_{(r,\theta)} h(r,\theta);
\label{eq:thrust_h_min}
\end{equation}
для линейного клина $h_{\min} = h_{\mathrm{out}}$.

Для характеристики формы клина вводятся безразмерные параметры конвергенции:
\begin{equation}
K = \frac{h_{\mathrm{in}}}{h_{\mathrm{out}}}, \qquad \delta = \frac{h_{\mathrm{in}} - h_{\mathrm{out}}}{h_{\mathrm{out}}} = K - 1.
\label{eq:thrust_convergence}
\end{equation}

При $K = 1$ ($\delta = 0$) зазор является плоскопараллельным и давление не генерируется. Оптимальное значение параметра $K$, обеспечивающее максимальную несущую способность, зависит от соотношения размеров подшипника и определяется численно.

Помимо линейного клина, на практике применяются более сложные профили зазора: ступенчатый (профиль Рэлея), параболический, составной (с плоским участком и клиновой частью). Все перечисленные варианты описываются тем же уравнением Рейнольдса при соответствующем задании функции $h(r,\theta)$. Линейный клин~\eqref{eq:thrust_h_wedge} используется в настоящей работе как базовый верификационный профиль; в реальной колодке профиль зазора $h(r,\theta)$ определяется наклоном опоры и формой рабочей поверхности, что может приводить к зависимости $h$ от радиальной координаты~$r$. Продольное сечение клинового зазора при фиксированном $r$ аналогично общей схеме смазочного слоя (рисунок~\ref{fig:lubrication_scheme}), где горизонтальная координата~$x$ заменяется на~$\theta$, нижняя поверхность соответствует неподвижной колодке, а верхняя~-- плоскому вращающемуся диску со скоростью~$U_\theta = \omega r$. Зазор сужается от~$h_{\mathrm{in}}$ на входном крае колодки до~$h_{\mathrm{out}}$ на выходном.



\subsubsection*{Уравнение Рейнольдса в цилиндрических координатах}

Обобщённое уравнение Рейнольдса~\eqref{eq:reynolds_full}, полученное в подразделе~2.1.1 в декартовых координатах, необходимо переписать в цилиндрических координатах $(r, \theta)$, естественных для задачи об упорном подшипнике. Переход осуществляется введением дуговой координаты $s = r\theta$, вдоль которой производная принимает вид $\partial/\partial s = (1/r)\,\partial/\partial\theta$. Оператор дивергенции и градиента в плоскости $(r, \theta)$ приобретает стандартные множители $1/r$ и $1/r^2$, присутствующие в левой части уравнения Рейнольдса.

В рассматриваемой задаче скорость скольжения направлена по окружной координате~$\theta$: нижняя поверхность неподвижна ($U_{\theta,1} = 0$), верхняя движется со скоростью $U_{\theta,2} = \omega r$; радиальные скорости обеих поверхностей равны нулю. Правая часть обобщённого уравнения Рейнольдса~\eqref{eq:reynolds_full} в цилиндрических координатах записывается в дивергентной форме:
\begin{equation}
\mathrm{RHS} = \frac{1}{r}\frac{\partial}{\partial\theta}\!\left[\frac{h}{2}
(U_{\theta,1} + U_{\theta,2})\right], \qquad U_{\theta,1} = 0,\quad U_{\theta,2} = \omega r,
\label{eq:thrust_rhs_div}
\end{equation}
откуда, учитывая, что произведение $\omega r$ не зависит от~$\theta$:
\begin{equation}
\frac{U_{\theta,1} + U_{\theta,2}}{2}\,\frac{1}{r}\,\frac{\partial h}{\partial \theta} = \frac{\omega r}{2}\,\frac{1}{r}\,\frac{\partial h}{\partial \theta} = \frac{\omega}{2}\,\frac{\partial h}{\partial \theta}.
\label{eq:thrust_rhs_derivation}
\end{equation}

Стационарное уравнение Рейнольдса для несжимаемой ньютоновской жидкости в цилиндрических координатах принимает вид:
\begin{equation}
\frac{1}{r}\frac{\partial}{\partial r}\!\left[r\,\frac{h^3}{12\mu}\,\frac{\partial p}{\partial r}\right] + \frac{1}{r^2}\frac{\partial}{\partial \theta}\!\left[\frac{h^3}{12\mu}\,\frac{\partial p}{\partial \theta}\right] = \frac{\omega}{2}\,\frac{\partial h}{\partial \theta}.
\label{eq:thrust_reynolds}
\end{equation}

Множители $1/r$ и $1/r^2$ перед производными в левой части обусловлены дивергентной формой эллиптического оператора $\nabla \cdot (h^3 \nabla p)$ в цилиндрических координатах. Оператор Лапласа является его частным случаем при $h = \mathrm{const}$.

Умножая обе части уравнения~\eqref{eq:thrust_reynolds} на~$12\mu$ (что допустимо при $\mu = \mathrm{const}$), получаем эквивалентную форму, удобную для численной реализации:
\begin{equation}
\frac{1}{r}\frac{\partial}{\partial r}\!\left[r\,h^3\,\frac{\partial p}{\partial r}\right] + \frac{1}{r^2}\frac{\partial}{\partial \theta}\!\left[h^3\,\frac{\partial p}{\partial \theta}\right] = 6\mu\omega\,\frac{\partial h}{\partial \theta}.
\label{eq:thrust_reynolds_num}
\end{equation}

Уравнение~\eqref{eq:thrust_reynolds_num} является основным расчётным уравнением для упорного подшипника в настоящей работе.

Левая часть уравнений~\eqref{eq:thrust_reynolds}--\eqref{eq:thrust_reynolds_num} представляет собой эллиптический оператор, описывающий перераспределение давления через вязкий смазочный слой: давление <<диффундирует>> как в радиальном, так и в окружном направлениях, причём <<проводимость>> пропорциональна кубу толщины зазора~$h^3$. Правая часть -- источниковый член, обусловленный изменением толщины зазора в направлении движения (клиновой эффект). При $\partial h/\partial \theta = 0$ (плоскопараллельный зазор) правая часть обращается в нуль, и при однородных граничных условиях уравнение допускает лишь тривиальное решение $p = 0$ -- давление не генерируется.

При наличии детерминированных микроструктур на рабочей поверхности функция $h(r,\theta)$ приобретает быстрые локальные вариации. Формально уравнение~\eqref{eq:thrust_reynolds_num} остаётся тем же, однако функция зазора существенно усложняется, что описывается в разделе~2.2.


\subsubsection*{Граничные условия}

Для замыкания уравнения Рейнольдса~\eqref{eq:thrust_reynolds_num} необходимо задать граничные условия на давление по всему контуру расчётной области~\eqref{eq:thrust_domain}. Рассмотрим три практически важных варианта.

\textit{Базовый вариант (атмосферные кромки).} На всех четырёх границах сектора давление принимается равным атмосферному:
\begin{equation}
\begin{aligned}
p(r,\,\theta_0) &= 0, \qquad & p(r,\,\theta_1) &= 0, \\
p(R_{\mathrm{in}},\,\theta) &= 0, \qquad & p(R_{\mathrm{out}},\,\theta) &= 0.
\end{aligned}
\label{eq:thrust_bc_base}
\end{equation}

Здесь давление отсчитывается от атмосферного ($p = p_{\mathrm{abs}} - p_{\mathrm{atm}}$). Условие~\eqref{eq:thrust_bc_base} соответствует изолированному сектору, кромки которого свободно сообщаются с окружающей средой, а принудительная подача смазки отсутствует. Данный вариант является наиболее распространённым при расчёте несамоустанавливающихся колодок.

\textit{Вариант с давлением подачи.} Если система смазки обеспечивает принудительную подачу жидкости через входную кромку сектора, граничное условие на $\theta = \theta_0$ модифицируется:
\begin{equation}
p(r,\,\theta_0) = p_{\mathrm{sup}}(r),
\label{eq:thrust_bc_supply}
\end{equation}
где $p_{\mathrm{sup}}(r)$ -- заданное давление подачи смазки; в простейшем случае $p_{\mathrm{sup}} = \mathrm{const}$. Остальные граничные условия остаются такими же, как в базовом варианте~\eqref{eq:thrust_bc_base}. Величина $p_{\mathrm{sup}}$ определяется конструкцией системы смазки и режимом работы насоса подачи.

\textit{Кавитация.} При решении уравнения Рейнольдса в областях расходящегося зазора расчётное давление может оказаться отрицательным. Отрицательное давление физически не реализуется: при снижении давления ниже давления насыщенных паров смазки жидкость разрывается и образуется кавитационная зона с парогазовой смесью. В простейшей постановке применяют ограничение $p \geq 0$ (условие Гюмбеля), при котором отрицательные давления обнуляются после решения. Такой подход, однако, нарушает закон сохранения массы на границе кавитационной зоны. Корректная массо-сохраняющая постановка, основанная на модели Jakobsson--Floberg--Olsson (JFO), вводится в разделе~2.3. Граница между зоной полного смазочного слоя и кавитационной зоной заранее неизвестна и определяется в процессе решения как часть задачи.

Следует отметить, что условия Дирихле $p = 0$ на всех четырёх границах сектора представляют лишь один из возможных вариантов замыкания задачи. В зависимости от конструкции узла и схемы смазки могут применяться и другие постановки: периодичность по $\theta$ (при моделировании полного кольцевого сегмента без разрывов), условие непротекания через радиальные границы $q_r = 0$, что эквивалентно $\partial p/\partial r = 0$ (при наличии уплотнений или перемычек), а также смешанные условия при комбинированной подаче смазки. Конкретный набор граничных условий определяется конструкцией узла и схемой подвода смазки.

Перечисленные варианты граничных условий схематически показаны на рисунке~\ref{fig:thrust_bc_schemes}.

% [РИСУНОК: Варианты граничных условий на секторе]
% Должен показывать три подрисунка (a), (b), (c):
% (a) Базовый: p = 0 на всех 4 границах сектора
% (b) С подпиткой: p = p_sup на θ = θ_0, остальные p = 0
% (c) Альтернативный: ∂p/∂r = 0 на радиальных границах (уплотнения)
%     или периодичность по θ
% На каждом подрисунке — контур сектора с обозначением ГУ на каждой стороне
\begin{figure}[!ht]
\centering
\fbox{\parbox{0.8\textwidth}{\centering
\vspace{3.5cm}
ПЛЕЙСХОЛДЕР ДЛЯ РИСУНКА\\
Варианты граничных условий на секторе:\\
(a) $p = 0$ на всех границах;\\
(b) $p = p_{\mathrm{sup}}$ на $\theta = \theta_0$, остальные $p = 0$;\\
(c) $\partial p/\partial r = 0$ на радиальных границах (уплотнения)
\vspace{3.5cm}
}}
\caption{Варианты граничных условий для упорного подшипника: (a)~$p=0$ на всех четырёх границах сектора (атмосферные кромки); (b)~$p=p_{\mathrm{sup}}$ на входной кромке $\theta=\theta_0$ (принудительная подача), остальные границы $p=0$; (c)~$\partial p/\partial r=0$ на радиальных границах (уплотнения), $p=0$ на дуговых кромках}
\label{fig:thrust_bc_schemes}
\end{figure}


\subsubsection*{Объёмные расходы}

Объёмные расходы жидкости через единицу длины определяются путём интегрирования профилей скорости по толщине смазочного слоя, аналогично выражениям~\eqref{eq:flow_rate_x}--\eqref{eq:flow_rate_z}, полученным в подразделе~2.1.1 в декартовых координатах. Ниже приведены формулы, адаптированные к цилиндрической системе координат.

Радиальный расход (на единицу длины по окружной координате):
\begin{equation}
q_r(r,\theta) = -\frac{h^3}{12\mu}\,\frac{\partial p}{\partial r}.
\label{eq:thrust_q_r}
\end{equation}

Поскольку радиальные скорости обеих поверхностей равны нулю, куэттовская составляющая в радиальном направлении отсутствует. Радиальный расход полностью определяется градиентом давления (пуазейлевская составляющая).

Окружной расход (на единицу длины по радиальной координате):
\begin{equation}
q_\theta(r,\theta) = \frac{h\,\omega\,r}{2} - \frac{h^3}{12\mu\,r}\,\frac{\partial p}{\partial \theta}.
\label{eq:thrust_q_theta}
\end{equation}

Первое слагаемое в~\eqref{eq:thrust_q_theta} представляет куэттовскую составляющую расхода, обусловленную увлечением жидкости вращающейся поверхностью. При $U_{\theta,1}=0$ и $U_{\theta,2}=\omega r$ средняя скорость потока составляет $\bar{U}_\theta = \omega r / 2$, а объёмный расход через единицу длины равен $h\,\bar{U}_\theta = h\omega r/2$. Второе слагаемое -- пуазейлевская составляющая; множитель $1/r$ в знаменателе возникает из-за метрического коэффициента при переходе от производной по дуге $\partial/\partial(r\theta)$ к производной по углу $\partial/\partial\theta$.

Суммарные объёмные расходы через радиальные границы подшипника получаются интегрированием удельных расходов по окружной кромке:
\begin{equation}
Q_{\mathrm{out}} = \int_{\theta_0}^{\theta_1} q_r(R_{\mathrm{out}}, \theta)\,R_{\mathrm{out}}\,\mathrm{d}\theta, \qquad Q_{\mathrm{in}} = \int_{\theta_0}^{\theta_1} q_r(R_{\mathrm{in}}, \theta)\,R_{\mathrm{in}}\,\mathrm{d}\theta.
\label{eq:thrust_Q_radial}
\end{equation}

Множитель $R_{\mathrm{out}}$ (соответственно, $R_{\mathrm{in}}$) перед $d\theta$ обусловлен элементом длины дуги $dl = r\,\mathrm{d}\theta$ на окружности радиуса $r$.

Разность $Q_{\mathrm{out}} - Q_{\mathrm{in}}$ определяет утечку смазки через радиальные кромки сектора. Суммарный окружной расход через входную и выходную окружные кромки дополняет баланс массы: полный расход жидкости, поступающей в сектор, равен полному расходу, покидающему его.

Баланс массы в стационарном режиме без выжимания записывается в дифференциальной форме:
\begin{equation}
\frac{1}{r}\,\frac{\partial (r\,q_r)}{\partial r} + \frac{1}{r}\,\frac{\partial q_\theta}{\partial \theta} = 0.
\label{eq:thrust_mass_balance}
\end{equation}

Подстановка выражений~\eqref{eq:thrust_q_r} и~\eqref{eq:thrust_q_theta} в уравнение~\eqref{eq:thrust_mass_balance} приводит к уравнению Рейнольдса~\eqref{eq:thrust_reynolds}, что подтверждает внутреннюю согласованность полученных соотношений.


Решение уравнения Рейнольдса даёт поле давления~$p$ в расчётной области. По найденному давлению вычисляются интегральные характеристики узла: несущая способность, расходы смазки, силы и моменты трения, минимальная толщина плёнки. Унифицированные формулы для всех выходных величин приведены в подразделе~2.1.7.

Сформулированная задача сводится к решению эллиптического уравнения Рейнольдса~\eqref{eq:thrust_reynolds_num} в области~\eqref{eq:thrust_domain} при заданном профиле зазора $h(r,\theta)$ и граничных условиях на давление. Структура уравнения (эллиптический оператор в левой части, источниковый член в правой) обеспечивает единственность решения при корректно поставленных граничных условиях Дирихле в отсутствие кавитации, т.\,е. для линейной постановки без ограничений на знак давления.

Численная дискретизация уравнения~\eqref{eq:thrust_reynolds_num} методом конечных разностей и алгоритм учёта кавитации на основе модели JFO рассматриваются в главе~3. При наличии микроструктур на рабочей поверхности функция зазора $h(r,\theta)$ приобретает быстрые локальные вариации, параметризация которых описывается в разделе~2.2.
