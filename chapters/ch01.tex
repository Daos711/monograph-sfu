% =============================================================
% 1 Объект и предмет исследования
% =============================================================
\chapter{Объект и предмет исследования}

\section{Подшипники скольжения}

Подшипники скольжения представляют собой опорные узлы машин и~механизмов, в~которых относительное перемещение сопряжённых поверхностей вала и~опоры происходит в~условиях скольжения. Несущая способность таких подшипников обеспечивается слоем смазочного материала, разделяющим рабочие поверхности. Подшипники скольжения являются одними из~древнейших элементов машин, однако и~в~настоящее время они остаются незаменимыми во~многих ответственных областях техники благодаря ряду уникальных преимуществ: высокой нагрузочной способности, демпфирующим свойствам, способности работать при~высоких скоростях вращения и~большому ресурсу.

\subsection{Классификация подшипников скольжения}

Подшипники скольжения классифицируют по~нескольким признакам.

По~направлению воспринимаемой нагрузки различают:
\begin{itemize}
\item радиальные (опорные) подшипники, воспринимающие нагрузку, направленную перпендикулярно оси вращения вала; основным конструктивным элементом является цилиндрический вкладыш, охватывающий цапфу вала;
\item упорные (осевые) подшипники, воспринимающие осевую нагрузку, действующую вдоль оси вращения; конструктивно представляют собой набор площадок (сегментов), расположенных перпендикулярно оси; классическим примером является подшипник Митчелла с~самоустанавливающимися сегментами;
\item радиально-упорные (комбинированные) подшипники, способные одновременно воспринимать радиальную и~осевую нагрузки.
\end{itemize}

По~способу создания несущего давления в~смазочном слое подшипники скольжения подразделяют на:
\begin{itemize}
\item гидродинамические, в~которых несущее давление формируется за~счёт вращения вала~-- вязкая смазка затягивается в~клиновидный зазор и~создаёт область повышенного давления, уравновешивающую внешнюю нагрузку;
\item гидростатические, в~которых смазочный материал подаётся в~зазор под~давлением от~внешнего источника (насоса), что~позволяет обеспечить несущую способность при~любой скорости вращения, включая нулевую;
\item гибридные, сочетающие оба~принципа~-- внешнюю подачу смазки под~давлением и~гидродинамический эффект при~вращении.
\end{itemize}

По~виду смазочного материала различают подшипники, работающие на~жидкой смазке (минеральные и~синтетические масла, вода, технологические жидкости), газовой смазке (воздух, инертные газы), а~также подшипники сухого трения, в~которых роль смазки выполняет сам~материал вкладыша (полимеры, графит, композиты).

Настоящая монография посвящена преимущественно гидродинамическим подшипникам скольжения, работающим на~жидкой смазке, как~наиболее распространённому типу в~промышленности.

\subsection{Принцип работы гидродинамического подшипника}

Работа гидродинамического подшипника скольжения основана на~явлении, открытом Б.~Тауэром экспериментально (1883~г.) и~теоретически объяснённом О.~Рейнольдсом (1886~г.): при~относительном перемещении двух поверхностей, разделённых слоем вязкой жидкости, в~области сужающегося зазора возникает избыточное давление, способное нести внешнюю нагрузку.

В~радиальном подшипнике скольжения клиновидный зазор образуется естественным образом: под~действием внешней нагрузки ось вала смещается относительно оси вкладыша на~величину эксцентриситета~$e$. При~этом зазор между поверхностями становится переменным по~окружности~-- от~максимального значения~$h_{\max} = c + e$ до~минимального~$h_{\min} = c - e$, где~$c = R - r$~-- радиальный зазор, $R$~-- радиус вкладыша, $r$~-- радиус вала. Относительный эксцентриситет $\varepsilon = e / c$ является одним из~основных безразмерных параметров, характеризующих режим работы подшипника ($0 \leq \varepsilon < 1$).

При~вращении вала вязкая смазка увлекается в~область сужающегося зазора, где~вследствие условия неразрывности потока и~сил~вязкости формируется область повышенного давления. Результирующая сила давления, действующая на~вал, уравновешивает внешнюю нагрузку. Положение равновесия определяется балансом сил~и~зависит от~скорости вращения, вязкости смазки, геометрии подшипника и~величины нагрузки.

В~упорном подшипнике клиновидный зазор создаётся конструктивно~-- наклоном рабочих площадок (колодок) относительно поверхности вращающегося диска (пяты). В~подшипнике Митчелла каждая колодка имеет возможность самоустановки на~точечной или~линейной опоре, что~обеспечивает оптимальный наклон при~различных нагрузках и~скоростях.

Аналитическое описание распределения давления в~смазочном слое основывается на~уравнении Рейнольдса~-- дифференциальном уравнении в~частных производных, которое является упрощённой формой уравнений Навье~-- Стокса для~тонкого вязкого слоя. Вывод и~формулировки уравнения Рейнольдса для~различных постановок задачи подробно рассмотрены в~разделе~2 настоящей монографии.

\subsection{Режимы трения}

Режим трения в~подшипнике скольжения определяется соотношением между толщиной смазочного слоя и~высотой шероховатости (микронеровностей) контактирующих поверхностей. Это~соотношение характеризуется параметром смазочного слоя

\begin{equation}
\Lambda = \frac{h_{\min}}{R_q}\,,
\end{equation}

\noindent где $h_{\min}$~-- минимальная толщина слоя, м;\\
\hphantom{где }$R_q = \sqrt{R_{q1}^2 + R_{q2}^2}$~-- комбинированная среднеквадратическая шероховатость поверхностей, м;\\
\hphantom{где }$R_{q1}$ и~$R_{q2}$~-- среднеквадратические шероховатости первой и~второй поверхности соответственно, м.

Принято выделять следующие режимы трения.

Жидкостное (гидродинамическое) трение реализуется при~$\Lambda > 3\text{--}5$. Поверхности полностью разделены слоем смазки, непосредственный контакт микронеровностей отсутствует. Сила трения обусловлена исключительно вязким сдвигом смазки. Коэффициент трения минимален и~составляет $f \sim 10^{-3}$. Износ практически отсутствует, ресурс подшипника максимален. Это~наиболее благоприятный режим работы.

Смешанное трение наблюдается при~$1 < \Lambda < 3\text{--}5$. Толщина смазочного слоя сопоставима с~высотой шероховатостей, и~нагрузка частично воспринимается контактом микронеровностей, частично~-- гидродинамическим давлением в~смазке. Коэффициент трения существенно возрастает по~сравнению с~жидкостным режимом, появляется износ.

Граничное трение имеет место при~$\Lambda < 1$. Смазочный слой имеет молекулярную толщину (единицы нанометров). Нагрузка воспринимается тонкими плёнками, адсорбированными на~поверхностях. Коэффициент трения составляет $f \sim 0{,}05\text{--}0{,}15$ и~определяется свойствами граничных плёнок и~материалов поверхностей.

Сухое трение~-- непосредственный контакт поверхностей при~полном отсутствии смазки. Коэффициент трения максимален ($f \sim 0{,}1\text{--}1{,}0$), износ интенсивен.

Зависимость коэффициента трения от~безразмерного параметра нагружения $\mu N / p$ (где~$\mu$~-- динамическая вязкость, $N$~-- частота вращения, $p$~-- удельная нагрузка) описывается кривой Штрибека (\textit{Stribeck curve}). Кривая наглядно демонстрирует переходы между режимами трения и~позволяет определить оптимальную область эксплуатации подшипника. Минимум коэффициента трения на~кривой Штрибека соответствует началу жидкостного режима.

В~процессе эксплуатации подшипник может проходить через несколько режимов трения~-- например, при~пуске и~остановке машины, когда скорость вращения недостаточна для~формирования несущего смазочного слоя. Именно в~этих переходных режимах происходит основной износ, и~именно здесь детерминированные микроструктуры рабочей поверхности могут оказать наиболее существенное положительное влияние за~счёт удержания смазки в~зоне контакта.

\subsection{Конструкции и~материалы}

Конструктивное исполнение подшипников скольжения весьма разнообразно и~определяется условиями эксплуатации, нагрузкой и~скоростью.

Радиальные подшипники выполняются с~различной формой внутренней поверхности вкладыша. Наиболее простой является цилиндрическая форма~-- вкладыш представляет собой цилиндрическую втулку с~одним радиальным зазором. Эллиптическая форма (лимонный подшипник) характеризуется наличием двух~зон сужения зазора, расположенных диаметрально, что~повышает устойчивость ротора. Многоклиновые подшипники (трёхклиновые, четырёхклиновые) имеют несколько сходящихся клиньев по~окружности и~применяются в~высокоскоростных машинах для~обеспечения устойчивости. Подшипники с~плавающими вкладышами (подшипники Лундстрёма) содержат внутреннее кольцо, вращающееся со~скоростью, промежуточной между скоростью вала и~корпуса, что~снижает потери на~трение. Подшипники с~качающимися колодками обеспечивают максимальную устойчивость ротора и~применяются в~наиболее ответственных турбомашинах.

Упорные подшипники конструктивно представляют собой набор колодок (сегментов), расположенных на~кольцевой опоре. По~способу создания клинового зазора различают подшипники с~фиксированным углом наклона колодок (подшипник Рэлея), с~самоустанавливающимися колодками на~точечной опоре (подшипник Митчелла) и~со~сплошным кольцом, имеющим спиральные канавки.

Материал вкладыша (антифрикционного слоя) является одним из~ключевых факторов, определяющих работоспособность подшипника. К~основным группам антифрикционных материалов относятся:
\begin{itemize}
\item баббиты (оловянные и~свинцовые)~-- классические антифрикционные сплавы на~основе олова или~свинца с~добавками меди и~сурьмы; обладают превосходной прирабатываемостью и~способностью поглощать (вмуровывать) абразивные частицы; применяются в~энергетическом машиностроении, судостроении и~тяжёлом машиностроении; ограничены по~допустимой температуре (до~120--150\,$^{\circ}$C) и~усталостной прочности;
\item бронзы (оловянные, свинцовые, алюминиевые)~-- обладают более высокой прочностью и~теплопроводностью по~сравнению с~баббитами; применяются при~повышенных нагрузках; требуют более тщательной обработки поверхности вала для~предотвращения задира;
\item полимерные материалы (ПТФЭ, полиамиды, полиимиды, фенольные композиты)~-- применяются в~условиях водяной смазки, сухого трения или~при~наличии абразивных частиц в~среде; обладают химической стойкостью и~хорошими антифрикционными свойствами; ограничены по~допустимой температуре и~теплопроводности;
\item композиционные материалы (металлополимерные, керамико-металлические)~-- сочетают свойства различных компонентов; позволяют создать вкладыш с~заданным комплексом свойств;
\item керамические материалы (карбид кремния, оксид алюминия)~-- применяются в~условиях особо агрессивных сред и~высоких температур.
\end{itemize}

Выбор материала определяется комплексом условий эксплуатации: нагрузкой, скоростью, температурой, видом смазочного материала, наличием абразивных частиц и~химической средой. Нанесение детерминированных микроструктур возможно практически на~все~перечисленные материалы, хотя технология обработки будет различаться.

\subsection{Области применения}

Область применения подшипников скольжения чрезвычайно широка и~охватывает практически все~отрасли машиностроения.

В~энергетике подшипники скольжения являются штатными опорами роторов паровых и~газовых турбин, турбогенераторов, гидрогенераторов и~компрессоров газоперекачивающих агрегатов. Мощные турбоагрегаты работают при~частотах вращения до~3000--3600~об/мин, нагрузках на~опору, достигающих десятков тонн, и~сроке службы 20--30~лет. В~этих условиях подшипники скольжения практически безальтернативны.

В~нефтегазовой отрасли подшипники скольжения используются в~центробежных насосах для~перекачки нефти, газа и~пластовых вод, в~винтовых и~поршневых компрессорах, а~также в~буровом оборудовании. Особый интерес представляют опоры шарошечных буровых долот, работающих в~экстремальных условиях: высокие осевые и~радиальные нагрузки, повышенная температура, наличие абразивных частиц в~промывочной жидкости, ограниченные возможности подачи смазки. Ресурс бурового долота в~значительной мере определяется работоспособностью подшипниковых опор шарошек.

В~химической промышленности подшипники скольжения применяются в~герметичных (бессальниковых) насосах и~компрессорах, где~недопустима утечка перекачиваемой среды. В~таких конструкциях подшипники работают на~перекачиваемой жидкости (кислоты, щёлочи, растворители), что~предъявляет особые требования к~химической стойкости материалов.

В~пищевой и~фармацевтической промышленности используются подшипники на~водяной смазке, обеспечивающие стерильность продукции.

В~транспортном машиностроении подшипники скольжения применяются в~двигателях внутреннего сгорания (коренные и~шатунные вкладыши коленчатого вала, втулки поршневого пальца), в~автоматических трансмиссиях, рулевых механизмах и~подвесках. Автомобильная промышленность является одним из~крупнейших потребителей подшипников скольжения по~количеству.

В~судостроении подшипники скольжения (дейдвудные подшипники) используются в~опорах гребных валов. Они работают на~морской воде, что~определяет выбор полимерных и~композиционных антифрикционных материалов.

В~прецизионном оборудовании (шпиндели металлорежущих станков, координатно-измерительные машины) применяются гидростатические и~аэростатические подшипники, обеспечивающие высочайшую точность вращения.

Во~всех перечисленных областях совершенствование подшипниковых узлов~-- снижение потерь на~трение, повышение нагрузочной способности и~ресурса~-- имеет существенное экономическое и~экологическое значение.

\subsection{Основные характеристики и~параметры}

Работа подшипника скольжения описывается рядом статических и~динамических характеристик.

К~статическим характеристикам относятся следующие.

Нагрузочная способность~-- максимальная нагрузка, которую способен нести смазочный слой при~заданных условиях (скорость, вязкость, геометрия) без~разрушения плёнки. Условия нагружения и~смазки удобно характеризовать числом Зоммерфельда

\begin{equation}
S = \dfrac{\mu N L D}{W}\left(\dfrac{R}{c}\right)^2,
\end{equation}

\noindent где $\mu$~-- динамическая вязкость, Па$\cdot$с;\\
\hphantom{где }$N$~-- частота вращения, об/с;\\
\hphantom{где }$L$ и~$D$~-- длина и~диаметр подшипника, м;\\
\hphantom{где }$W$~-- нагрузка, Н;\\
\hphantom{где }$R/c$~-- отношение радиуса к~зазору.

Относительный эксцентриситет $\varepsilon = e/c$ определяет положение вала во~вкладыше и~связан с~нагрузочной способностью~-- чем~больше нагрузка, тем~больше $\varepsilon$ и~тем~меньше минимальная толщина смазочного слоя

\begin{equation}
h_{\min} = c\,(1-\varepsilon).
\end{equation}

Коэффициент трения~$f$ и~момент трения~$M_{\text{тр}}$ характеризуют потери энергии. При~жидкостном трении они определяются вязким сдвигом смазки и~зависят от~вязкости, скорости и~толщины слоя; в~типичных условиях $f \sim 10^{-3}\text{--}10^{-2}$.

Расход (утечки) смазки~$Q$ определяет объём смазки, вытекающей через торцы подшипника в~единицу времени. Расход влияет на~тепловой баланс (вынос тепла с~маслом) и~проектирование системы смазки.

Максимальная температура смазочного слоя~$T_{\max}$ ограничивает допустимый режим работы и~определяется балансом между теплом, выделяемым при~вязком трении, и~теплоотводом через утечки и~стенки подшипника.

Максимальное давление в~смазочном слое~$p_{\max}$ определяет прочностные требования к~конструкции подшипника и~материалу вкладыша.

К~динамическим характеристикам относятся коэффициенты жёсткости ($k_{xx}$, $k_{xy}$, $k_{yx}$, $k_{yy}$) и~демпфирования ($d_{xx}$, $d_{xy}$, $d_{yx}$, $d_{yy}$) смазочного слоя, определяемые как~производные силовой реакции плёнки по~перемещениям и~скоростям вала. Эти~коэффициенты описывают реакцию смазочного слоя на~малые возмущения положения ротора и~являются ключевыми входными данными для~анализа динамики и~устойчивости роторных систем. Характерной особенностью подшипников скольжения является наличие перекрёстных коэффициентов жёсткости ($k_{xy} \neq k_{yx}$), обусловленных вращением смазки; именно они~являются источником неконсервативных сил, способных вызвать автоколебания ротора.

\subsection{Направления совершенствования}

Основными задачами, стоящими перед разработчиками подшипниковых узлов, являются: повышение нагрузочной способности, снижение потерь на~трение, обеспечение динамической устойчивости ротора, продление ресурса, снижение шума и~расхода смазки.

Традиционные направления совершенствования включают:
\begin{itemize}
\item оптимизацию макрогеометрии подшипника~-- выбор профиля расточки (цилиндрический, эллиптический, многоклиновой), соотношения длины к~диаметру~$L/D$, величины радиального зазора~$c$;
\item совершенствование систем смазки~-- оптимизация расположения и~размеров маслоподводящих отверстий и~канавок, подбор вязкости масла;
\item применение новых антифрикционных материалов~-- разработку композитных вкладышей, многослойных покрытий;
\item модификацию конструкции~-- переход к~подшипникам с~качающимися колодками, использование демпферных опор, применение активного управления.
\end{itemize}

Относительно новым и~перспективным направлением является модификация микрогеометрии рабочей поверхности~-- нанесение детерминированных микроструктур (поверхностное текстурирование). В~отличие от~перечисленных выше подходов, оно~не~требует изменения конструкции подшипника, совместимо с~существующими материалами и~системами смазки, а~эффект достигается за~счёт управления гидродинамикой смазочного слоя на~микроскопическом уровне. Данное направление рассмотрено в~следующем подразделе.


\section{Детерминированные микроструктуры рабочей поверхности}

Детерминированные микроструктуры (ДМ)~-- это~регулярные рельефные элементы заданной геометрии, размеров и~расположения, целенаправленно создаваемые на~рабочей поверхности деталей узлов трения. Термин <<детерминированные>> подчёркивает принципиальное отличие таких структур от~случайной (стохастической) шероховатости, неизбежно присутствующей на~любой обработанной поверхности: параметры каждого элемента микрорельефа определены заранее и~воспроизводимы.

\subsection{Определение и~терминология}

В~отечественной литературе для~обозначения рассматриваемых структур используются несколько терминов: <<регулярный микрорельеф>>, <<дискретный микрорельеф>>, <<текстурированная поверхность>>, <<детерминированные микроструктуры>>. В~зарубежной литературе наиболее распространён термин \textit{surface texturing} (поверхностное текстурирование), а~сами~структуры обозначаются как~\textit{surface texture}, \textit{micro-dimples}, \textit{micro-grooves}, \textit{micro-pockets}. В~настоящей работе термины <<детерминированные микроструктуры>>, <<микрорельеф>> и~<<текстурирование поверхности>> используются как~синонимы.

Нормативной основой классификации регулярного микрорельефа в~отечественной практике является ГОСТ~24773-81 <<Поверхности с~регулярным микрорельефом. Классификация, параметры и~характеристики>>. Данный стандарт определяет:
\begin{itemize}
\item классификацию элементов регулярного микрорельефа по~форме (канавки, лунки, выступы) и~по~расположению на~поверхности;
\item основные параметры микрорельефа: шаг~$l$, глубина (высота)~$h$, ширина~$b$;
\item характеристики микрорельефа: относительная площадь элементов, плотность размещения;
\item систему обозначений параметров микрорельефа на~чертежах и~в~технической документации.
\end{itemize}

Согласно ГОСТ~24773-81, регулярный микрорельеф представляет собой совокупность элементов определённой геометрической формы, расположенных в~заданном порядке на~обработанной поверхности. Стандарт был~разработан преимущественно для~деталей, обработанных методами поверхностного пластического деформирования (вибронакатывание, алмазное выглаживание), однако его~классификация и~система параметров применимы и~к~микроструктурам, создаваемым современными технологиями (лазерная обработка, фотолитография).

Исследования влияния регулярного микрорельефа на~трение и~износ деталей машин ведутся с~1960--1970-х~годов (работы Ю.~Г.~Шнейдера, И.~В.~Крагельского и~др.). Однако качественный скачок в~данной области произошёл в~1990--2000-е~годы благодаря работам И.~Эцион (I.~Etsion) и~его~группы, которые систематически исследовали влияние лазерного текстурирования на~характеристики механических уплотнений, поршневых колец и~подшипников скольжения. В~настоящее время данное направление активно развивается десятками исследовательских групп во~всём мире.

\subsection{Классификация элементов микрорельефа}

Элементы детерминированных микроструктур классифицируют по~нескольким признакам.

\textit{По~типу элемента} различают:
\begin{itemize}
\item углубления (лунки, карманы)~-- выемки в~поверхности детали; наиболее распространённый тип, обеспечивающий эффекты генерации давления, удержания смазки и~захвата частиц износа;
\item канавки~-- протяжённые углубления; в~зависимости от~ориентации выделяют продольные (в~направлении скольжения), поперечные (перпендикулярно направлению скольжения), наклонные и~спиральные канавки;
\item выступы (бугорки)~-- элементы, возвышающиеся над~базовой поверхностью; формируют локальные зоны конвергентного зазора;
\item комбинированные элементы~-- сочетание углублений и~выступов, а~также элементы сложной формы.
\end{itemize}

\textit{По~форме поперечного сечения} лунки подразделяют на:
\begin{itemize}
\item сферические (сегментные)~-- образованы пересечением поверхности со~сферой заданного радиуса; характеризуются плавным изменением глубины от~центра к~краям; наиболее распространены в~исследованиях;
\item цилиндрические (с~плоским дном)~-- имеют вертикальные стенки и~плоское дно; обеспечивают больший объём при~одинаковом диаметре по~сравнению со~сферическими; характеризуются резким перепадом глубины на~краях, что~влияет на~характер течения смазки;
\item конические~-- имеют линейно изменяющуюся глубину;
\item параболические~-- имеют параболический профиль, промежуточный между сферическим и~цилиндрическим;
\item каплевидные~-- несимметричные элементы, ориентированные относительно направления скольжения; асимметрия профиля может усилить гидродинамический эффект.
\end{itemize}

\textit{По~форме в~плане} элементы могут быть круглыми, эллиптическими, прямоугольными, треугольными, шевронными (V-образными). Шевронные элементы представляют особый интерес, поскольку V-образная форма, ориентированная вершиной в~направлении скольжения, эффективно нагнетает смазку к~центру контакта и~может существенно увеличить нагрузочную способность.

\subsection{Параметры и~расположение элементов}

Геометрические параметры отдельного элемента микрорельефа включают:
\begin{itemize}
\item характерный размер (диаметр) элемента~$d$; для~подшипников скольжения типичные значения составляют от~50 до~500~мкм;
\item глубину (или~высоту для~выступов)~$h_d$; типичные значения~-- от~5 до~50~мкм;
\item отношение глубины к~диаметру~$h_d/d$; типичные значения~-- от~0{,}05 до~0{,}5; при~малых значениях ($h_d/d < 0{,}1$) элемент называют мелким, при~больших ($h_d/d > 0{,}3$)~-- глубоким;
\item площадь единичного элемента в~плане~$s$.
\end{itemize}

Совокупность элементов на~поверхности характеризуется:
\begin{itemize}
\item относительной площадью (плотностью) текстурирования~$S_p$~-- отношением суммарной площади элементов к~общей площади текстурированной зоны; типичные значения~-- от~5 до~40\,\%; многочисленные исследования показывают, что~оптимальное значение~$S_p$ для~подшипников скольжения обычно находится в~диапазоне 10--30\,\%;
\item шагом расположения~$l$~-- расстоянием между центрами соседних элементов;
\item схемой (паттерном) расположения;
\item зоной текстурирования~-- частью поверхности, на~которую нанесены элементы.
\end{itemize}

Схема расположения элементов на~рабочей поверхности существенно влияет на~эффективность текстурирования. Основные варианты включают следующие.

Кольцевое (концентрическое) расположение~-- элементы размещаются вдоль концентрических окружностей. Элементы в~соседних рядах могут быть расположены друг~напротив друга или~со~смещением. Схема наиболее проста в~реализации и~характерна для~ранних работ. Применяется преимущественно в~упорных подшипниках и~торцовых уплотнениях.

Шахматное расположение~-- элементы размещаются в~узлах прямоугольной или~треугольной сетки, причём каждый чётный ряд смещён на~половину шага относительно нечётного. По~сравнению с~кольцевым расположением обеспечивает более равномерное покрытие поверхности и~отсутствие <<дорожек>> без~элементов.

Расположение по~принципу филлотаксиса~-- биомиметическая схема, основанная на~закономерности, наблюдаемой в~расположении семян в~корзинке подсолнечника, чешуек сосновой шишки, листьев на~стебле и~других природных объектов. Положение $n$-го элемента задаётся формулами

\begin{equation}
r_n = a\sqrt{n}, \qquad \varphi_n = n \cdot \varphi_{\text{зол}},
\end{equation}

\noindent где $n$~-- порядковый номер элемента;\\
\hphantom{где }$r_n$~-- расстояние от~центра до~$n$-го элемента, м;\\
\hphantom{где }$\varphi_n$~-- угловое положение $n$-го элемента, рад;\\
\hphantom{где }$a$~-- масштабный коэффициент, м$\cdot$элемент$^{-1/2}$;\\
\hphantom{где }$\varphi_{\text{зол}} = 2\pi\left(1 - \dfrac{1}{\Phi}\right) \approx 2{,}3999~\text{рад} \ (\approx 137{,}508^{\circ})$~-- золотой угол;\\
\hphantom{где }$\Phi = (1+\sqrt{5})/2$~-- золотое сечение.

Такое расположение обеспечивает наиболее равномерную плотность элементов без~выраженных направлений периодичности. Отсутствие <<строк>> и~<<столбцов>> может быть полезным для~снижения тональных составляющих шума и~вибрации, обусловленных регулярным прохождением элементов через зону нагружения.

Выбор зоны текстурирования~-- важный параметр проектирования. В~радиальных подшипниках элементы могут быть нанесены на~всю поверхность вкладыша (полное текстурирование), на~часть поверхности в~зоне сходящегося зазора, в~зоне расходящегося зазора, вблизи области максимального давления или~на~поверхность вала. Оптимальный выбор зоны зависит от~эксцентриситета, скорости, нагрузки и~целей текстурирования (снижение трения, увеличение нагрузочной способности, подавление кавитации). В~упорных подшипниках текстурируется, как~правило, вся~поверхность колодки или~её~часть вблизи входной кромки.

\subsection{Технологии формирования микрорельефа}

Развитие методов создания детерминированных микроструктур тесно связано с~прогрессом в~области технологий микрообработки. Современные методы формирования микрорельефа можно разделить на~несколько групп.

\textit{Лазерное текстурирование} (\textit{laser surface texturing}, LST) является в~настоящее время наиболее распространённым и~универсальным методом. Импульсное лазерное излучение удаляет материал из~локальной области поверхности, формируя лунку или~канавку заданных размеров. Основные параметры процесса~-- энергия и~длительность импульса, частота следования импульсов, диаметр пятна, скорость перемещения луча~-- определяют геометрию и~качество элементов. Типичная точность по~глубине составляет $\pm 1\text{--}2$~мкм. Метод применим к~широкому спектру материалов (металлы, керамика, полимеры) и~обеспечивает высокую производительность. К~недостаткам относятся формирование навала (выступа) по~краям лунки и~изменение структуры материала в~зоне термического воздействия.

\textit{Фотолитография}~-- технология, заимствованная из~микроэлектроники. На~поверхность наносится фоторезист, экспонируется через маску ультрафиолетовым излучением, затем~рисунок переносится на~подложку методом травления. Позволяет создавать элементы с~разрешением до~единиц микрометров и~высокой однородностью. Применяется преимущественно в~лабораторных исследованиях, поскольку требует многоэтапного процесса, специального оборудования и~плоской (или~слабоизогнутой) подложки.

\textit{Электроэрозионная обработка} (ЭЭО)~-- удаление материала посредством электрических разрядов между электродом-инструментом и~заготовкой. Применяется для~обработки электропроводящих материалов, в~том~числе высокотвёрдых (закалённые стали, твёрдые сплавы). Позволяет создавать элементы сложной формы, однако производительность ниже, чем~у~лазерной обработки.

\textit{Химическое и~электрохимическое травление}~-- избирательное удаление материала через маску (фоторезист, лакокрасочное покрытие). Позволяет обрабатывать большие площади одновременно. Применяется для~нанесения неглубоких (единицы микрометров) структур на~плоские и~цилиндрические поверхности.

\textit{Вибронакатывание и~микровдавливание}~-- механические методы формирования микрорельефа. Деформирующий инструмент (шарик, ролик, алмазный индентор) прижимается к~вращающейся заготовке с~наложением вибрации. Формируются лунки или~канавки за~счёт пластической деформации, а~не~удаления материала, что~обеспечивает упрочнение поверхностного слоя и~формирование сжимающих остаточных напряжений. Метод описан в~работах Ю.~Г.~Шнейдера и~стандартизирован в~ГОСТ~24773-81.

\textit{3D-печать и~аддитивные технологии}~-- перспективное направление, позволяющее формировать микроструктуры непосредственно в~процессе изготовления детали. Точность современных аддитивных технологий пока~недостаточна для~создания элементов микрометрового масштаба, однако технология быстро развивается.

Выбор технологии определяется материалом детали, требуемой геометрией микроструктур, размерами обрабатываемой поверхности, точностью и~экономическими факторами. Лазерное текстурирование в~настоящее время является наиболее технологичным методом, сочетающим высокую точность, производительность, гибкость настройки и~применимость к~широкому кругу задач.

\subsection{Связь с~концепцией зелёной трибологии}

В~последние годы в~мировой трибологической науке и~практике активно развивается концепция зелёной трибологии (\textit{green tribology}), сформулированная в~работах М.~Носоновского и~Б.~Бхушана. Зелёная трибология направлена на~снижение негативного воздействия фрикционных процессов на~окружающую среду и~включает ряд принципов:
\begin{itemize}
\item минимизация потерь энергии на~трение и~износ;
\item сокращение расхода и~использование экологически безопасных смазочных материалов;
\item продление срока службы деталей и~снижение потребности в~запасных частях;
\item использование биомиметических подходов~-- заимствование оптимальных решений из~живой природы;
\item снижение уровня шума и~вибрации.
\end{itemize}

Детерминированные микроструктуры в~полной мере соответствуют принципам зелёной трибологии по~нескольким основаниям.

Во-первых, снижение коэффициента трения, достигаемое за~счёт текстурирования, ведёт к~уменьшению потребляемой энергии. По~оценкам, потери на~трение составляют около 20\,\% мирового потребления энергии, из~них~значительная часть приходится на~подшипники и~уплотнения. Даже~снижение трения на~10--20\,\% в~масштабах промышленного предприятия или~отрасли даёт существенный экономический и~экологический эффект.

Во-вторых, текстурирование не~требует изменения конструкции узла, не~предполагает использования дополнительных химических присадок и~модификаторов~-- эффект достигается исключительно за~счёт управления гидродинамикой смазочного слоя.

В-третьих, способность микроструктур удерживать смазку в~зоне контакта позволяет уменьшить расход смазочного материала и~обеспечить работоспособность при~его~кратковременном дефиците, что~снижает экологическую нагрузку.

В-четвёртых, расположение элементов по~принципу филлотаксиса является прямым примером биомиметического подхода~-- золотой угол, обеспечивающий наиболее равномерное заполнение поверхности, заимствован из~ботаники.

Наконец, увеличение ресурса деталей за~счёт снижения износа уменьшает потребление материалов и~энергии на~изготовление запасных частей.

Таким образом, исследование и~оптимизация детерминированных микроструктур рабочей поверхности представляют собой актуальное направление не~только с~точки зрения повышения технических характеристик машин, но~и~в~контексте устойчивого развития и~ресурсосбережения.

\subsection{Влияние микроструктур на~трибологические характеристики}

Механизмы влияния детерминированных микроструктур на~характеристики подшипников скольжения многообразны. Выделяют несколько основных эффектов.

\textit{Генерация дополнительного гидродинамического давления.} Каждый элемент микрорельефа (лунка, канавка) представляет собой локальное расширение зазора, за~которым следует сужение. При~движении поверхности вязкая смазка, проходя через область элемента, испытывает последовательное расширение и~сжатие, что~формирует асимметричное распределение давления. В~зоне сужения зазора (на~выходе из~лунки) давление оказывается выше, чем~в~зоне расширения (на~входе), что~создаёт результирующую положительную добавку к~несущей способности. Этот~механизм, называемый <<эффектом микроподшипника>>, наиболее эффективен при~параллельном или~слабоклиновом зазоре, т.\,е. при~малых эксцентриситетах или~в~механических уплотнениях. При~больших эксцентриситетах собственный гидродинамический эффект клинового зазора подшипника доминирует, и~влияние микроструктур становится менее выраженным.

\textit{Эффект резервуара смазки.} Лунки и~канавки на~рабочей поверхности служат микрорезервуарами, удерживающими смазочный материал. При~остановке машины смазка не~полностью вытекает из~зоны контакта, а~частично сохраняется в~углублениях. При~последующем пуске эта~смазка поступает в~зону трения, обеспечивая разделение поверхностей в~критический момент, когда гидродинамический эффект ещё~не~сформировался. Данный механизм особенно важен для~подшипников, работающих в~режиме частых пусков и~остановок, а~также для~подшипников с~ограниченной подачей смазки.

\textit{Эффект ловушки для~частиц износа.} В~условиях смешанного и~граничного трения на~поверхностях образуются частицы износа, которые, оставаясь в~зоне контакта, вызывают вторичное (абразивное) повреждение поверхностей. Углубления микрорельефа захватывают и~удерживают эти частицы, предотвращая их~рециркуляцию в~зоне трения. Эффект особенно важен для~подшипников, работающих в~загрязнённой среде,~-- например, в~опорах шарошечных буровых долот, где~промывочная жидкость содержит абразивные частицы породы.

\textit{Влияние на~кавитацию.} Кавитация (образование зон пониженного давления, заполненных паром или~газом) является характерным явлением в~подшипниках скольжения. Она~возникает в~расходящейся части зазора, где~давление снижается ниже~давления насыщения пара или~давления растворённых газов. Кавитация ограничивает область положительного давления и~тем~самым влияет на~нагрузочную способность, трение и~расход смазки. Микроструктуры, расположенные в~зоне кавитации, могут изменять её~характер~-- например, способствовать образованию множества мелких кавитационных зон вместо одной крупной, что~влияет на~распределение давления и~может как~увеличить, так~и~уменьшить нагрузочную способность.

\textit{Совокупное влияние на~коэффициент трения и~нагрузочную способность.} Результирующее воздействие микроструктур на~характеристики подшипника определяется суперпозицией перечисленных эффектов и~зависит от~множества факторов: геометрии и~глубины элементов, плотности текстурирования~$S_p$, расположения текстурированной зоны, эксцентриситета вала, скорости вращения, вязкости смазки. Экспериментальные данные свидетельствуют о~возможности снижения коэффициента трения на~20--50\,\% по~сравнению с~гладкой поверхностью при~оптимальных параметрах текстурирования. Нагрузочная способность может как~увеличиваться (при~текстурировании зоны расходящегося зазора), так~и~уменьшаться (при~неудачном расположении лунок в~зоне максимального давления). Именно эта~многопараметричность и~неоднозначность результата обуславливают необходимость применения численного моделирования для~оптимизации микроструктур, что~является основной задачей настоящей монографии.

\subsection{Влияние микроструктур на~динамические характеристики}

Помимо статических (установившихся) параметров, детерминированные микроструктуры влияют на~динамическое поведение подшипникового узла и~роторной системы в~целом.

Смазочный слой подшипника скольжения обладает упругими и~демпфирующими свойствами, которые описываются восемью линеаризованными коэффициентами~-- четырьмя коэффициентами жёсткости ($k_{xx}$, $k_{xy}$, $k_{yx}$, $k_{yy}$) и~четырьмя коэффициентами демпфирования ($d_{xx}$, $d_{xy}$, $d_{yx}$, $d_{yy}$). Эти~коэффициенты зависят от~распределения давления в~смазочном слое и, следовательно, от~геометрии рабочей поверхности. Нанесение микроструктур изменяет поле давления и~характер течения смазки, что~приводит к~изменению всех~восьми динамических коэффициентов.

Устойчивость ротора, вращающегося в~подшипниках скольжения, определяется соотношением между жёсткостными и~демпфирующими характеристиками. Характерным видом потери устойчивости является автоколебание ротора~-- так~называемый масляный вихрь (\textit{oil whirl}), частота которого близка к~половине частоты вращения, и~масляный бич (\textit{oil whip}), при~котором амплитуда автоколебаний резко возрастает. Эти~явления могут привести к~контакту ротора со~статором и~разрушению подшипника.

Имеющиеся исследования показывают, что~оптимально подобранные микроструктуры способны:
\begin{itemize}
\item увеличить пороговую скорость возникновения автоколебаний, расширив тем~самым область устойчивой работы;
\item изменить соотношение прямых и~перекрёстных коэффициентов жёсткости в~благоприятную для~устойчивости сторону;
\item увеличить коэффициенты демпфирования, улучшив виброзащитные свойства подшипника;
\item снизить амплитуду вынужденных колебаний ротора при~прохождении критических скоростей.
\end{itemize}

Вместе с~тем неудачный выбор параметров и~расположения микроструктур может оказать негативное влияние на~динамику системы~-- снизить порог устойчивости или~увеличить вибрации. Это~ещё~раз~подчёркивает необходимость всестороннего численного анализа.

Детальному рассмотрению влияния микроструктур на~динамическую устойчивость системы <<ротор~-- подшипник скольжения>> посвящён раздел~8 настоящей монографии.

\subsection{Влияние микроструктур на~шумовые характеристики}

Шум подшипниковых узлов является важным фактором при~проектировании машин, к~которым предъявляются требования по~уровню акустического излучения: насосное оборудование, бытовая техника, автомобильные двигатели, прецизионные станки.

Источниками шума в~подшипниках скольжения являются:
\begin{itemize}
\item нестационарные гидродинамические процессы в~смазочном слое~-- пульсации давления, турбулентные вихри;
\item кавитация~-- схлопывание пузырьков пара или~растворённого газа, сопровождающееся высокочастотными акустическими импульсами;
\item вибрация элементов конструкции, возбуждаемая пульсациями давления в~смазочном слое и~передаваемая через корпус подшипника;
\item контакт микронеровностей при~смешанном трении.
\end{itemize}

Детерминированные микроструктуры способны оказывать влияние на~шумовые характеристики по~нескольким каналам.

Увеличение толщины смазочного слоя и~снижение вероятности контакта микронеровностей уменьшает высокочастотные компоненты шума, связанные с~ударными взаимодействиями на~микроуровне.

Изменение условий кавитации~-- характера, протяжённости и~интенсивности кавитационной зоны~-- непосредственно влияет на~генерацию кавитационного шума.

Изменение динамических коэффициентов подшипника влияет на~вибрационное состояние роторной системы, а~через~него~-- на~структурный шум, излучаемый корпусом.

Регулярная структура микрорельефа при~кольцевом или~шахматном расположении элементов может вносить характерные тональные составляющие в~спектр шума. Частота этих составляющих определяется произведением числа элементов на~окружности (или~в~направлении скольжения) на~частоту вращения. Такой <<частотный след>> текстуры может быть нежелательным, если~он~попадает в~диапазон повышенной чувствительности человеческого слуха или~совпадает с~резонансами конструкции.

Применение нерегулярных схем расположения элементов~-- в~частности, по~принципу филлотаксиса~-- позволяет рассредоточить энергию периодических возмущений по~более широкому спектру частот и~тем~самым снизить уровень тонального шума. Данное свойство является ещё~одним аргументом в~пользу использования биомиметических схем расположения микроструктур.

Систематические исследования акустических характеристик подшипников с~детерминированными микроструктурами в~настоящее время немногочисленны. Имеющиеся данные указывают на~возможность как~снижения, так~и~увеличения уровня шума в~зависимости от~параметров микрорельефа и~режимов работы. Комплексная оптимизация микроструктур с~одновременным учётом трибологических, динамических и~шумовых характеристик остаётся актуальной и~в~значительной мере открытой исследовательской задачей.

Таким образом, детерминированные микроструктуры рабочей поверхности представляют собой эффективный и~универсальный инструмент управления характеристиками подшипников скольжения. Они~позволяют улучшить трибологические, динамические и~акустические свойства узлов трения без~изменения конструкции и~применяемых материалов. Вместе с~тем многопараметричность задачи и~неоднозначность влияния микроструктур требуют разработки адекватных математических моделей и~эффективных численных методов. Постановкам задач и~методам их~решения посвящены последующие разделы настоящей монографии.