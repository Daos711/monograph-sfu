% =============================================================
% Введение (2 страницы)
% =============================================================
\chapter*{Введение}
\addcontentsline{toc}{chapter}{Введение}

Подшипники скольжения являются одними из наиболее распространённых и~ответственных узлов трения в~современном машиностроении. Они применяются в~турбоагрегатах, центробежных насосах, компрессорах, буровом оборудовании и~многих других машинах, работающих в~условиях высоких нагрузок, скоростей и~температур. Надёжность и~ресурс подшипникового узла во~многом определяют работоспособность машины в~целом, а~потери на~трение в~подшипниках составляют существенную долю общих энергетических потерь механизма. В~связи с~этим задача повышения трибологических характеристик подшипников скольжения~-- снижения потерь на~трение, увеличения нагрузочной способности, улучшения динамической устойчивости~-- остаётся актуальной на~протяжении десятилетий.

Одним из~перспективных направлений решения этой задачи является нанесение на~рабочую поверхность подшипника детерминированных микроструктур~-- регулярного микрорельефа заданной геометрии и~расположения. В~отличие от~случайной шероховатости, детерминированные микроструктуры создаются целенаправленно методами лазерной обработки, фотолитографии, электроэрозионной обработки и~другими технологиями. Регулярные элементы микрорельефа~-- лунки, канавки, выступы~-- формируют в~смазочном слое дополнительные гидродинамические эффекты: локальные зоны повышенного давления, резервуары для~удержания смазки, ловушки для~продуктов износа. Эти эффекты позволяют улучшить несущую способность подшипника, снизить коэффициент трения и~температуру в~зоне контакта.

Интерес к~детерминированным микроструктурам существенно возрос в~последние два десятилетия, что связано как с~развитием технологий микрообработки поверхностей, так и~с~формированием концепции зелёной трибологии. Данная концепция направлена на~снижение энергопотребления, уменьшение выбросов и~продление срока службы узлов трения за~счёт оптимизации фрикционного взаимодействия. Детерминированные микроструктуры полностью отвечают этим принципам: они не~требуют изменения конструкции узла, совместимы с~существующими смазочными материалами и~способны существенно повысить энергоэффективность машины.

Несмотря на~значительное количество экспериментальных и~теоретических работ в~данной области, ряд вопросов остаётся недостаточно изученным. В~частности, отсутствует систематизированное описание математических моделей и~численных методов, позволяющих комплексно анализировать влияние геометрии, расположения и~плотности микроструктур на~характеристики смазочного слоя с~учётом кавитации, термовязкостных эффектов и~шероховатости поверхности. Кроме того, мало освещены вопросы влияния микроструктур на~динамическую устойчивость системы <<ротор~-- подшипник>>, что имеет принципиальное значение для~высокооборотных машин.

Целью настоящей монографии является систематизация и~развитие методов анализа гидродинамических характеристик подшипников скольжения с~детерминированными микроструктурами рабочей поверхности.

Для достижения поставленной цели решаются следующие задачи:
\begin{itemize}
\item формулировка постановок гидродинамических задач смазочного слоя для~радиальных и~упорных подшипников скольжения на~основе уравнения Рейнольдса с~учётом геометрии микрорельефа;
\item разработка и~реализация численных методов решения задач смазочного слоя с~учётом кавитации (модель Якобссона~-- Флоберга~-- Олссона), термовязкостных эффектов и~шероховатости поверхности;
\item анализ влияния параметров детерминированных микроструктур (формы, глубины, плотности и~схемы расположения элементов) на~трибологические характеристики подшипниковых узлов;
\item проведение расчётных исследований радиальных подшипников скольжения центробежного насоса и~шарошечного долота, а~также упорного подшипника скольжения;
\item исследование влияния микроструктур рабочей поверхности на~динамическую устойчивость системы <<ротор~-- подшипник скольжения>>.
\end{itemize}

Монография состоит из~восьми разделов. Первый раздел содержит общие сведения об~объекте и~предмете исследования. Во~втором и~третьем разделах представлены математические постановки задач и~численные методы их решения. Четвёртый раздел посвящён учёту нелинейных эффектов. В~пятом, шестом и~седьмом разделах приведены результаты анализа конкретных подшипниковых узлов. Восьмой раздел посвящён вопросам динамической устойчивости.