\chapter{Постановка и численное моделирование гидродинамических задач смазочного слоя с детерминированными микроструктурами рабочей поверхности}

% =============================================================
% ГЛАВА 2
% Структура с плейсхолдерами для рисунков и таблиц
% =============================================================

\section{Классическая модель для гладкой поверхности}

\subsection{Ключевые допущения и вывод уравнения Рейнольдса}

Рассмотрим движение вязкой несжимаемой ньютоновской жидкости в тонком зазоре между двумя твёрдыми поверхностями. Введём декартову систему координат $(x, y, z)$, где плоскость $xz$ совпадает с нижней поверхностью, а ось $y$ направлена перпендикулярно поверхностям (рисунок~\ref{fig:lubrication_scheme}). Толщина смазочного слоя обозначается как $h(x, z, t)$, где $t$ -- время.
Давление~$p$ во всей главе понимается как избыточное: $p = p_{\mathrm{abs}} - p_{\mathrm{atm}}$, так что $p = 0$ соответствует атмосферному давлению.

% [РИСУНОК 2.1: Схема смазочного слоя]
% Должен показывать:
% - Две поверхности (верхняя и нижняя)
% - Координатные оси x, y, z
% - Толщина слоя h(x,z,t)
% - Скорости поверхностей U₁, U₂
% - Профиль скорости u(y) внутри слоя
% - Давление p(x,z,t)
\begin{figure}[!ht]
\centering
\fbox{\parbox{0.8\textwidth}{\centering
\vspace{3cm}
ПЛЕЙСХОЛДЕР ДЛЯ РИСУНКА 2.1\\
Схема смазочного слоя между двумя поверхностями\\
(найти рисунок "Reynolds lubrication schematic")
\vspace{3cm}
}}
\caption{Схема смазочного слоя между двумя поверхностями}
\label{fig:lubrication_scheme}
\end{figure}

\subsubsection*{Основные допущения}

Вывод уравнения Рейнольдса основан на следующих физических допущениях:

\begin{enumerate}
\item Жидкость ньютоновская: касательные напряжения пропорциональны скоростям деформации, в частности $\tau_{xy} = \mu\,\partial u/\partial y$, $\tau_{yz} = \mu\,\partial w/\partial y$.

\item Жидкость несжимаемая: $\rho = \mathrm{const}$.

\item Тонкий слой: $H \ll L$, где $H$ -- характерный (порядковый) масштаб толщины смазочного слоя $h(x,z,t)$; малые уклоны поверхностей $|\partial h/\partial x| \ll 1$, $|\partial h/\partial z| \ll 1$.

\item Течение ламинарное, инерционные эффекты пренебрежимо малы: $\mathrm{Re}\cdot\epsilon \ll 1$, где $\epsilon = H/L$.

\item Давление постоянно по толщине слоя: $\partial p/\partial y \approx 0$ (следствие смазочной аппроксимации).

\item Динамическая вязкость постоянна: $\mu = \mathrm{const}$ (температурная зависимость рассматривается отдельно при необходимости).

\item На границах жидкости и твёрдых поверхностей выполняется условие прилипания (no-slip).

\item Объёмные силы (гравитация и др.) малы по сравнению с градиентом давления и не учитываются.
\end{enumerate}

\subsubsection*{Исходные уравнения}

Движение жидкости описывается уравнениями Навье--Стокса для несжимаемой жидкости:

\begin{equation}
\rho\left(\frac{\partial \mathbf{v}}{\partial t} + (\mathbf{v} \cdot \nabla)\mathbf{v}\right) = -\nabla p + \mu \nabla^2 \mathbf{v},
\label{eq:navier_stokes}
\end{equation}

и уравнением неразрывности:

\begin{equation}
\nabla \cdot \mathbf{v} = 0,
\label{eq:continuity}
\end{equation}

\noindent где $\mathbf{v} = (u, v, w)$ -- вектор скорости, $\rho$ -- плотность жидкости, $p$ -- давление, $\mu$ -- динамическая вязкость.

В компонентной форме уравнения движения приведены в приложении~А. Уравнение неразрывности~\eqref{eq:continuity} в компонентной записи имеет вид:

\begin{equation}
\frac{\partial u}{\partial x} + \frac{\partial v}{\partial y} + \frac{\partial w}{\partial z} = 0.
\label{eq:continuity_comp}
\end{equation}

Далее, на основе анализа порядков величин, будет получена смазочная аппроксимация и выведено уравнение Рейнольдса.

\subsubsection*{Масштабирование и оценка порядков величин}

Для анализа уравнений введём характерные масштабы задачи:

$L$ -- характерный размер в направлениях $x$ и $z$ (длина подшипника);

$H$ -- характерная толщина смазочного слоя;

$U$ -- характерная скорость движения поверхностей;

$P$ -- характерное давление.

Безразмерные переменные вводятся следующим образом:

\begin{equation}
\bar{x} = \frac{x}{L}, \quad \bar{y} = \frac{y}{H}, \quad \bar{z} = \frac{z}{L}, \quad \bar{u} = \frac{u}{U}, \quad \bar{v} = \frac{v}{V_0}, \quad \bar{w} = \frac{w}{U}, \quad \bar{p} = \frac{p}{P},
\end{equation}

где $V_0$ -- характерная скорость в направлении $y$, которую нужно определить из уравнения неразрывности.

Из уравнения неразрывности следует оценка:

\begin{equation}
\frac{U}{L} \sim \frac{V_0}{H} \quad \Rightarrow \quad V_0 \sim U\frac{H}{L} = U\,\epsilon.
\end{equation}

Ключевым параметром в теории смазки является отношение:

\begin{equation}
\epsilon = \frac{H}{L} \ll 1,
\label{eq:epsilon}
\end{equation}

которое характеризует малость толщины слоя по сравнению с его протяжённостью.

Подставляя безразмерные переменные в компонентные уравнения движения, получаем оценки для различных членов. Из $x$-компоненты уравнения движения при смазочной аппроксимации доминирующий баланс имеет вид:

\begin{equation}
\frac{\partial p}{\partial x} \sim \mu\,\frac{\partial^2 u}{\partial y^2} \quad \Rightarrow \quad \frac{P}{L} \sim \mu\,\frac{U}{H^2} \quad \Rightarrow \quad P \sim \mu\,U\,\frac{L}{H^2}.
\label{eq:pressure_scale}
\end{equation}

Характерное давление определяется балансом вязкого сопротивления потоку в тонком зазоре и продольного градиента давления.

Вводим число Рейнольдса для смазочного слоя:

\begin{equation}
\mathrm{Re} = \frac{\rho U H}{\mu}.
\end{equation}

Отношение инерционных членов к вязким оценивается как:

\begin{equation}
\frac{\rho U^2/L}{P/L} \sim \frac{\rho U^2}{\mu U L/H^2} = \underbrace{\frac{\rho U H}{\mu}}_{\mathrm{Re}}\;\underbrace{\frac{H}{L}}_{\epsilon} = \mathrm{Re}\,\epsilon.
\label{eq:Re_epsilon}
\end{equation}

При условии $\mathrm{Re}\,\epsilon \ll 1$ (что выполняется в большинстве практических случаев) инерционные члены в уравнениях движения пренебрежимо малы по сравнению с вязкими и градиентом давления. Малость этого параметра означает, что инерция жидкости подавлена тонкостью смазочного слоя ($\epsilon = H/L \ll 1$), и течение полностью определяется балансом вязких сил и градиента давления.

Анализируя порядки производных с учётом $\epsilon \ll 1$, получаем:

\begin{equation}
\frac{\partial^2}{\partial y^2} \gg \frac{\partial^2}{\partial x^2}, \quad \frac{\partial^2}{\partial y^2} \gg \frac{\partial^2}{\partial z^2}.
\end{equation}

Таким образом, в $x$- и $z$-компонентах уравнения движения доминируют члены:

\begin{equation}
\frac{\partial p}{\partial x} \sim \mu\frac{\partial^2 u}{\partial y^2}, \quad \frac{\partial p}{\partial z} \sim \mu\frac{\partial^2 w}{\partial y^2}.
\end{equation}

Из $y$-компоненты уравнения движения следует, что $\partial p/\partial y \sim \epsilon^2 \ll 1$, откуда:

\begin{equation}
\frac{\partial p}{\partial y} \approx 0.
\label{eq:dp_dy_zero}
\end{equation}

Это означает, что давление в смазочном слое не зависит от координаты $y$: $p = p(x, z, t)$.

\subsubsection*{Упрощённые уравнения движения}

С учётом сделанных допущений уравнения движения принимают вид:

\begin{align}
\frac{\partial p}{\partial x} &= \mu\frac{\partial^2 u}{\partial y^2}, \label{eq:simplified_x} \\
\frac{\partial p}{\partial z} &= \mu\frac{\partial^2 w}{\partial y^2}. \label{eq:simplified_z}
\end{align}

Уравнение неразрывности сохраняет форму~\eqref{eq:continuity_comp}.

\subsubsection*{Граничные условия}

На поверхностях выполняются условия прилипания (no-slip):

\begin{equation}
\begin{aligned}
&y = 0: \quad u = U_1(x, z, t), \quad w = W_1(x, z, t), \quad v = V_1(x, z, t), \\
&y = h: \quad u = U_2(x, z, t), \quad w = W_2(x, z, t), \quad v = V_2(x, z, t),
\end{aligned}
\label{eq:boundary_conditions}
\end{equation}

где $U_i$, $W_i$, $V_i$ -- компоненты скорости соответствующих поверхностей. В общем случае скорости поверхностей могут зависеть от координат и времени. Далее при выводе уравнения Рейнольдса величины $U_i$ и $W_i$ полагаются постоянными вдоль соответствующей поверхности, что корректно для основных типов подшипников, рассматриваемых в настоящей работе.

\subsubsection*{Профиль скорости}

Интегрируя уравнение~\eqref{eq:simplified_x} дважды по $y$ и используя граничные условия~\eqref{eq:boundary_conditions}, получаем профиль скорости в направлении $x$:

\begin{equation}
u(x, y, z, t) = U_1 + \frac{y}{h}(U_2 - U_1) + \frac{1}{2\mu}\frac{\partial p}{\partial x}y(y - h).
\label{eq:velocity_profile_u}
\end{equation}

Аналогично для направления $z$:

\begin{equation}
w(x, y, z, t) = W_1 + \frac{y}{h}(W_2 - W_1) + \frac{1}{2\mu}\frac{\partial p}{\partial z}y(y - h).
\label{eq:velocity_profile_w}
\end{equation}

Профиль скорости~\eqref{eq:velocity_profile_u} состоит из двух частей: линейной (куэттовский поток, обусловленный движением поверхностей) и параболической (пуазейлевский поток, обусловленный градиентом давления).

\subsubsection*{Объёмный расход}

Объёмный расход жидкости через единицу длины перпендикулярно направлению $x$ определяется интегрированием скорости по толщине слоя:

\begin{equation}
q_x = \int_0^h u \, dy.
\end{equation}

Подставляя выражение~\eqref{eq:velocity_profile_u}:

\begin{equation}
q_x = \int_0^h \left[U_1 + \frac{y}{h}(U_2 - U_1) + \frac{1}{2\mu}\frac{\partial p}{\partial x}y(y - h)\right] dy.
\end{equation}

Вычисляем интегралы:

\begin{equation}
\begin{aligned}
\int_0^h U_1 \, dy &= U_1 h, \\
\int_0^h \frac{y}{h}(U_2 - U_1) \, dy &= \frac{U_2 - U_1}{h} \cdot \frac{h^2}{2} = \frac{h}{2}(U_2 - U_1), \\
\int_0^h \frac{1}{2\mu}\frac{\partial p}{\partial x}y(y - h) \, dy &= \frac{1}{2\mu}\frac{\partial p}{\partial x}\left(\frac{h^3}{3} - \frac{h^3}{2}\right) = -\frac{h^3}{12\mu}\frac{\partial p}{\partial x}.
\end{aligned}
\end{equation}

Окончательно для расхода в направлении $x$:

\begin{equation}
q_x = \frac{h}{2}(U_1 + U_2) - \frac{h^3}{12\mu}\frac{\partial p}{\partial x}.
\label{eq:flow_rate_x}
\end{equation}

Аналогично для направления $z$:

\begin{equation}
q_z = \frac{h}{2}(W_1 + W_2) - \frac{h^3}{12\mu}\frac{\partial p}{\partial z}.
\label{eq:flow_rate_z}
\end{equation}

\subsubsection*{Вывод уравнения Рейнольдса}

Из уравнения неразрывности~\eqref{eq:continuity}, интегрируя по толщине слоя от $y = 0$ до $y = h$:

\begin{equation}
\int_0^h \frac{\partial u}{\partial x} \, dy + \int_0^h \frac{\partial v}{\partial y} \, dy + \int_0^h \frac{\partial w}{\partial z} \, dy = 0.
\end{equation}

Используя правило Лейбница для дифференцирования интеграла:

\begin{equation}
\frac{\partial}{\partial x}\int_0^h u \, dy = \int_0^h \frac{\partial u}{\partial x} \, dy + u|_{y=h}\frac{\partial h}{\partial x},
\end{equation}

и аналогично для $z$. Таким образом, интегралы производных в уравнении неразрывности выражаются как:

\begin{equation}
\int_0^h \frac{\partial u}{\partial x}\,dy = \frac{\partial}{\partial x}\!\int_0^h u\,dy
  - u\big|_{y=h}\,\frac{\partial h}{\partial x}, \qquad
\int_0^h \frac{\partial w}{\partial z}\,dy = \frac{\partial}{\partial z}\!\int_0^h w\,dy
  - w\big|_{y=h}\,\frac{\partial h}{\partial z}.
\label{eq:leibniz_applied}
\end{equation}

Для второго интеграла:

\begin{equation}
\int_0^h \frac{\partial v}{\partial y} \, dy = v|_{y=h} - v|_{y=0} = V_2 - V_1.
\end{equation}

Кинематическое условие на верхней поверхности $y = h(x,z,t)$:

\begin{equation}
V_2 = \frac{\partial h}{\partial t} + U_2\frac{\partial h}{\partial x} + W_2\frac{\partial h}{\partial z}.
\label{eq:kinematic_upper}
\end{equation}

Нижняя поверхность полагается плоской и неподвижной в направлении $y$, откуда $V_1 = 0$. Здесь использовано условие прилипания на верхней поверхности: $u(h) = U_2$, $w(h) = W_2$, $v(h) = V_2$. Подставляя~\eqref{eq:leibniz_applied} и кинематическое условие~\eqref{eq:kinematic_upper} в интегрированное уравнение неразрывности, после сокращения членов $U_2\,\partial h/\partial x$ и $W_2\,\partial h/\partial z$, приходим к уравнению:

Используя определения расходов~\eqref{eq:flow_rate_x} и~\eqref{eq:flow_rate_z}:

\begin{equation}
\frac{\partial q_x}{\partial x} + \frac{\partial q_z}{\partial z} + \frac{\partial h}{\partial t} = 0.
\label{eq:continuity_integrated}
\end{equation}

Уравнение~\eqref{eq:continuity_integrated} допускает наглядную интерпретацию в терминах контрольного объёма (рисунок~\ref{fig:control_volume}): изменение толщины слоя $\partial h/\partial t$ компенсируется потоками $q_x$ и $q_z$ через боковые грани элемента $dx \times dz$.

% [РИСУНОК: Контрольный объём смазочного слоя]
% Должен показывать:
% - Элемент dx × dz, толщина h
% - Потоки q_x и q_x + ∂q_x/∂x·dx через левую и правую грани
% - Потоки q_z и q_z + ∂q_z/∂z·dz через переднюю и заднюю грани
% - Стрелка ∂h/∂t сверху (выжимание)
% - Формула баланса: ∂q_x/∂x + ∂q_z/∂z + ∂h/∂t = 0
\begin{figure}[!ht]
\centering
\fbox{\parbox{0.8\textwidth}{\centering
\vspace{3cm}
ПЛЕЙСХОЛДЕР ДЛЯ РИСУНКА\\
Контрольный объём смазочного слоя:\\
элемент $dx \times dz$, толщина $h$,\\
потоки $q_x$, $q_z$ через боковые грани,\\
выжимание $\partial h/\partial t$
\vspace{3cm}
}}
\caption{Контрольный объём смазочного слоя и баланс расходов}
\label{fig:control_volume}
\end{figure}

Подставляя выражения для $q_x$ и $q_z$:

\begin{equation}
\frac{\partial}{\partial x}\left[\frac{h}{2}(U_1 + U_2) - \frac{h^3}{12\mu}\frac{\partial p}{\partial x}\right] + \frac{\partial}{\partial z}\left[\frac{h}{2}(W_1 + W_2) - \frac{h^3}{12\mu}\frac{\partial p}{\partial z}\right] + \frac{\partial h}{\partial t} = 0.
\end{equation}

Раскрывая производные:

\begin{equation}
\frac{\partial}{\partial x}\left(\frac{h^3}{12\mu}\frac{\partial p}{\partial x}\right) + \frac{\partial}{\partial z}\left(\frac{h^3}{12\mu}\frac{\partial p}{\partial z}\right) = \frac{\partial}{\partial x}\left[\frac{h}{2}(U_1 + U_2)\right] + \frac{\partial}{\partial z}\left[\frac{h}{2}(W_1 + W_2)\right] + \frac{\partial h}{\partial t}.
\label{eq:reynolds_full}
\end{equation}

Уравнение~\eqref{eq:reynolds_full} является \textbf{обобщённым уравнением Рейнольдса} для случая нестационарного течения с подвижными границами произвольной формы.

Для стационарного случая ($\partial h/\partial t = 0$) и при условии $W_1 = W_2 = 0$ (движение только в направлении $x$) уравнение упрощается:

\begin{equation}
\frac{\partial}{\partial x}\left(\frac{h^3}{12\mu}\frac{\partial p}{\partial x}\right) + \frac{\partial}{\partial z}\left(\frac{h^3}{12\mu}\frac{\partial p}{\partial z}\right) = \frac{U_1 + U_2}{2}\frac{\partial h}{\partial x}.
\label{eq:reynolds_steady}
\end{equation}

Уравнение~\eqref{eq:reynolds_steady} представляет собой \textbf{классическое уравнение Рейнольдса} для гидродинамической смазки.

В случае одной движущейся поверхности (например, $U_1 = U$, $U_2 = 0$):

\begin{equation}
\frac{\partial}{\partial x}\left(\frac{h^3}{\mu}\frac{\partial p}{\partial x}\right) + \frac{\partial}{\partial z}\left(\frac{h^3}{\mu}\frac{\partial p}{\partial z}\right) = 6U\frac{\partial h}{\partial x}.
\label{eq:reynolds_one_surface}
\end{equation}

Полученное уравнение Рейнольдса~\eqref{eq:reynolds_one_surface} является основным уравнением теории гидродинамической смазки и связывает распределение давления $p(x, z)$ в смазочном слое с геометрией зазора $h(x, z)$ и скоростью движения поверхности $U$.

Следует отметить, что при решении уравнения Рейнольдса в областях расходящегося зазора могут возникать зоны отрицательного давления, не имеющие физического смысла. Для корректного описания таких областей необходима массо-сохраняющая модель кавитации, которая рассматривается в разделе~2.3 на основе подхода Jakobsson--Floberg--Olsson (JFO).


\subsection{Упорная постановка задачи}

% Упорный подшипник (thrust bearing)
% Полная постановка задачи: геометрия, кинематика, уравнение Рейнольдса,
% граничные условия, расходы, целевые функционалы

Рассмотрим упорный (осевой) подшипник скольжения -- узел, в котором осевая нагрузка воспринимается тонким слоем вязкой жидкости, заключённым между двумя поверхностями: вращающимся диском (ротором) и неподвижной опорной поверхностью (колодкой, или сектором). Несущая способность подшипника обусловлена гидродинамическим давлением, которое генерируется в сходящемся зазоре при относительном движении поверхностей.

Упорные подшипники скольжения находят широкое применение в инженерной практике. В турбинах гидро- и теплоэнергетических установок они воспринимают осевое усилие, создаваемое рабочим телом на рабочем колесе. В центробежных насосах и компрессорах упорные подшипники фиксируют осевое положение ротора, предотвращая контакт вращающихся и неподвижных элементов. В тяжёлых прокатных станах и вертикальных гидрогенераторах упорные подшипники несут вес ротора и всего рабочего оборудования.

Цель настоящей постановки состоит в следующем: для заданной геометрии зазора $h(r,\theta)$ и режима движения поверхностей получить распределение давления $p(r,\theta)$ в смазочном слое, а затем вычислить несущую способность $W$, расход смазки $Q$, момент трения $M$ и другие эксплуатационные характеристики. Все допущения, принятые при выводе уравнения Рейнольдса в подразделе~2.1.1, сохраняют силу: жидкость ньютоновская и несжимаемая, течение ламинарное и стационарное, давление постоянно по толщине смазочного слоя. Далее рассматривается один сектор (колодка) подшипника; полный подшипник состоит из $N$ одинаковых секторов, расположенных равномерно по окружности.


\subsubsection*{Геометрия и система координат}

Для описания упорного подшипника естественно использовать цилиндрические координаты $(r, \theta, y)$:
\begin{itemize}
\item $r$ -- радиальная координата в плоскости подшипника;
\item $\theta$ -- окружная координата, отсчитываемая по направлению вращения;
\item $y$ -- координата по толщине смазочного слоя, направленная перпендикулярно рабочим поверхностям (от $y = 0$ на неподвижной колодке до $y = h$ на вращающемся диске).
\end{itemize}

Область расчёта для одного сектора определяется следующим образом:
\begin{equation}
r \in [R_{\mathrm{in}},\; R_{\mathrm{out}}], \qquad \theta \in [\theta_0,\; \theta_1],
\label{eq:thrust_domain}
\end{equation}
где $R_{\mathrm{in}}$ и $R_{\mathrm{out}}$ -- внутренний и наружный радиусы рабочей зоны подшипника, $\theta_0$ и $\theta_1$ -- угловые границы сектора. Угловая протяжённость сектора обозначается
\begin{equation}
\beta = \theta_1 - \theta_0.
\label{eq:thrust_beta}
\end{equation}
Граница $\theta_0$ соответствует входной кромке (leading edge), а $\theta_1$ -- выходной кромке (trailing edge) по направлению вращения диска. Толщина смазочного слоя в каждой точке расчётной области составляет $y \in [0,\; h(r,\theta)]$.

Для подшипника из $N$ одинаковых секторов, расположенных равномерно по окружности, угловая протяжённость одного сектора определяется как $\beta = 2\pi/N$ за вычетом углового зазора между соседними колодками. Расчётная область одного сектора схематически изображена на рисунке~\ref{fig:thrust_bearing}.

% [РИСУНОК: Схема упорного подшипника — вид сверху на сектор]
% Должен показывать:
% - Вид сверху на один сектор подшипника
% - Оси r, θ, направление вращения ω
% - Границы R_in, R_out, θ_0, θ_1
% - Угловую протяжённость β
\begin{figure}[!ht]
\centering
\fbox{\parbox{0.8\textwidth}{\centering
\vspace{3cm}
ПЛЕЙСХОЛДЕР ДЛЯ РИСУНКА\\
Схема упорного подшипника -- вид сверху на сектор\\
Оси $r$, $\theta$, направление вращения $\omega$,\\
границы $R_{\mathrm{in}}$, $R_{\mathrm{out}}$, $\theta_0$, $\theta_1$
\vspace{3cm}
}}
\caption{Схема упорного подшипника: вид сверху на один сектор}
\label{fig:thrust_bearing}
\end{figure}


\subsubsection*{Кинематика движения поверхностей}

Нижняя поверхность ($y = 0$) представляет собой неподвижную колодку, верхняя поверхность ($y = h$) -- вращающийся диск с постоянной угловой скоростью $\omega$. Тангенциальная (окружная) скорость скольжения вращающегося диска в точке с радиальной координатой $r$ составляет:
\begin{equation}
U_\theta(r) = \omega\,r.
\label{eq:thrust_U_theta}
\end{equation}
Радиальные скорости обеих поверхностей равны нулю: $U_{r,1} = U_{r,2} = 0$. В стационарном режиме нормальные скорости поверхностей также обращаются в нуль: $V_1 = V_2 = 0$, что исключает эффект выжимания ($\partial h/\partial t = 0$).

Динамическая вязкость смазки полагается постоянной: $\mu = \mathrm{const}$ (допущение~6 из подраздела~2.1.1). Влияние температурной зависимости вязкости $\mu(T)$ на характеристики подшипника рассматривается отдельно в главе~4.

Вращение диска создаёт тангенциальный увлекающий поток (куэттовскую составляющую течения): жидкость, увлекаемая подвижной поверхностью за счёт условия прилипания, движется в окружном направлении. Когда такой поток проходит через участок зазора с уменьшающейся толщиной, условие неразрывности (сохранения массы) приводит к росту давления в смазочном слое -- именно этот механизм, называемый клиновым эффектом, обеспечивает несущую способность подшипника.


\subsubsection*{Закон толщины смазочного слоя}

В общем случае толщина смазочного слоя является заданной функцией координат:
\begin{equation}
h = h(r, \theta),
\label{eq:thrust_h_general}
\end{equation}
определяющей профиль зазора между колодкой и вращающимся диском. В разделе~2.2 функция $h(r,\theta)$ будет расширена для учёта детерминированных микроструктур рабочей поверхности.

Для тестирования численных методов и верификации результатов широко используется базовая геометрия -- линейный клин, в котором толщина зазора линейно изменяется по окружной координате:
\begin{equation}
h(r,\theta) = h_{\mathrm{in}} + (h_{\mathrm{out}} - h_{\mathrm{in}})\,\frac{\theta - \theta_0}{\theta_1 - \theta_0},
\label{eq:thrust_h_wedge}
\end{equation}
где $h_{\mathrm{in}} = h(\theta_0)$ -- толщина смазочного слоя на входной кромке сектора, $h_{\mathrm{out}} = h(\theta_1)$ -- толщина на выходной кромке. Необходимым условием генерации давления является сходящийся характер зазора по направлению движения, т.\,е.
\begin{equation}
h_{\mathrm{in}} > h_{\mathrm{out}}.
\label{eq:thrust_convergence_condition}
\end{equation}
В частном случае $h_{\mathrm{in}} = h_{\mathrm{out}}$ (плоскопараллельный зазор) правая часть уравнения Рейнольдса обращается в нуль и давление не генерируется.

В формуле~\eqref{eq:thrust_h_wedge} величины $h_{\mathrm{in}}$ и $h_{\mathrm{out}}$ приняты постоянными по радиальной координате $r$, что соответствует плоскому клину. Минимальная толщина смазочного слоя обозначается
\begin{equation}
h_{\min} = \min_{(r,\theta)} h(r,\theta);
\label{eq:thrust_h_min}
\end{equation}
для линейного клина $h_{\min} = h_{\mathrm{out}}$.

Для характеристики формы клина вводятся безразмерные параметры конвергенции:
\begin{equation}
K = \frac{h_{\mathrm{in}}}{h_{\mathrm{out}}}, \qquad \delta = \frac{h_{\mathrm{in}} - h_{\mathrm{out}}}{h_{\mathrm{out}}} = K - 1.
\label{eq:thrust_convergence}
\end{equation}
При $K = 1$ ($\delta = 0$) зазор является плоскопараллельным и давление не генерируется. Оптимальное значение параметра $K$, обеспечивающее максимальную несущую способность, зависит от соотношения размеров подшипника и определяется численно.

Помимо линейного клина, на практике применяются более сложные профили зазора: ступенчатый (профиль Рэлея), параболический, составной (с плоским участком и клиновой частью). Все перечисленные варианты описываются тем же уравнением Рейнольдса при соответствующем задании функции $h(r,\theta)$. Линейный клин~\eqref{eq:thrust_h_wedge} используется в настоящей работе как базовый верификационный профиль; в реальной колодке профиль зазора $h(r,\theta)$ определяется наклоном опоры и формой рабочей поверхности, что может приводить к зависимости $h$ от радиальной координаты~$r$. Продольное сечение клинового зазора при фиксированном $r$ аналогично общей схеме смазочного слоя (рисунок~\ref{fig:lubrication_scheme}), где горизонтальная координата~$x$ заменяется на~$\theta$, нижняя поверхность соответствует неподвижной колодке, а верхняя~--- плоскому вращающемуся диску со скоростью~$U_\theta = \omega r$. Зазор сужается от~$h_{\mathrm{in}}$ на входном крае колодки до~$h_{\mathrm{out}}$ на выходном.



\subsubsection*{Уравнение Рейнольдса в цилиндрических координатах}

Обобщённое уравнение Рейнольдса~\eqref{eq:reynolds_full}, полученное в подразделе~2.1.1 в декартовых координатах, необходимо переписать в цилиндрических координатах $(r, \theta)$, естественных для задачи об упорном подшипнике. Переход осуществляется введением дуговой координаты $s = r\theta$, вдоль которой производная принимает вид $\partial/\partial s = (1/r)\,\partial/\partial\theta$. Оператор дивергенции и градиента в плоскости $(r, \theta)$ приобретает стандартные множители $1/r$ и $1/r^2$, присутствующие в левой части уравнения Рейнольдса.

В рассматриваемой задаче скорость скольжения направлена по окружной координате~$\theta$: нижняя поверхность неподвижна ($U_{\theta,1} = 0$), верхняя движется со скоростью $U_{\theta,2} = \omega r$; радиальные скорости обеих поверхностей равны нулю. Подставляя эти условия в правую часть обобщённого уравнения Рейнольдса и учитывая, что окружная производная от $h$ содержит метрический множитель $1/r$, получаем:
\begin{equation}
\frac{U_{\theta,1} + U_{\theta,2}}{2}\,\frac{1}{r}\,\frac{\partial h}{\partial \theta} = \frac{\omega r}{2}\,\frac{1}{r}\,\frac{\partial h}{\partial \theta} = \frac{\omega}{2}\,\frac{\partial h}{\partial \theta}.
\label{eq:thrust_rhs_derivation}
\end{equation}

Стационарное уравнение Рейнольдса для несжимаемой ньютоновской жидкости в цилиндрических координатах принимает вид:
\begin{equation}
\frac{1}{r}\frac{\partial}{\partial r}\!\left[r\,\frac{h^3}{12\mu}\,\frac{\partial p}{\partial r}\right] + \frac{1}{r^2}\frac{\partial}{\partial \theta}\!\left[\frac{h^3}{12\mu}\,\frac{\partial p}{\partial \theta}\right] = \frac{\omega}{2}\,\frac{\partial h}{\partial \theta}.
\label{eq:thrust_reynolds}
\end{equation}
Множители $1/r$ и $1/r^2$ перед производными в левой части обусловлены дивергентной формой эллиптического оператора $\nabla \cdot (h^3 \nabla p)$ в цилиндрических координатах. Оператор Лапласа является его частным случаем при $h = \mathrm{const}$.

Умножая обе части уравнения~\eqref{eq:thrust_reynolds} на $12\mu$, получаем эквивалентную форму, удобную для численной реализации:
\begin{equation}
\frac{1}{r}\frac{\partial}{\partial r}\!\left[r\,h^3\,\frac{\partial p}{\partial r}\right] + \frac{1}{r^2}\frac{\partial}{\partial \theta}\!\left[h^3\,\frac{\partial p}{\partial \theta}\right] = 6\mu\omega\,\frac{\partial h}{\partial \theta}.
\label{eq:thrust_reynolds_num}
\end{equation}
Уравнение~\eqref{eq:thrust_reynolds_num} является основным расчётным уравнением для упорного подшипника в настоящей работе.

Левая часть уравнений~\eqref{eq:thrust_reynolds}--\eqref{eq:thrust_reynolds_num} представляет собой эллиптический оператор, описывающий перераспределение давления через вязкий смазочный слой: давление <<диффундирует>> как в радиальном, так и в окружном направлениях, причём <<проводимость>> пропорциональна кубу толщины зазора~$h^3$. Правая часть -- источниковый член, обусловленный изменением толщины зазора в направлении движения (клиновой эффект). При $\partial h/\partial \theta = 0$ (плоскопараллельный зазор) правая часть обращается в нуль, и при однородных граничных условиях уравнение допускает лишь тривиальное решение $p = 0$ -- давление не генерируется.

При наличии детерминированных микроструктур на рабочей поверхности функция $h(r,\theta)$ приобретает быстрые локальные вариации. Формально уравнение~\eqref{eq:thrust_reynolds_num} остаётся тем же, однако функция зазора существенно усложняется, что описывается в разделе~2.2.


\subsubsection*{Граничные условия}

Для замыкания уравнения Рейнольдса~\eqref{eq:thrust_reynolds_num} необходимо задать граничные условия на давление по всему контуру расчётной области~\eqref{eq:thrust_domain}. Рассмотрим три практически важных варианта.

\textit{Базовый вариант (атмосферные кромки).} На всех четырёх границах сектора давление принимается равным атмосферному:
\begin{equation}
\begin{aligned}
p(r,\,\theta_0) &= 0, \qquad & p(r,\,\theta_1) &= 0, \\
p(R_{\mathrm{in}},\,\theta) &= 0, \qquad & p(R_{\mathrm{out}},\,\theta) &= 0.
\end{aligned}
\label{eq:thrust_bc_base}
\end{equation}
Здесь давление отсчитывается от атмосферного ($p = p_{\mathrm{abs}} - p_{\mathrm{atm}}$). Условие~\eqref{eq:thrust_bc_base} соответствует изолированному сектору, кромки которого свободно сообщаются с окружающей средой, а принудительная подача смазки отсутствует. Данный вариант является наиболее распространённым при расчёте несамоустанавливающихся колодок.

\textit{Вариант с давлением подачи.} Если система смазки обеспечивает принудительную подачу жидкости через входную кромку сектора, граничное условие на $\theta = \theta_0$ модифицируется:
\begin{equation}
p(r,\,\theta_0) = p_{\mathrm{sup}}(r),
\label{eq:thrust_bc_supply}
\end{equation}
где $p_{\mathrm{sup}}(r)$ -- заданное давление подачи смазки; в простейшем случае $p_{\mathrm{sup}} = \mathrm{const}$. Остальные граничные условия остаются такими же, как в базовом варианте~\eqref{eq:thrust_bc_base}. Величина $p_{\mathrm{sup}}$ определяется конструкцией системы смазки и режимом работы насоса подачи.

\textit{Кавитация.} При решении уравнения Рейнольдса в областях расходящегося зазора расчётное давление может оказаться отрицательным. Отрицательное давление физически не реализуется: при снижении давления ниже давления насыщенных паров смазки жидкость разрывается и образуется кавитационная зона с парогазовой смесью. В простейшей постановке применяют ограничение $p \geq 0$ (условие Гюмбеля), при котором отрицательные давления обнуляются после решения. Такой подход, однако, нарушает закон сохранения массы на границе кавитационной зоны. Корректная массо-сохраняющая постановка, основанная на модели Jakobsson--Floberg--Olsson (JFO), вводится в разделе~2.3. Граница между зоной полного смазочного слоя и кавитационной зоной заранее неизвестна и определяется в процессе решения как часть задачи.

Следует отметить, что условия Дирихле $p = 0$ на всех четырёх границах сектора представляют лишь один из возможных вариантов замыкания задачи. В зависимости от конструкции узла и схемы смазки могут применяться и другие постановки: периодичность по $\theta$ (при моделировании полного кольцевого сегмента без разрывов), условие непротекания через радиальные границы $q_r = 0$, что эквивалентно $\partial p/\partial r = 0$ (при наличии уплотнений или перемычек), а также смешанные условия при комбинированной подаче смазки. Конкретный набор граничных условий определяется конструкцией узла и схемой подвода смазки.

Перечисленные варианты граничных условий схематически показаны на рисунке~\ref{fig:thrust_bc_schemes}.

% [РИСУНОК: Варианты граничных условий на секторе]
% Должен показывать три подрисунка (a), (b), (c):
% (a) Базовый: p = 0 на всех 4 границах сектора
% (b) С подпиткой: p = p_sup на θ = θ_0, остальные p = 0
% (c) Альтернативный: ∂p/∂r = 0 на радиальных границах (уплотнения)
%     или периодичность по θ
% На каждом подрисунке — контур сектора с обозначением ГУ на каждой стороне
\begin{figure}[!ht]
\centering
\fbox{\parbox{0.8\textwidth}{\centering
\vspace{3.5cm}
ПЛЕЙСХОЛДЕР ДЛЯ РИСУНКА\\
Варианты граничных условий на секторе:\\
(a) $p = 0$ на всех границах;\\
(b) $p = p_{\mathrm{sup}}$ на $\theta = \theta_0$, остальные $p = 0$;\\
(c) $\partial p/\partial r = 0$ на радиальных границах (уплотнения)
\vspace{3.5cm}
}}
\caption{Варианты граничных условий для упорного подшипника: (a)~$p=0$ на всех четырёх границах сектора (атмосферные кромки); (b)~$p=p_{\mathrm{sup}}$ на входной кромке $\theta=\theta_0$ (принудительная подача), остальные границы $p=0$; (c)~$\partial p/\partial r=0$ на радиальных границах (уплотнения), $p=0$ на дуговых кромках}
\label{fig:thrust_bc_schemes}
\end{figure}


\subsubsection*{Объёмные расходы}

Объёмные расходы жидкости через единицу длины определяются путём интегрирования профилей скорости по толщине смазочного слоя, аналогично выражениям~\eqref{eq:flow_rate_x}--\eqref{eq:flow_rate_z}, полученным в подразделе~2.1.1 в декартовых координатах. Ниже приведены формулы, адаптированные к цилиндрической системе координат.

Радиальный расход (на единицу длины по окружной координате):
\begin{equation}
q_r(r,\theta) = -\frac{h^3}{12\mu}\,\frac{\partial p}{\partial r}.
\label{eq:thrust_q_r}
\end{equation}
Поскольку радиальные скорости обеих поверхностей равны нулю, куэттовская составляющая в радиальном направлении отсутствует. Радиальный расход полностью определяется градиентом давления (пуазейлевская составляющая).

Окружной расход (на единицу длины по радиальной координате):
\begin{equation}
q_\theta(r,\theta) = \frac{h\,\omega\,r}{2} - \frac{h^3}{12\mu\,r}\,\frac{\partial p}{\partial \theta}.
\label{eq:thrust_q_theta}
\end{equation}
Первое слагаемое в~\eqref{eq:thrust_q_theta} представляет куэттовскую составляющую расхода, обусловленную увлечением жидкости вращающейся поверхностью. При $U_{\theta,1}=0$ и $U_{\theta,2}=\omega r$ средняя скорость потока составляет $\bar{U}_\theta = \omega r / 2$, а объёмный расход через единицу длины равен $h\,\bar{U}_\theta = h\omega r/2$. Второе слагаемое -- пуазейлевская составляющая; множитель $1/r$ в знаменателе возникает из-за метрического коэффициента при переходе от производной по дуге $\partial/\partial(r\theta)$ к производной по углу $\partial/\partial\theta$.

Суммарные объёмные расходы через радиальные границы подшипника получаются интегрированием удельных расходов по окружной кромке:
\begin{equation}
Q_{\mathrm{out}} = \int_{\theta_0}^{\theta_1} q_r(R_{\mathrm{out}}, \theta)\,R_{\mathrm{out}}\,d\theta, \qquad Q_{\mathrm{in}} = \int_{\theta_0}^{\theta_1} q_r(R_{\mathrm{in}}, \theta)\,R_{\mathrm{in}}\,d\theta.
\label{eq:thrust_Q_radial}
\end{equation}
Множитель $R_{\mathrm{out}}$ (соответственно, $R_{\mathrm{in}}$) перед $d\theta$ обусловлен элементом длины дуги $dl = r\,d\theta$ на окружности радиуса $r$.

Разность $Q_{\mathrm{out}} - Q_{\mathrm{in}}$ определяет утечку смазки через радиальные кромки сектора. Суммарный окружной расход через входную и выходную окружные кромки дополняет баланс массы: полный расход жидкости, поступающей в сектор, равен полному расходу, покидающему его.

Баланс массы в стационарном режиме без выжимания записывается в дифференциальной форме:
\begin{equation}
\frac{1}{r}\,\frac{\partial (r\,q_r)}{\partial r} + \frac{1}{r}\,\frac{\partial q_\theta}{\partial \theta} = 0.
\label{eq:thrust_mass_balance}
\end{equation}
Подстановка выражений~\eqref{eq:thrust_q_r} и~\eqref{eq:thrust_q_theta} в уравнение~\eqref{eq:thrust_mass_balance} приводит к уравнению Рейнольдса~\eqref{eq:thrust_reynolds}, что подтверждает внутреннюю согласованность полученных соотношений.


\subsubsection*{Целевые функционалы}

Результатом решения уравнения Рейнольдса является поле давления $p(r,\theta)$. На его основе вычисляются следующие интегральные характеристики подшипника.

\textit{(а) Несущая способность (осевая нагрузка).} Сила, воспринимаемая одним сектором подшипника:
\begin{equation}
W = \int_{\theta_0}^{\theta_1}\int_{R_{\mathrm{in}}}^{R_{\mathrm{out}}} p(r,\theta)\,r\,dr\,d\theta.
\label{eq:thrust_W}
\end{equation}
Множитель~$r$ в подынтегральном выражении обусловлен элементом площади в цилиндрических координатах $dA = r\,dr\,d\theta$. Для подшипника, состоящего из $N$ одинаковых секторов, полная воспринимаемая нагрузка составляет $W_{\mathrm{total}} = N\,W$.

\textit{(б) Координаты центра давления.} Точка приложения равнодействующей давления определяется через декартовы моменты в плоскости подшипника:
\begin{equation}
\begin{aligned}
x_c &= \frac{1}{W}\int_{\theta_0}^{\theta_1}\!\int_{R_{\mathrm{in}}}^{R_{\mathrm{out}}}
       p(r,\theta)\,r^2\,\cos\theta\,dr\,d\theta, \\
z_c &= \frac{1}{W}\int_{\theta_0}^{\theta_1}\!\int_{R_{\mathrm{in}}}^{R_{\mathrm{out}}}
       p(r,\theta)\,r^2\,\sin\theta\,dr\,d\theta,
\end{aligned}
\label{eq:thrust_pressure_center_xy}
\end{equation}
откуда координаты центра давления в полярной форме:
\begin{equation}
r_c = \sqrt{x_c^2 + z_c^2}, \qquad \theta_c = \operatorname{atan2}(z_c,\; x_c).
\label{eq:thrust_pressure_center}
\end{equation}
Для узкого сектора ($\beta \ll 2\pi$) выражения~\eqref{eq:thrust_pressure_center_xy} упрощаются: $\cos\theta \approx 1$, $\sin\theta \approx \theta$, и формулы принимают вид $r_c \approx (1/W)\int p\,r^2\,dr\,d\theta$, $\theta_c \approx (1/W)\int p\,\theta\,r\,dr\,d\theta$.

Знание координат центра давления необходимо при проектировании самоустанавливающихся колодок, опорная точка которых должна совпадать с центром давления для обеспечения устойчивой работы.

Качественный вид распределения давления на секторе и положение равнодействующей показаны на рисунке~\ref{fig:thrust_pressure_field}.

% [РИСУНОК: Качественное поле давления на секторе упорного подшипника]
% Должен показывать:
% - Сектор (R_in, R_out, θ_0, θ_1) — вид сверху
% - Контурные линии (или псевдокарту) p(r,θ) с максимумом ближе к выходной кромке
% - Стрелку W (равнодействующая) и точку (r_c, θ_c)
% - Обозначения границ с p = 0
\begin{figure}[!ht]
\centering
\fbox{\parbox{0.8\textwidth}{\centering
\vspace{3cm}
ПЛЕЙСХОЛДЕР ДЛЯ РИСУНКА\\
Качественное распределение давления $p(r,\theta)$ на секторе:\\
изолинии давления, максимум ближе к выходной кромке,\\
равнодействующая $W$ и центр давления $(r_c, \theta_c)$
\vspace{3cm}
}}
\caption{Качественное распределение давления на секторе упорного подшипника}
\label{fig:thrust_pressure_field}
\end{figure}

\textit{(в) Профиль окружной скорости и сдвиговые напряжения.} Профиль скорости в окружном направлении записывается по аналогии с~\eqref{eq:velocity_profile_u}, с заменой $x \to r\theta$ и подстановкой $U_1 = 0$, $U_2 = \omega r$:
\begin{equation}
u_\theta(y) = \frac{y}{h}\,\omega r + \frac{1}{2\mu}\,\frac{1}{r}\,\frac{\partial p}{\partial \theta}\,y(y - h).
\label{eq:thrust_velocity_theta}
\end{equation}
Первый член описывает линейный (куэттовский) профиль, обусловленный движением верхней поверхности; второй -- параболическую (пуазейлевскую) добавку, определяемую градиентом давления.

Сдвиговое напряжение на вращающейся поверхности ($y = h$) определяется дифференцированием~\eqref{eq:thrust_velocity_theta}:
\begin{equation}
\tau_w(r,\theta) = \mu\,\frac{\partial u_\theta}{\partial y}\bigg|_{y=h} = \frac{\mu\,\omega\,r}{h} + \frac{h}{2r}\,\frac{\partial p}{\partial \theta}.
\label{eq:thrust_tau_w}
\end{equation}
В выражении~\eqref{eq:thrust_tau_w} первое слагаемое представляет вязкое (куэттовское) трение, пропорциональное скорости скольжения и обратно пропорциональное толщине зазора. Второе слагаемое -- вклад градиента давления (пуазейлевская составляющая), который может как увеличивать, так и уменьшать суммарное касательное напряжение в зависимости от знака $\partial p/\partial \theta$.

\textit{(г) Момент трения.} Суммарный момент сил трения относительно оси вращения подшипника:
\begin{equation}
M = \int_{\theta_0}^{\theta_1}\int_{R_{\mathrm{in}}}^{R_{\mathrm{out}}} \tau_w(r,\theta)\,r^2\,dr\,d\theta.
\label{eq:thrust_M}
\end{equation}
Множитель $r^2$ в подынтегральном выражении складывается из двух сомножителей: $r$ от элемента площади $dA = r\,dr\,d\theta$ и ещё один $r$ -- плечо касательной силы относительно оси вращения.

\textit{(д) Мощность потерь на трение:}
\begin{equation}
P_{\mathrm{loss}} = \omega\,M.
\label{eq:thrust_P_loss}
\end{equation}
Величина $P_{\mathrm{loss}}$ определяет тепловыделение в смазочном слое и является входным параметром для теплового расчёта подшипника.

\textit{(е) Коэффициент трения.} Суммарная тангенциальная сила трения на вращающейся поверхности:
\begin{equation}
F_t = \int_{\theta_0}^{\theta_1}\int_{R_{\mathrm{in}}}^{R_{\mathrm{out}}} \tau_w(r,\theta)\,r\,dr\,d\theta.
\label{eq:thrust_F_t}
\end{equation}
Коэффициент трения определяется как отношение силы трения к несущей способности:
\begin{equation}
f = \frac{F_t}{W}.
\label{eq:thrust_f_strict}
\end{equation}
В инженерных расчётах часто используют приближённое соотношение $F_t \approx M/R_m$, что приводит к оценке
\begin{equation}
f \approx \frac{M}{R_m\,W},
\label{eq:thrust_f}
\end{equation}
где $R_m = (R_{\mathrm{in}} + R_{\mathrm{out}})/2$ -- средний радиус рабочей зоны подшипника, принимаемый как характерный радиус приложения равнодействующей силы трения. Меньшие значения $f$ соответствуют более низким потерям на трение при заданной несущей способности.

\textit{(ж) Минимальная толщина смазочного слоя:}
\begin{equation}
h_{\min} = \min_{(r,\theta)} h(r,\theta).
\label{eq:thrust_h_min_target}
\end{equation}
Величина $h_{\min}$ определяет запас по толщине зазора и, следовательно, безопасность подшипника от контакта поверхностей. При проектировании $h_{\min}$ должна превышать суммарную шероховатость рабочих поверхностей с определённым коэффициентом запаса. Основные обозначения, введённые в настоящем подразделе, сведены в таблицу~\ref{tab:thrust_notations}.

\begin{table}[!ht]
\centering
\caption{Основные обозначения для упорного подшипника}
\label{tab:thrust_notations}
\begin{tabular}{|l|l|l|}
\hline
\textbf{Обозначение} & \textbf{Величина} & \textbf{Размерность} \\
\hline
$R_{\mathrm{in}}$, $R_{\mathrm{out}}$ & Внутренний и наружный радиусы & м \\
$\theta_0$, $\theta_1$, $\beta$ & Границы и угол сектора & рад \\
$\omega$ & Угловая скорость & рад/с \\
$\mu$ & Динамическая вязкость & Па$\cdot$с \\
$h(r,\theta)$ & Толщина смазочного слоя & м \\
$h_{\mathrm{in}}$, $h_{\mathrm{out}}$ & Толщина на входной и выходной кромках & м \\
$h_{\min}$ & Минимальная толщина зазора & м \\
$K$, $\delta$ & Параметры конвергенции клина & -- \\
$p(r,\theta)$ & Давление в смазочном слое & Па \\
$W$ & Несущая способность (нагрузка) & Н \\
$M$ & Момент трения & Н$\cdot$м \\
$P_{\mathrm{loss}}$ & Мощность потерь на трение & Вт \\
$Q_{\mathrm{in}}$, $Q_{\mathrm{out}}$ & Расходы через радиальные границы & м$^3$/с \\
$f$ & Коэффициент трения & -- \\
\hline
\end{tabular}
\end{table}

Сформулированная задача сводится к решению эллиптического уравнения Рейнольдса~\eqref{eq:thrust_reynolds_num} в области~\eqref{eq:thrust_domain} при заданном профиле зазора $h(r,\theta)$ и граничных условиях на давление. Структура уравнения (эллиптический оператор в левой части, источниковый член в правой) обеспечивает единственность решения при корректно поставленных граничных условиях Дирихле в отсутствие кавитации, т.\,е. для линейной постановки без ограничений на знак давления.

Численная дискретизация уравнения~\eqref{eq:thrust_reynolds_num} методом конечных разностей и алгоритм учёта кавитации на основе модели JFO рассматриваются в главе~3. При наличии микроструктур на рабочей поверхности функция зазора $h(r,\theta)$ приобретает быстрые локальные вариации, параметризация которых описывается в разделе~2.2.


\subsection{Радиальная постановка задачи}

% Радиальный подшипник (journal bearing)

Рассмотрим радиальный (опорный) подшипник скольжения -- узел, в котором цилиндрический вал (цапфа) вращается внутри неподвижной втулки (вкладыша), а тонкий слой вязкой жидкости, заполняющий зазор между валом и втулкой, воспринимает радиальную нагрузку, перпендикулярную оси вала. Несущая способность подшипника обусловлена гидродинамическим давлением, генерируемым в сходящейся части зазора при вращении вала.

Радиальные подшипники скольжения широко применяются в инженерной практике. В двигателях внутреннего сгорания они служат опорами коленчатого вала и шатунов. В паровых и газовых турбоагрегатах радиальные подшипники обеспечивают устойчивую работу ротора при высоких частотах вращения. В центробежных насосах и компрессорах они воспринимают радиальные усилия, возникающие от неуравновешенности и аэро- или гидродинамических сил. В шарошечных долотах для бурения нефтяных и газовых скважин радиальные подшипники работают в условиях высоких нагрузок и ограниченного пространства.

Цель настоящей постановки состоит в следующем: для заданной геометрии зазора $h(\theta)$, определяемой эксцентриситетом вала, и режима движения поверхностей получить распределение давления $p(\theta, z)$ в смазочном слое, а затем вычислить несущую способность $W$, момент трения $M$, расход смазки $Q$ и другие эксплуатационные характеристики. Все допущения, принятые при выводе уравнения Рейнольдса в подразделе~2.1.1, сохраняют силу: жидкость ньютоновская и несжимаемая, течение ламинарное и стационарное, давление постоянно по толщине смазочного слоя. В отличие от упорного подшипника (подраздел~2.1.2), где нагрузка осевая и расчётная область представляет собой сектор в плоскости $(r, \theta)$, в радиальном подшипнике нагрузка действует перпендикулярно оси вала, а расчётная область формируется на <<развёрнутой>> цилиндрической поверхности зазора в координатах $(\theta, z)$.


\subsubsection*{Геометрия и система координат}

Основными геометрическими параметрами радиального подшипника являются:
\begin{itemize}
\item $R$ -- радиус вала (цапфы);
\item $R_b$ -- радиус внутренней поверхности втулки (вкладыша);
\item $c = R_b - R$ -- радиальный зазор;
\item $e$ -- эксцентриситет, т.\,е. расстояние между центром втулки $O_b$ и центром вала $O_j$;
\item $\varepsilon = e/c$ -- относительный эксцентриситет ($0 \leq \varepsilon < 1$).
\end{itemize}

Прямая, соединяющая центр втулки $O_b$ с центром вала $O_j$, называется линией центров. Положение вала внутри втулки характеризуется двумя величинами: эксцентриситетом~$e$ (или относительным эксцентриситетом~$\varepsilon$) и углом нагрузки $\varphi$ (attitude angle) -- углом между линией центров и направлением результирующей силы давления.

Для описания задачи используется <<развёрнутая>> цилиндрическая система координат:
\begin{itemize}
\item $\theta$ -- окружная координата, отсчитываемая от точки максимального зазора (диаметрально противоположной точке минимального зазора) в направлении вращения вала; при этом $h(\theta = 0) = h_{\max}$ и $h(\theta = \pi) = h_{\min}$;
\item $z$ -- осевая координата вдоль оси вала, $z \in [-L/2,\; L/2]$, где $L$ -- длина подшипника (ширина рабочей поверхности втулки);
\item $y$ -- координата по толщине смазочного слоя, направленная от неподвижной втулки ($y = 0$) к поверхности вала ($y = h$).
\end{itemize}

При таком выборе начала отсчёта угла~$\theta$ выражение для толщины зазора $h(\theta)$ приобретает наиболее простую форму (см.\ далее), а угол нагрузки $\varphi$ определяется как результат расчёта:
\begin{equation}
\varphi = \operatorname{atan2}(W_t,\; W_r),
\label{eq:journal_attitude}
\end{equation}
где $W_r$ и $W_t$ -- компоненты результирующей силы давления вдоль и перпендикулярно линии центров соответственно (определяются ниже в подразделе <<Целевые функционалы>>).

Область расчёта представляет собой прямоугольник:
\begin{equation}
\theta \in [0,\; 2\pi], \qquad z \in [-L/2,\; L/2].
\label{eq:journal_domain}
\end{equation}
В отличие от упорного подшипника (подраздел~2.1.2), здесь область по окружной координате~$\theta$ охватывает полную окружность, что обеспечивает периодичность решения. <<Третья>> координата -- осевая~$z$ (а не радиальная~$r$, как в упорном подшипнике). Поперечное сечение подшипника с основными обозначениями показано на рисунке~\ref{fig:journal_bearing}.

% [РИСУНОК: Схема радиального подшипника]
% Должен показывать:
% - Поперечное сечение: вал радиуса R, втулка радиуса R_b = R + c
% - Центры: O_b (втулки) и O_j (вала), эксцентриситет e
% - Линия центров O_b — O_j
% - Угол θ от линии центров в направлении вращения
% - h_min = c(1-ε) и h_max = c(1+ε)
% - Направление вращения ω
% - Вектор результирующей нагрузки W и attitude angle φ
\begin{figure}[!ht]
\centering
\fbox{\parbox{0.8\textwidth}{\centering
\vspace{3cm}
ПЛЕЙСХОЛДЕР ДЛЯ РИСУНКА\\
Поперечное сечение радиального подшипника:\\
вал радиуса $R$, втулка радиуса $R_b = R + c$,\\
центры $O_b$ и $O_j$, эксцентриситет $e$,\\
$h_{\min} = c(1-\varepsilon)$, $h_{\max} = c(1+\varepsilon)$,\\
направление вращения $\omega$, нагрузка $W$, угол $\varphi$
\vspace{3cm}
}}
\caption{Поперечное сечение радиального подшипника с основными обозначениями}
\label{fig:journal_bearing}
\end{figure}

Развёрнутый вид расчётной области представлен на рисунке~\ref{fig:journal_unwrapped}: зазор <<разрезан>> вдоль образующей при $\theta = 0$ и развёрнут в прямоугольник, верхняя граница которого соответствует профилю $h(\theta)$.

% [РИСУНОК: Развёрнутый вид зазора радиального подшипника]
% Должен показывать:
% - Развёрнутый вид зазора h(θ) как прямоугольник θ × z
% - Синусоидальный профиль h(θ) сверху
% - Границы z = ±L/2
% - Обозначения ГУ на границах
\begin{figure}[!ht]
\centering
\fbox{\parbox{0.8\textwidth}{\centering
\vspace{3cm}
ПЛЕЙСХОЛДЕР ДЛЯ РИСУНКА\\
Развёрнутый вид зазора радиального подшипника:\\
область $(\theta, z)$, профиль $h(\theta) = c(1+\varepsilon\cos\theta)$,\\
граничные условия на торцах $z = \pm L/2$,\\
периодичность по $\theta$
\vspace{3cm}
}}
\caption{Развёрнутый вид расчётной области радиального подшипника}
\label{fig:journal_unwrapped}
\end{figure}


\subsubsection*{Кинематика движения поверхностей}

Вал вращается с постоянной угловой скоростью~$\omega$ внутри неподвижной втулки. Окружная скорость поверхности вала составляет:
\begin{equation}
U = \omega\,R.
\label{eq:journal_U}
\end{equation}
Осевые скорости обеих поверхностей равны нулю. В стационарном режиме положение вала фиксировано: $\varepsilon = \mathrm{const}$, и нормальные скорости поверхностей обращаются в нуль, что исключает эффект выжимания ($\partial h/\partial t = 0$). При нестационарном анализе (динамика ротора) эксцентриситет $\varepsilon$ и угол нагрузки $\varphi$ зависят от времени, и в правой части уравнения Рейнольдса появляется squeeze-term $\partial h/\partial t$; такой случай выходит за рамки настоящего раздела.

Динамическая вязкость смазки полагается постоянной: $\mu = \mathrm{const}$ (допущение~6 из подраздела~2.1.1). Влияние температурной зависимости вязкости $\mu(T)$ рассматривается в главе~4.

Вращение вала создаёт увлекающий (куэттовский) поток смазки в окружном направлении. В зоне сходящегося зазора ($0 < \theta < \pi$, где толщина уменьшается от $h_{\max}$ к $h_{\min}$) условие неразрывности приводит к росту давления. В зоне расходящегося зазора ($\pi < \theta < 2\pi$) давление стремится к отрицательным значениям, однако физически отрицательное давление не реализуется: жидкость разрывается и образуется кавитационная зона. Именно сочетание клинового эффекта и кавитации определяет распределение давления и несущую способность радиального подшипника.


\subsubsection*{Закон толщины смазочного слоя}

Толщина смазочного слоя определяется геометрией двух эксцентричных цилиндров -- вала и втулки. При условии тонкого зазора ($c/R \ll 1$, что согласуется с допущением~3 из подраздела~2.1.1) и пренебрежении членами порядка $(c/R)^2$ и выше, толщина зазора записывается в виде:
\begin{equation}
h(\theta) = c\,(1 + \varepsilon\,\cos\theta),
\label{eq:journal_h}
\end{equation}
где $c = R_b - R$ -- радиальный зазор, $\varepsilon = e/c$ -- относительный эксцентриситет.

Из формулы~\eqref{eq:journal_h} следует:
\begin{itemize}
\item максимальная толщина зазора $h_{\max} = c(1 + \varepsilon)$ достигается при $\theta = 0$ (точка, диаметрально противоположная минимальному зазору);
\item минимальная толщина зазора $h_{\min} = c(1 - \varepsilon)$ достигается при $\theta = \pi$ (точка наибольшего сближения вала и втулки).
\end{itemize}

Толщина зазора~\eqref{eq:journal_h} не зависит от осевой координаты~$z$, что соответствует осевой однородности геометрии (вал и втулка имеют цилиндрическую форму с параллельными осями). Зависимость $h$ от $z$ может возникать при перекосе вала (misalignment), однако этот случай в настоящей работе не рассматривается.

В общем случае, при наличии детерминированных микроструктур на рабочей поверхности, толщина смазочного слоя записывается как суперпозиция:
\begin{equation}
h(\theta, z) = c\,(1 + \varepsilon\,\cos\theta) + \Delta h(\theta, z),
\label{eq:journal_h_textured}
\end{equation}
где $\Delta h(\theta, z)$ -- добавка, описывающая геометрию микроструктуры (раздел~2.2).

Для характеристики геометрии радиального подшипника вводятся следующие безразмерные параметры:
\begin{itemize}
\item $\varepsilon = e/c$ -- относительный эксцентриситет, являющийся основным параметром нагружения ($\varepsilon = 0$ соответствует концентричному расположению, $\varepsilon \to 1$ -- контакту поверхностей);
\item $\psi = c/R$ -- относительный зазор (для типичных подшипников $\psi \sim 10^{-3}$);
\item $L/D$ -- отношение длины подшипника к диаметру ($D = 2R$), определяющее степень влияния осевых утечек: при $L/D > 1$ подшипник считается длинным, при $L/D < 0{,}5$ -- коротким, промежуточные значения соответствуют подшипнику конечной длины.
\end{itemize}


\subsubsection*{Уравнение Рейнольдса для радиального подшипника}

Обобщённое уравнение Рейнольдса~\eqref{eq:reynolds_full}, полученное в подразделе~2.1.1 в декартовых координатах, необходимо переписать для цилиндрической поверхности зазора радиального подшипника. Введём дуговую координату $x = R\theta$ (расстояние вдоль окружности вала), вдоль которой производная принимает вид $\partial/\partial x = (1/R)\,\partial/\partial\theta$. Осевая координата~$z$ остаётся без изменений.

В рассматриваемой задаче втулка неподвижна ($U_1 = 0$), вал движется с окружной скоростью $U_2 = \omega R = U$; осевые скорости обеих поверхностей равны нулю ($W_1 = W_2 = 0$). В стационарном режиме ($\partial h/\partial t = 0$) правая часть обобщённого уравнения Рейнольдса содержит единственный член:
\begin{equation}
\frac{U_1 + U_2}{2}\,\frac{1}{R}\,\frac{\partial h}{\partial \theta} = \frac{U}{2R}\,\frac{\partial h}{\partial \theta} = \frac{\omega}{2}\,\frac{\partial h}{\partial \theta}.
\label{eq:journal_rhs_derivation}
\end{equation}

Стационарное уравнение Рейнольдса для несжимаемой ньютоновской жидкости в координатах $(\theta, z)$ принимает вид:
\begin{equation}
\frac{1}{R^2}\frac{\partial}{\partial \theta}\!\left[\frac{h^3}{12\mu}\,\frac{\partial p}{\partial \theta}\right] + \frac{\partial}{\partial z}\!\left[\frac{h^3}{12\mu}\,\frac{\partial p}{\partial z}\right] = \frac{\omega}{2}\,\frac{\partial h}{\partial \theta}.
\label{eq:journal_reynolds}
\end{equation}
Множитель $1/R^2$ перед первым слагаемым в левой части обусловлен двукратным применением оператора $\partial/\partial x = (1/R)\,\partial/\partial\theta$ при переходе от декартовой координаты~$x$ к угловой~$\theta$.

Умножая обе части уравнения~\eqref{eq:journal_reynolds} на $12\mu$, получаем эквивалентную форму, удобную для численной реализации:
\begin{equation}
\frac{1}{R^2}\frac{\partial}{\partial \theta}\!\left[h^3\,\frac{\partial p}{\partial \theta}\right] + \frac{\partial}{\partial z}\!\left[h^3\,\frac{\partial p}{\partial z}\right] = 6\mu\omega\,\frac{\partial h}{\partial \theta}.
\label{eq:journal_reynolds_num}
\end{equation}
Уравнение~\eqref{eq:journal_reynolds_num} является основным расчётным уравнением для радиального подшипника в настоящей работе.

Подставляя конкретное выражение для толщины зазора~\eqref{eq:journal_h}, получаем производную в правой части:
\begin{equation}
\frac{\partial h}{\partial \theta} = -c\,\varepsilon\,\sin\theta.
\label{eq:journal_dh_dtheta}
\end{equation}
Правая часть уравнения~\eqref{eq:journal_reynolds_num} принимает вид $-6\mu\omega\,c\,\varepsilon\,\sin\theta$. Источниковый член в правой части пропорционален $-\sin\theta$, поэтому максимум генерации давления приходится на $\theta \approx \pi/2$, где скорость сужения зазора наибольшая. Давление генерируется в зоне сходящегося зазора ($0 < \theta < \pi$, $\sin\theta > 0$) и стремится к отрицательным значениям в зоне расходящегося зазора ($\pi < \theta < 2\pi$), где вступает в действие кавитация.

Левая часть уравнений~\eqref{eq:journal_reynolds}--\eqref{eq:journal_reynolds_num} представляет собой эллиптический оператор $\nabla \cdot (h^3 \nabla p)$, записанный в координатах $(\theta, z)$ на цилиндрической поверхности. По структуре он аналогичен оператору в уравнении для упорного подшипника~\eqref{eq:thrust_reynolds_num}, однако здесь вместо координат $(r, \theta)$ используются $(\theta, z)$. Правая часть -- источниковый член, пропорциональный $\sin\theta$: максимум генерации давления приходится на $\theta = \pi/2$. При $\varepsilon = 0$ (концентричное расположение вала и втулки) $\partial h/\partial \theta = 0$, правая часть обращается в нуль и давление не генерируется -- подшипник не несёт нагрузки.

Уравнение~\eqref{eq:journal_reynolds_num} допускает два классических предельных случая, для которых известны аналитические решения.

\textit{Бесконечно длинный подшипник ($L/D \to \infty$).} В этом пределе давление не зависит от осевой координаты: $\partial p/\partial z = 0$, и уравнение~\eqref{eq:journal_reynolds_num} вырождается в обыкновенное дифференциальное уравнение по~$\theta$. Полное аналитическое решение этого ОДУ с периодическими граничными условиями и без учёта кавитации известно как решение Зоммерфельда (Sommerfeld, 1904). Учёт кавитации приводит к половинному решению (half-Sommerfeld), в котором давление обнуляется в зоне расходящегося зазора.

\textit{Бесконечно короткий подшипник ($L/D \to 0$).} В этом пределе осевой градиент давления значительно превышает окружной: $\partial^2 p/\partial z^2 \gg (1/R^2)\,\partial^2 p/\partial \theta^2$, и окружной член в левой части уравнения~\eqref{eq:journal_reynolds_num} может быть опущен. Уравнение вырождается в ОДУ по~$z$ при каждом фиксированном~$\theta$, что допускает аналитическое решение. Это приближение известно как приближение Дюбуа--Оквирка (Dubois--Ocvirk, short-bearing approximation) и даёт удовлетворительные результаты при $L/D < 0{,}5$.

Для подшипника конечной длины ($0{,}5 \lesssim L/D \lesssim 2$) ни один из предельных случаев не обеспечивает достаточной точности, и уравнение~\eqref{eq:journal_reynolds_num} решается численно (глава~3).


\subsubsection*{Граничные условия}

Для замыкания уравнения Рейнольдса~\eqref{eq:journal_reynolds_num} необходимо задать граничные условия на давление по всему контуру расчётной области~\eqref{eq:journal_domain}.

\textit{По окружной координате~$\theta$.} Область по~$\theta$ охватывает полную окружность, поэтому естественным условием является периодичность:
\begin{equation}
p(0, z) = p(2\pi, z), \qquad \frac{\partial p}{\partial \theta}\bigg|_{\theta=0} = \frac{\partial p}{\partial \theta}\bigg|_{\theta=2\pi}.
\label{eq:journal_bc_periodic}
\end{equation}

\textit{По осевой координате~$z$.} На торцах подшипника давление принимается равным атмосферному:
\begin{equation}
p(\theta,\,-L/2) = 0, \qquad p(\theta,\,L/2) = 0,
\label{eq:journal_bc_axial}
\end{equation}
где давление отсчитывается от атмосферного ($p = p_{\mathrm{abs}} - p_{\mathrm{atm}}$), как и в подразделе~2.1.2. При наличии принудительной подачи смазки через осевую канавку условие на соответствующем угле $\theta = \theta_{\mathrm{groove}}$ модифицируется: $p(\theta_{\mathrm{groove}}, z) = p_{\mathrm{sup}}$, где $p_{\mathrm{sup}}$ -- давление подачи. Помимо условий Дирихле на торцах, в специальных конструкциях могут применяться условия непротекания $q_z = 0$, эквивалентные $\partial p/\partial z = 0$ (при наличии уплотнений на торцах подшипника), или смешанные условия, когда на части окружности задаётся давление подачи, а на остальной -- атмосферное давление.

\textit{Кавитация.} Принципиальной особенностью радиального подшипника является неизбежное возникновение кавитации в зоне расходящегося зазора ($\pi < \theta < 2\pi$), где расчётное давление стремится к отрицательным значениям. В простейшей постановке применяют ограничение $p \geq 0$ (условие Гюмбеля), при котором отрицательные давления обнуляются после решения. Более физичным является условие Рейнольдса (Swift--Stieber), требующее одновременного выполнения $p = 0$ и $\partial p/\partial \theta = 0$ на границе кавитации. Оба подхода, однако, не обеспечивают точного сохранения массы. Корректная массо-сохраняющая постановка, основанная на модели Jakobsson--Floberg--Olsson (JFO), вводится в разделе~2.3.

Граница кавитационной зоны является внутренней границей задачи, положение которой заранее неизвестно и определяется в процессе решения. Периодичность задаётся как граничное условие на контуре области~\eqref{eq:journal_domain}, а кавитационная зона является внутренней частью решения. Само поле давления $p(\theta, z)$ при этом остаётся непрерывным и периодическим, хотя и состоит из двух участков: зоны полного смазочного слоя ($p > 0$) и зоны кавитации ($p = 0$). Качественное распределение давления $p(\theta)$ в среднем сечении подшипника ($z = 0$) показано на рисунке~\ref{fig:journal_pressure_profile}.

% [РИСУНОК: Качественный профиль давления в радиальном подшипнике]
% Должен показывать:
% - Ось абсцисс: θ от 0 до 2π
% - p(θ) при z=0: рост в зоне 0 < θ < π, максимум вблизи θ = π/2
% - Зона кавитации: p = 0 при π < θ < 2π (приблизительно)
% - Границы разрыва и восстановления
% - Пунктир: полное решение Зоммерфельда (без кавитации, отрицательные давления)
\begin{figure}[!ht]
\centering
\fbox{\parbox{0.8\textwidth}{\centering
\vspace{3cm}
ПЛЕЙСХОЛДЕР ДЛЯ РИСУНКА\\
Качественное распределение давления $p(\theta)$ при $z=0$:\\
зона генерации давления ($0 < \theta < \pi$),\\
зона кавитации ($p=0$),\\
границы разрыва и восстановления,\\
пунктир -- полное решение Зоммерфельда (без кавитации)
\vspace{3cm}
}}
\caption{Качественное распределение давления в радиальном подшипнике и зона кавитации}
\label{fig:journal_pressure_profile}
\end{figure}


\subsubsection*{Объёмные расходы}

Объёмные расходы жидкости через единицу длины определяются путём интегрирования профилей скорости по толщине смазочного слоя, аналогично подразделу~2.1.2.

Окружной расход (на единицу длины по осевой координате~$z$):
\begin{equation}
q_\theta(\theta, z) = \frac{h\,U}{2} - \frac{h^3}{12\mu\,R}\,\frac{\partial p}{\partial \theta} = \frac{h\,\omega R}{2} - \frac{h^3}{12\mu\,R}\,\frac{\partial p}{\partial \theta}.
\label{eq:journal_q_theta}
\end{equation}
Первое слагаемое представляет куэттовскую составляющую расхода, обусловленную увлечением жидкости вращающимся валом; второе -- пуазейлевскую составляющую, определяемую окружным градиентом давления. Множитель $1/R$ во втором слагаемом обусловлен переходом от производной по дуге $\partial/\partial(R\theta)$ к производной по углу $\partial/\partial\theta$.

Осевой расход (на единицу длины по окружной координате):
\begin{equation}
q_z(\theta, z) = -\frac{h^3}{12\mu}\,\frac{\partial p}{\partial z}.
\label{eq:journal_q_z}
\end{equation}
Поскольку осевые скорости обеих поверхностей равны нулю, осевой расход полностью определяется градиентом давления (пуазейлевская составляющая). Именно осевой расход обусловливает боковую утечку смазки через торцы подшипника.

Суммарная утечка через торцы подшипника определяется интегрированием осевого расхода по окружности:
\begin{equation}
Q_{\mathrm{side}} = \int_0^{2\pi} \bigl[ q_z(\theta,\,L/2) - q_z(\theta,\,-L/2) \bigr]\,R\,d\theta.
\label{eq:journal_Q_side}
\end{equation}
Множитель $R\,d\theta$ соответствует элементу длины дуги на окружности вала. Здесь $q_z(\theta, L/2) > 0$ соответствует потоку, вытекающему через торец $z = L/2$ (по направлению внешней нормали $+\mathbf{e}_z$), а $q_z(\theta, -L/2) < 0$ -- потоку, вытекающему через торец $z = -L/2$ (нормаль $-\mathbf{e}_z$). Обозначая утечки через отдельные торцы как $Q_+ = \int_0^{2\pi} q_z(\theta, L/2)\,R\,d\theta$ и $Q_- = -\int_0^{2\pi} q_z(\theta, -L/2)\,R\,d\theta$, получаем $Q_{\mathrm{side}} = Q_+ + Q_-$.

При симметрии давления относительно средней плоскости подшипника ($p(\theta, z) = p(\theta, -z)$) утечки через оба торца одинаковы, и выражение~\eqref{eq:journal_Q_side} упрощается:
\begin{equation}
Q_{\mathrm{side}} = 2\int_0^{2\pi} q_z(\theta,\,L/2)\,R\,d\theta.
\label{eq:journal_Q_side_sym}
\end{equation}

Боковая утечка является основным механизмом потери смазки в радиальном подшипнике. При уменьшении отношения $L/D$ доля давления, <<разгружаемая>> через торцы, возрастает, что снижает несущую способность. В предельном случае бесконечно короткого подшипника ($L/D \to 0$) осевые утечки полностью доминируют, и именно это обстоятельство позволяет пренебречь окружным членом в уравнении Рейнольдса (приближение Дюбуа--Оквирка).

Баланс массы в стационарном режиме без выжимания записывается в дифференциальной форме (аналогично уравнению~\eqref{eq:thrust_mass_balance} для упорного подшипника):
\begin{equation}
\frac{1}{R}\,\frac{\partial q_\theta}{\partial \theta} + \frac{\partial q_z}{\partial z} = 0.
\label{eq:journal_mass_balance}
\end{equation}
Подстановка выражений~\eqref{eq:journal_q_theta} и~\eqref{eq:journal_q_z} в уравнение~\eqref{eq:journal_mass_balance} приводит к уравнению Рейнольдса~\eqref{eq:journal_reynolds}.


\subsubsection*{Целевые функционалы}

Результатом решения уравнения Рейнольдса является поле давления $p(\theta, z)$. На его основе вычисляются следующие интегральные характеристики подшипника.

\textit{(а) Несущая способность.} В радиальном подшипнике результирующая сила давления представляет собой вектор в плоскости поперечного сечения. Давление действует на вал по нормали к его поверхности, направленной к центру вала. Поскольку угол~$\theta$ отсчитывается от точки максимального зазора ($\theta = 0$), направление линии центров (вектор эксцентриситета от~$O_b$ к~$O_j$) соответствует $\theta = \pi$. Введём ортонормированный базис $(\mathbf{e}_r, \mathbf{e}_t)$: вектор $\mathbf{e}_r$ направлен вдоль линии центров к минимальному зазору, а $\mathbf{e}_t$ получен поворотом $\mathbf{e}_r$ на $+90^\circ$ по направлению вращения. Проецируя элементарную силу $p\,R\,d\theta\,dz$ на этот базис и учитывая, что наружная нормаль к валу при угле~$\theta$ составляет с направлением~$\mathbf{e}_r$ угол $(\theta - \pi)$, а $\cos(\theta - \pi) = -\cos\theta$ и $\sin(\theta - \pi) = -\sin\theta$, получаем компоненты:
\begin{equation}
W_r = -\int_{-L/2}^{L/2}\int_0^{2\pi} p(\theta, z)\,\cos\theta\,R\,d\theta\,dz,
\label{eq:journal_Wr}
\end{equation}
\begin{equation}
W_t = -\int_{-L/2}^{L/2}\int_0^{2\pi} p(\theta, z)\,\sin\theta\,R\,d\theta\,dz,
\label{eq:journal_Wt}
\end{equation}
\begin{equation}
W = \sqrt{W_r^2 + W_t^2}.
\label{eq:journal_W}
\end{equation}
Знак <<минус>> в~\eqref{eq:journal_Wr}--\eqref{eq:journal_Wt} обусловлен тем, что $\cos\theta$ и $\sin\theta$ задают направление наружной нормали к валу, тогда как сила давления на вал направлена внутрь. Компонента $W_r$ действует вдоль линии центров (в направлении от центра втулки к центру вала), $W_t$ -- перпендикулярно ей.

Угол нагрузки (attitude angle)~\eqref{eq:journal_attitude} определяет угол между линией центров и направлением результирующей силы давления. В статике внешняя сила $\mathbf{W}_{\mathrm{ext}}$, приложенная к валу, уравновешивается результирующей силой давления: $\mathbf{W} + \mathbf{W}_{\mathrm{ext}} = 0$, т.\,е. $\mathbf{W}$ равна по модулю и противоположна по направлению $\mathbf{W}_{\mathrm{ext}}$. Угол~$\varphi$ можно задавать как относительно $\mathbf{W}$, так и относительно $-\mathbf{W}_{\mathrm{ext}}$; в настоящей работе он определяется через компоненты $\mathbf{W}$ по формуле~\eqref{eq:journal_attitude}. Физический смысл угла $\varphi$ состоит в следующем: вращение вала <<увлекает>> зону высокого давления в направлении вращения, поэтому линия действия результирующей силы давления не совпадает с линией центров, а повёрнута на угол $\varphi$ в направлении вращения. При малых эксцентриситетах ($\varepsilon \to 0$) угол $\varphi$ стремится к $\pi/2$ (сила давления почти перпендикулярна линии центров); при больших эксцентриситетах ($\varepsilon \to 1$) $\varphi \to 0$ (сила давления практически совпадает с линией центров).

\textit{(б) Число Зоммерфельда.} Основным безразмерным параметром проектирования радиальных подшипников является число Зоммерфельда:
\begin{equation}
S = \frac{\mu\,N_s\,L\,D}{W}\left(\frac{R}{c}\right)^2,
\label{eq:journal_sommerfeld}
\end{equation}
где $N_s = \omega/(2\pi)$ -- частота вращения (об/с), $D = 2R$ -- диаметр подшипника. Число $S$ связывает режим работы ($\mu$, $N_s$, $W$) с геометрией ($R$, $c$, $L$) и является аналогом безразмерной несущей способности. Связь между $S$ и $\varepsilon$ определяется решением уравнения Рейнольдса: при заданном $S$ определяется эксцентриситет $\varepsilon$ (и наоборот). Малые значения $S$ соответствуют тяжёлой нагрузке ($\varepsilon \to 1$, малый минимальный зазор), большие значения~-- лёгкой нагрузке ($\varepsilon \to 0$, вал расположен почти концентрично).

\textit{(в) Профиль окружной скорости и сдвиговые напряжения.} Профиль скорости в окружном направлении записывается по аналогии с~\eqref{eq:thrust_velocity_theta}:
\begin{equation}
u_\theta(y) = \frac{y}{h}\,U + \frac{1}{2\mu}\,\frac{1}{R}\,\frac{\partial p}{\partial \theta}\,y(y - h).
\label{eq:journal_velocity_theta}
\end{equation}
Первый член описывает линейный (куэттовский) профиль, обусловленный движением вала; второй -- параболическую (пуазейлевскую) добавку.

Сдвиговое напряжение на поверхности вала ($y = h$) определяется дифференцированием~\eqref{eq:journal_velocity_theta}:
\begin{equation}
\tau_w(\theta, z) = \frac{\mu\,U}{h} + \frac{h}{2R}\,\frac{\partial p}{\partial \theta}.
\label{eq:journal_tau_w}
\end{equation}

\textit{(г) Момент трения.} Суммарный момент сил трения относительно оси вала:
\begin{equation}
M = R^2 \int_{-L/2}^{L/2}\int_0^{2\pi} \tau_w(\theta, z)\,d\theta\,dz.
\label{eq:journal_M}
\end{equation}
Множитель $R^2$ складывается из двух сомножителей: $R$ от элемента площади $dA = R\,d\theta\,dz$ и ещё один $R$ -- плечо касательной силы относительно оси вала.

\textit{(д) Сила трения и коэффициент трения.} Суммарная тангенциальная сила трения на поверхности вала:
\begin{equation}
F_t = \int_{-L/2}^{L/2}\int_0^{2\pi} \tau_w(\theta, z)\,R\,d\theta\,dz.
\label{eq:journal_F_t}
\end{equation}
Коэффициент трения определяется как отношение силы трения к несущей способности:
\begin{equation}
f = \frac{F_t}{W}.
\label{eq:journal_f}
\end{equation}
В инженерных расчётах часто используют приближённое соотношение $F_t \approx M/R$, что приводит к оценке $f \approx M/(R\,W)$. Зависимость $f$ от числа Зоммерфельда $S$ (или от $\varepsilon$) носит название кривой Штрибека и является классическим результатом теории гидродинамической смазки; её подробный анализ приводится в главах~5--7.

\textit{(е) Мощность потерь на трение:}
\begin{equation}
P_{\mathrm{loss}} = \omega\,M.
\label{eq:journal_P_loss}
\end{equation}

\textit{(ж) Утечка через торцы} $Q_{\mathrm{side}}$ определяется формулами~\eqref{eq:journal_Q_side}--\eqref{eq:journal_Q_side_sym}.

\textit{(з) Минимальная толщина смазочного слоя:}
\begin{equation}
h_{\min} = c\,(1 - \varepsilon).
\label{eq:journal_h_min}
\end{equation}
Величина $h_{\min}$ определяет запас по толщине зазора; при проектировании $h_{\min}$ должна превышать суммарную шероховатость рабочих поверхностей с определённым коэффициентом запаса.

В практической постановке задачи внешняя радиальная нагрузка $\mathbf{W}_{\mathrm{ext}}$ и параметры режима ($\omega$, $\mu$, $c$, $R$, $L$) считаются заданными, тогда как положение вала $(\varepsilon, \varphi)$ является неизвестным. Равновесное положение определяется из условия статического баланса сил: $\mathbf{W}(\varepsilon, \varphi) = -\mathbf{W}_{\mathrm{ext}}$, т.\,е. из системы двух нелинейных уравнений относительно $\varepsilon$ и $\varphi$. Численное решение этой системы методом итерационного поиска описывается в главе~3.

Основные обозначения, введённые в настоящем подразделе, сведены в таблицу~\ref{tab:journal_notations}.

\begin{table}[!ht]
\centering
\caption{Основные обозначения для радиального подшипника}
\label{tab:journal_notations}
\begin{tabular}{|l|l|l|}
\hline
\textbf{Обозначение} & \textbf{Величина} & \textbf{Размерность} \\
\hline
$R$ & Радиус вала & м \\
$R_b$ & Радиус внутренней поверхности втулки & м \\
$c = R_b - R$ & Радиальный зазор & м \\
$e$ & Эксцентриситет & м \\
$\varepsilon$ & Относительный эксцентриситет $e/c$ & -- \\
$\varphi$ & Угол нагрузки (attitude angle) & рад \\
$\psi$ & Относительный зазор $c/R$ & -- \\
$L$, $D$ & Длина и диаметр подшипника & м \\
$\omega$ & Угловая скорость вала & рад/с \\
$\mu$ & Динамическая вязкость & Па$\cdot$с \\
$h(\theta)$ & Толщина смазочного слоя & м \\
$p(\theta, z)$ & Давление & Па \\
$W$ & Несущая способность (нагрузка) & Н \\
$S$ & Число Зоммерфельда & -- \\
$M$ & Момент трения & Н$\cdot$м \\
$Q_{\mathrm{side}}$ & Утечка через торцы & м$^3$/с \\
$f$ & Коэффициент трения & -- \\
\hline
\end{tabular}
\end{table}

Сформулированная задача сводится к решению эллиптического уравнения Рейнольдса~\eqref{eq:journal_reynolds_num} в прямоугольной области~\eqref{eq:journal_domain} с периодическими условиями по окружной координате~$\theta$ и условиями Дирихле по осевой координате~$z$. Принципиальной особенностью радиального подшипника, отличающей его от упорного (подраздел~2.1.2), является неизбежное возникновение кавитации в зоне расходящегося зазора: при любом ненулевом эксцентриситете $\varepsilon > 0$ правая часть уравнения Рейнольдса меняет знак, и в отсутствие ограничений на давление решение содержит области с $p < 0$.

Численная дискретизация уравнения~\eqref{eq:journal_reynolds_num} методом конечных разностей и алгоритм учёта кавитации на основе модели JFO рассматриваются в главе~3. При наличии микроструктур на рабочей поверхности функция зазора~\eqref{eq:journal_h_textured} приобретает быстрые локальные вариации, параметризация которых описывается в разделе~2.2.


\subsection{Возвратно-поступательная постановка задачи}

% Пара плунжер-цилиндр
%
% СОДЕРЖАНИЕ: аналогично 2.1.2

Рассмотрим пару плунжер-цилиндр...

% [РИСУНОК 2.4: Схема пары плунжер-цилиндр]
\begin{figure}[!ht]
\centering
\fbox{\parbox{0.8\textwidth}{\centering
\vspace{3cm}
ПЛЕЙСХОЛДЕР ДЛЯ РИСУНКА 2.4\\
Схема пары плунжер-цилиндр\\
Координаты, направление движения, зазор
\vspace{3cm}
}}
\caption{Схема пары плунжер-цилиндр}
\label{fig:plunger_cylinder}
\end{figure}


\subsection{Допущения и область применимости модели}

% СОДЕРЖАНИЕ:
% - Ньютоновская жидкость (μ = const)
% - Тонкий слой (h << L)
% - Ламинарность (Re << 1)
% - Изотермичность / неизотермичность
% - Несжимаемость (ρ = const)
% - Отсутствие инерции
% - Где модель ломается

Допущения, принятые при выводе уравнения Рейнольдса, перечислены в подразделе~2.1.1. Здесь обсудим границы применимости полученной модели и условия, при которых отдельные допущения могут нарушаться.

В рамках смазочной аппроксимации ключевым является малый параметр $\epsilon = H/L$. При росте числа Рейнольдса или при больших уклонах поверхности точность модели снижается. Ниже перечислены основные факторы, требующие расширения модели.

% СОДЕРЖАНИЕ 2.1.5 (границы применимости):
% - При каких значениях Re·ε модель теряет точность (инерция)
% - Турбулентность: при каких Re ламинарность нарушается
% - Нагрев: когда нужна μ(T) и уравнение энергии
% - Неньютоновское поведение: полимерные/загущённые масла
% - Большие уклоны поверхности: |dh/dx| ~ 1
% - Упругое деформирование поверхностей (EHL)
% - Что из перечисленного снимается/расширяется далее в монографии

% TODO: Расписать границы применимости по пунктам выше


\subsection{Граничные условия и условия замыкания}

% СОДЕРЖАНИЕ:
% - Что на входе/выходе
% - Боковые границы
% - Отрицательное давление
% - Реализация кавитации (мост к разделу 2.3)

Для замыкания задачи необходимо задать граничные условия...

% TODO: Описать ГУ для каждой постановки


\subsection{Целевые функционалы и выходные величины}

% СОДЕРЖАНИЕ:
% - Нагрузка W
% - Расход Q
% - Потери/трение F
% - Коэффициент трения f
% - Минимальная толщина пленки h_min
% - Кавитационная зона

Основными выходными величинами являются...

% [ТАБЛИЦА 2.1: Безразмерные группы и масштабы]
\begin{table}[h]
\centering
\caption{Безразмерные группы и характерные масштабы}
\label{tab:dimensionless_groups}
\begin{tabular}{|l|l|l|}
\hline
\textbf{Величина} & \textbf{Размерная} & \textbf{Безразмерная} \\
\hline
Давление & $p$ & $\bar{p} = p h_0^2 / (\mu U L)$ \\
Координата & $x$ & $\bar{x} = x / L$ \\
Толщина & $h$ & $\bar{h} = h / h_0$ \\
% TODO: Добавить остальные & & \\
\hline
\end{tabular}
\end{table}


% =============================================================
\section{Модель с детерминированными микроструктурами}

\subsection{Параметризация микрорельефа: $h(x,z)$ и параметры}

% СОДЕРЖАНИЕ:
% - Как задается функция h(x,z)
% - Параметры: глубина h_d, радиусы a и b, шаг p, доля площади φ

Для учёта детерминированных микроструктур рабочую поверхность представляем в виде гладкой базовой поверхности с наложенными регулярными элементами. Суммарная толщина смазочного слоя записывается как суперпозиция:
\begin{equation}
h(x, z) = h_{\mathrm{base}}(x, z) + \Delta h(x, z),
\label{eq:h_superposition}
\end{equation}
где $h_{\mathrm{base}}$ -- толщина зазора для гладкой поверхности (клиновой профиль, определённый в подразделах~2.1.2--2.1.4), а $\Delta h$ -- локальная добавка, описывающая геометрию микроструктуры ($\Delta h \leq 0$ для углублений, $\Delta h \geq 0$ для выступов). Принцип суперпозиции проиллюстрирован на рисунке~\ref{fig:h_superposition}.

% [РИСУНОК: Суперпозиция зазора h = h_base + Δh]
% Должен показывать два вида:
% (1) Продольный разрез: базовый клин h_base + ямка Δh → суммарный h
% (2) Вид сверху: пятно одного элемента микроструктуры на фоне сектора
\begin{figure}[!ht]
\centering
\fbox{\parbox{0.8\textwidth}{\centering
\vspace{3.5cm}
ПЛЕЙСХОЛДЕР ДЛЯ РИСУНКА\\
Суперпозиция зазора $h = h_{\mathrm{base}} + \Delta h$:\\
(a) продольный разрез -- базовый клин + локальная ямка;\\
(b) вид сверху -- пятно элемента микроструктуры на секторе
\vspace{3.5cm}
}}
\caption{Суперпозиция толщины зазора: базовый профиль и локальная микроструктура}
\label{fig:h_superposition}
\end{figure}

% [РИСУНОК 2.5: Типовая микроструктура с параметрами]
\begin{figure}[!ht]
\centering
\fbox{\parbox{0.8\textwidth}{\centering
\vspace{3cm}
ПЛЕЙСХОЛДЕР ДЛЯ РИСУНКА 2.5\\
Типовая микроструктура (лунка/канавка)\\
Параметры: $a$, $b$, $h_d$, профиль в разрезе
\vspace{3cm}
}}
\caption{Типовая микроструктура с основными параметрами}
\label{fig:microtexture_params}
\end{figure}

% [ТАБЛИЦА 2.2: Параметры микроструктуры]
\begin{table}[h]
\centering
\caption{Параметры детерминированной микроструктуры}
\label{tab:texture_params}
\begin{tabular}{|l|l|l|}
\hline
\textbf{Параметр} & \textbf{Обозначение} & \textbf{Размерность} \\
\hline
Глубина элемента & $h_d$ & мкм \\
Радиус (полуось) & $a, b$ & мкм \\
Шаг раскладки & $p$ & мкм \\
Доля площади & $\phi$ & -- \\
% TODO: Добавить остальные & & \\
\hline
\end{tabular}
\end{table}


\subsection{Типы геометрии микроструктур}

% СОДЕРЖАНИЕ:
% - Лунки (круглые/эллиптические/эллипсоидные)
% - Канавки/борозды (продольные/поперечные)
% - Комбинированные элементы

Рассмотрим основные типы геометрии микроструктур...

% [РИСУНОК 2.6: Геометрия различных микроструктур]
\begin{figure}[!ht]
\centering
\fbox{\parbox{0.8\textwidth}{\centering
\vspace{3cm}
ПЛЕЙСХОЛДЕР ДЛЯ РИСУНКА 2.6\\
Типы микроструктур:\\
(a) круглая лунка, (b) эллиптическая лунка,\\
(c) продольная канавка, (d) поперечная канавка
\vspace{3cm}
}}
\caption{Основные типы геометрии микроструктур}
\label{fig:texture_geometries}
\end{figure}


\subsection{Типы раскладки микроструктур}

% СОДЕРЖАНИЕ:
% - Регулярная решетка (шахматная/прямоугольная/гексагональная)
% - Кольцевая/секторальная
% - Филлотаксис

Рассмотрим варианты пространственного расположения микроструктур...

% [РИСУНОК 2.7: Шахматная раскладка]
\begin{figure}[!ht]
\centering
\fbox{\parbox{0.6\textwidth}{\centering
\vspace{2cm}
ПЛЕЙСХОЛДЕР ДЛЯ РИСУНКА 2.7\\
Шахматная раскладка микролунок\\
(вид сверху, координаты центров)
\vspace{2cm}
}}
\caption{Шахматная раскладка микроструктур}
\label{fig:layout_staggered}
\end{figure}

% [РИСУНОК 2.8: Кольцевая раскладка]
\begin{figure}[!ht]
\centering
\fbox{\parbox{0.6\textwidth}{\centering
\vspace{2cm}
ПЛЕЙСХОЛДЕР ДЛЯ РИСУНКА 2.8\\
Кольцевая раскладка (для радиального подшипника)
\vspace{2cm}
}}
\caption{Кольцевая раскладка микроструктур}
\label{fig:layout_annular}
\end{figure}

% [РИСУНОК 2.9: Филлотаксис]
\begin{figure}[!ht]
\centering
\fbox{\parbox{0.6\textwidth}{\centering
\vspace{2cm}
ПЛЕЙСХОЛДЕР ДЛЯ РИСУНКА 2.9\\
Раскладка по принципу филлотаксиса
\vspace{2cm}
}}
\caption{Раскладка микроструктур по принципу филлотаксиса}
\label{fig:layout_phyllotaxis}
\end{figure}


\subsection{Безразмеризация и определяющие параметры}

% СОДЕРЖАНИЕ:
% - Переход к безразмерным переменным
% - Набор определяющих параметров
% - Упрощенные формы для конкретных геометрий

Переход к безразмерным переменным...

% TODO: Написать про безразмеризацию


% =============================================================
\section{Кавитация и уравнения JFO (массо-сохраняющая постановка)}

\subsection{Физика разрыва пленки и ограничения модели с обнулением давления}

% СОДЕРЖАНИЕ:
% - Что происходит при отрицательном давлении
% - Почему простое p=0 неверно (нарушение массового баланса)
% - Физическая картина кавитации

При решении уравнения Рейнольдса в ряде случаев получаются 
отрицательные значения давления...

% [РИСУНОК 2.10: Схема кавитации]
\begin{figure}[!ht]
\centering
\fbox{\parbox{0.8\textwidth}{\centering
\vspace{3cm}
ПЛЕЙСХОЛДЕР ДЛЯ РИСУНКА 2.10\\
Схема кавитации в смазочном слое:\\
зона полного заполнения (p > 0),\\
зона кавитации (p = 0, частичное заполнение)
\vspace{3cm}
}}
\caption{Схематическое представление кавитации в смазочном слое}
\label{fig:cavitation_scheme}
\end{figure}


\subsection{Уравнения Jakobsson-Floberg-Olsson}

% СОДЕРЖАНИЕ:
% - Полная формулировка JFO
% - Условия переключения
% - Форма, пригодная для численной реализации

Массо-сохраняющая модель кавитации, предложенная Jakobsson, Floberg и Olsson...

% TODO: Написать уравнения JFO

Принципиальное отличие модели JFO от простого обнуления давления проиллюстрировано на рисунке~\ref{fig:jfo_1d_profile}.

% [РИСУНОК: 1D-профиль давления и степени заполнения в модели JFO]
% Должен показывать:
% - Ось абсцисс: координата вдоль направления течения (θ или x)
% - Верхний график: p(θ) — давление; в зоне кавитации p = 0 (горизонтальная линия)
% - Нижний график (или наложенный): α(θ) или θ_f(θ) — степень заполнения;
%   α = 1 в зоне полного слоя, α < 1 в зоне кавитации
% - Вертикальные штриховые линии: граница rupture (разрыва) и reformation (восстановления)
% - Для сравнения: пунктиром показать "обнулённое" p < 0 (модель Гюмбеля)
\begin{figure}[!ht]
\centering
\fbox{\parbox{0.8\textwidth}{\centering
\vspace{3.5cm}
ПЛЕЙСХОЛДЕР ДЛЯ РИСУНКА\\
1D-профиль вдоль направления течения:\\
давление $p(\theta)$ и степень заполнения $\alpha(\theta)$;\\
границы разрыва (rupture) и восстановления (reformation);\\
пунктир -- решение с обнулением $p \geq 0$ (Гюмбель)
\vspace{3.5cm}
}}
\caption{Сравнение моделей кавитации: обнуление давления (Гюмбель) и массо-сохраняющая постановка (JFO)}
\label{fig:jfo_1d_profile}
\end{figure}


\subsection{Связь с численной реализацией}

% СОДЕРЖАНИЕ:
% - Что именно будет реализовано в главе 3
% - Как JFO встраивается в численную схему
% - Алгоритм переключения между зонами

В главе 3 будет представлена численная реализация описанной модели...

% [РИСУНОК 2.11: Блок-схема модели]
\begin{figure}[!ht]
\centering
\fbox{\parbox{0.8\textwidth}{\centering
\vspace{3cm}
ПЛЕЙСХОЛДЕР ДЛЯ РИСУНКА 2.11\\
Блок-схема:\\
Входные параметры → Уравнения Рейнольдса + JFO →\\
→ Численное решение → Выходные величины
\vspace{3cm}
}}
\caption{Блок-схема математической модели}
\label{fig:model_flowchart}
\end{figure}


% =============================================================
\section*{Выводы по главе 2}
\addcontentsline{toc}{section}{Выводы по главе 2}

% СОДЕРЖАНИЕ (0.5-1.5 страницы):
% - 5-10 пунктов о принятой модели
% - Какие допущения
% - Какие безразмерные формы получены
% - Какие граничные условия/замыкание (JFO)
% - Какие параметры микроструктуры будут варьироваться

\begin{enumerate}
\item В главе представлена математическая постановка задачи гидродинамической 
смазки с учетом детерминированных микроструктур рабочей поверхности.

\item Получено уравнение Рейнольдса для описания распределения давления 
в тонком смазочном слое...

% TODO: Добавить остальные выводы (всего 5-10 пунктов)

\end{enumerate}