\chapter{Постановка и численное моделирование гидродинамических задач смазочного слоя с детерминированными микроструктурами рабочей поверхности}

% =============================================================
% ГЛАВА 2
% Структура с плейсхолдерами для рисунков и таблиц
% =============================================================

\section{Классическая модель для гладкой поверхности}

\subsection{Ключевые допущения и вывод уравнения Рейнольдса}

Рассмотрим движение вязкой несжимаемой ньютоновской жидкости в тонком зазоре между двумя твёрдыми поверхностями. Введём декартову систему координат $(x, y, z)$, где плоскость $xz$ совпадает с нижней поверхностью, а ось $y$ направлена перпендикулярно поверхностям (рисунок~\ref{fig:lubrication_scheme}). Толщина смазочного слоя обозначается как $h(x, z, t)$, где $t$ -- время.

% [РИСУНОК 2.1: Схема смазочного слоя]
% Должен показывать:
% - Две поверхности (верхняя и нижняя)
% - Координатные оси x, y, z
% - Толщина слоя h(x,z,t)
% - Скорости поверхностей U₁, U₂
% - Профиль скорости u(y) внутри слоя
% - Давление p(x,z,t)
\begin{figure}[!ht]
\centering
\fbox{\parbox{0.8\textwidth}{\centering
\vspace{3cm}
ПЛЕЙСХОЛДЕР ДЛЯ РИСУНКА 2.1\\
Схема смазочного слоя между двумя поверхностями\\
(найти рисунок "Reynolds lubrication schematic")
\vspace{3cm}
}}
\caption{Схема смазочного слоя между двумя поверхностями}
\label{fig:lubrication_scheme}
\end{figure}

\subsubsection*{Основные допущения}

Вывод уравнения Рейнольдса основан на следующих физических допущениях:

\begin{enumerate}
\item Жидкость ньютоновская: касательные напряжения пропорциональны скоростям деформации, в частности $\tau_{xy} = \mu\,\partial u/\partial y$, $\tau_{yz} = \mu\,\partial w/\partial y$.

\item Жидкость несжимаемая: $\rho = \mathrm{const}$.

\item Тонкий слой: $H \ll L$, где $H$ -- характерный (порядковый) масштаб толщины смазочного слоя $h(x,z,t)$; малые уклоны поверхностей $|\partial h/\partial x| \ll 1$, $|\partial h/\partial z| \ll 1$.

\item Течение ламинарное, инерционные эффекты пренебрежимо малы: $\mathrm{Re}\cdot\varepsilon \ll 1$, где $\varepsilon = H/L$.

\item Давление постоянно по толщине слоя: $\partial p/\partial y \approx 0$ (следствие смазочной аппроксимации).

\item Динамическая вязкость постоянна: $\mu = \mathrm{const}$ (температурная зависимость рассматривается отдельно при необходимости).

\item На границах жидкости и твёрдых поверхностей выполняется условие прилипания (no-slip).

\item Объёмные силы (гравитация и др.) малы по сравнению с градиентом давления и не учитываются.
\end{enumerate}

\subsubsection*{Исходные уравнения}

Движение жидкости описывается уравнениями Навье--Стокса для несжимаемой жидкости:

\begin{equation}
\rho\left(\frac{\partial \mathbf{v}}{\partial t} + (\mathbf{v} \cdot \nabla)\mathbf{v}\right) = -\nabla p + \mu \nabla^2 \mathbf{v},
\label{eq:navier_stokes}
\end{equation}

и уравнением неразрывности:

\begin{equation}
\nabla \cdot \mathbf{v} = 0,
\label{eq:continuity}
\end{equation}

где $\mathbf{v} = (u, v, w)$ -- вектор скорости;
\hphantom{где }$\rho$ -- плотность жидкости;
\hphantom{где }$p$ -- давление;
\hphantom{где }$\mu$ -- динамическая вязкость.

В компонентной форме уравнения движения приведены в приложении~А. Уравнение неразрывности~\eqref{eq:continuity} в компонентной записи имеет вид:

\begin{equation}
\frac{\partial u}{\partial x} + \frac{\partial v}{\partial y} + \frac{\partial w}{\partial z} = 0.
\label{eq:continuity_comp}
\end{equation}

Далее, на основе анализа порядков величин, будет получена смазочная аппроксимация и выведено уравнение Рейнольдса.

\subsubsection*{Масштабирование и оценка порядков величин}

Для анализа уравнений введём характерные масштабы задачи:

$L$ -- характерный размер в направлениях $x$ и $z$ (длина подшипника);

$H$ -- характерная толщина смазочного слоя;

$U$ -- характерная скорость движения поверхностей;

$P$ -- характерное давление.

Безразмерные переменные вводятся следующим образом:

\begin{equation}
\bar{x} = \frac{x}{L}, \quad \bar{y} = \frac{y}{H}, \quad \bar{z} = \frac{z}{L}, \quad \bar{u} = \frac{u}{U}, \quad \bar{v} = \frac{v}{V}, \quad \bar{w} = \frac{w}{U}, \quad \bar{p} = \frac{p}{P},
\end{equation}

где $V$ -- характерная скорость в направлении $y$, которую нужно определить из уравнения неразрывности.

Из уравнения неразрывности следует оценка:

\begin{equation}
\frac{U}{L} \sim \frac{V}{H} \quad \Rightarrow \quad V \sim U\frac{H}{L}.
\end{equation}

Ключевым параметром в теории смазки является отношение:

\begin{equation}
\varepsilon = \frac{H}{L} \ll 1,
\label{eq:epsilon}
\end{equation}

которое характеризует малость толщины слоя по сравнению с его протяжённостью.

Подставляя безразмерные переменные в компонентные уравнения движения, получаем оценки для различных членов. Рассмотрим $y$-компоненту уравнения движения:

\begin{equation}
\rho\frac{U^2}{L}\varepsilon \sim \frac{P}{H} + \mu\frac{U}{H^2}.
\end{equation}

Определяем характерный масштаб давления из баланса вязких сил и градиента давления:

\begin{equation}
P \sim \mu U\frac{L}{H^2}.
\label{eq:pressure_scale}
\end{equation}

Характерное давление определяется балансом вязкого сопротивления потоку в тонком зазоре и продольного градиента давления.

Вводим число Рейнольдса для смазочного слоя:

\begin{equation}
\text{Re} = \frac{\rho U H}{\mu}.
\end{equation}

При условии $\text{Re} \cdot \varepsilon \ll 1$ (что выполняется в большинстве практических случаев) инерционные члены в уравнениях движения пренебрежимо малы по сравнению с вязкими и градиентом давления. Малость этого параметра означает, что инерция жидкости подавлена тонкостью смазочного слоя ($\varepsilon = H/L \ll 1$), и течение полностью определяется балансом вязких сил и градиента давления.

Анализируя порядки производных с учётом $\varepsilon \ll 1$, получаем:

\begin{equation}
\frac{\partial^2}{\partial y^2} \gg \frac{\partial^2}{\partial x^2}, \quad \frac{\partial^2}{\partial y^2} \gg \frac{\partial^2}{\partial z^2}.
\end{equation}

Таким образом, в $x$- и $z$-компонентах уравнения движения доминируют члены:

\begin{equation}
\frac{\partial p}{\partial x} \sim \mu\frac{\partial^2 u}{\partial y^2}, \quad \frac{\partial p}{\partial z} \sim \mu\frac{\partial^2 w}{\partial y^2}.
\end{equation}

Из $y$-компоненты уравнения движения следует, что $\partial p/\partial y \sim \varepsilon^2 \ll 1$, откуда:

\begin{equation}
\frac{\partial p}{\partial y} \approx 0.
\label{eq:dp_dy_zero}
\end{equation}

Это означает, что давление в смазочном слое не зависит от координаты $y$: $p = p(x, z, t)$.

\subsubsection*{Упрощённые уравнения движения}

С учётом сделанных допущений уравнения движения принимают вид:

\begin{align}
\frac{\partial p}{\partial x} &= \mu\frac{\partial^2 u}{\partial y^2}, \label{eq:simplified_x} \\
\frac{\partial p}{\partial z} &= \mu\frac{\partial^2 w}{\partial y^2}. \label{eq:simplified_z}
\end{align}

Уравнение неразрывности сохраняет форму~\eqref{eq:continuity_comp}.

\subsubsection*{Граничные условия}

На поверхностях выполняются условия прилипания (no-slip):

\begin{equation}
\begin{aligned}
&y = 0: \quad u = U_1(x, z, t), \quad w = W_1(x, z, t), \quad v = V_1(x, z, t), \\
&y = h: \quad u = U_2(x, z, t), \quad w = W_2(x, z, t), \quad v = V_2(x, z, t),
\end{aligned}
\label{eq:boundary_conditions}
\end{equation}

где $U_i$, $W_i$, $V_i$ -- компоненты скорости соответствующих поверхностей.

\subsubsection*{Профиль скорости}

Интегрируя уравнение~\eqref{eq:simplified_x} дважды по $y$ и используя граничные условия~\eqref{eq:boundary_conditions}, получаем профиль скорости в направлении $x$:

\begin{equation}
u(x, y, z, t) = U_1 + \frac{y}{h}(U_2 - U_1) + \frac{1}{2\mu}\frac{\partial p}{\partial x}y(y - h).
\label{eq:velocity_profile_u}
\end{equation}

Аналогично для направления $z$:

\begin{equation}
w(x, y, z, t) = W_1 + \frac{y}{h}(W_2 - W_1) + \frac{1}{2\mu}\frac{\partial p}{\partial z}y(y - h).
\label{eq:velocity_profile_w}
\end{equation}

Профиль скорости~\eqref{eq:velocity_profile_u} состоит из двух частей: линейной (куэттовский поток, обусловленный движением поверхностей) и параболической (пуазейлевский поток, обусловленный градиентом давления).

\subsubsection*{Объёмный расход}

Объёмный расход жидкости через единицу длины перпендикулярно направлению $x$ определяется интегрированием скорости по толщине слоя:

\begin{equation}
q_x = \int_0^h u \, dy.
\end{equation}

Подставляя выражение~\eqref{eq:velocity_profile_u}:

\begin{equation}
q_x = \int_0^h \left[U_1 + \frac{y}{h}(U_2 - U_1) + \frac{1}{2\mu}\frac{\partial p}{\partial x}y(y - h)\right] dy.
\end{equation}

Вычисляем интегралы:

\begin{equation}
\begin{aligned}
\int_0^h U_1 \, dy &= U_1 h, \\
\int_0^h \frac{y}{h}(U_2 - U_1) \, dy &= \frac{U_2 - U_1}{h} \cdot \frac{h^2}{2} = \frac{h}{2}(U_2 - U_1), \\
\int_0^h \frac{1}{2\mu}\frac{\partial p}{\partial x}y(y - h) \, dy &= \frac{1}{2\mu}\frac{\partial p}{\partial x}\left(\frac{h^3}{3} - \frac{h^3}{2}\right) = -\frac{h^3}{12\mu}\frac{\partial p}{\partial x}.
\end{aligned}
\end{equation}

Окончательно для расхода в направлении $x$:

\begin{equation}
q_x = \frac{h}{2}(U_1 + U_2) - \frac{h^3}{12\mu}\frac{\partial p}{\partial x}.
\label{eq:flow_rate_x}
\end{equation}

Аналогично для направления $z$:

\begin{equation}
q_z = \frac{h}{2}(W_1 + W_2) - \frac{h^3}{12\mu}\frac{\partial p}{\partial z}.
\label{eq:flow_rate_z}
\end{equation}

\subsubsection*{Вывод уравнения Рейнольдса}

Из уравнения неразрывности~\eqref{eq:continuity}, интегрируя по толщине слоя от $y = 0$ до $y = h$:

\begin{equation}
\int_0^h \frac{\partial u}{\partial x} \, dy + \int_0^h \frac{\partial v}{\partial y} \, dy + \int_0^h \frac{\partial w}{\partial z} \, dy = 0.
\end{equation}

Используя правило Лейбница для дифференцирования интеграла:

\begin{equation}
\frac{\partial}{\partial x}\int_0^h u \, dy = \int_0^h \frac{\partial u}{\partial x} \, dy + u|_{y=h}\frac{\partial h}{\partial x},
\end{equation}

и аналогично для $z$.

Для второго интеграла:

\begin{equation}
\int_0^h \frac{\partial v}{\partial y} \, dy = v|_{y=h} - v|_{y=0} = V_2 - V_1.
\end{equation}

Кинематическое условие на верхней поверхности $y = h(x,z,t)$:

\begin{equation}
V_2 = \frac{\partial h}{\partial t} + U_2\frac{\partial h}{\partial x} + W_2\frac{\partial h}{\partial z}.
\label{eq:kinematic_upper}
\end{equation}

Нижняя поверхность полагается плоской и неподвижной в направлении $y$, откуда $V_1 = 0$.

Подставляя в уравнение неразрывности и используя определения расходов~\eqref{eq:flow_rate_x} и~\eqref{eq:flow_rate_z}:

\begin{equation}
\frac{\partial q_x}{\partial x} + \frac{\partial q_z}{\partial z} + \frac{\partial h}{\partial t} = 0.
\label{eq:continuity_integrated}
\end{equation}

Уравнение~\eqref{eq:continuity_integrated} допускает наглядную интерпретацию в терминах контрольного объёма (рисунок~\ref{fig:control_volume}): изменение толщины слоя $\partial h/\partial t$ компенсируется потоками $q_x$ и $q_z$ через боковые грани элемента $dx \times dz$.

% [РИСУНОК: Контрольный объём смазочного слоя]
% Должен показывать:
% - Элемент dx × dz, толщина h
% - Потоки q_x и q_x + ∂q_x/∂x·dx через левую и правую грани
% - Потоки q_z и q_z + ∂q_z/∂z·dz через переднюю и заднюю грани
% - Стрелка ∂h/∂t сверху (выжимание)
% - Формула баланса: ∂q_x/∂x + ∂q_z/∂z + ∂h/∂t = 0
\begin{figure}[!ht]
\centering
\fbox{\parbox{0.8\textwidth}{\centering
\vspace{3cm}
ПЛЕЙСХОЛДЕР ДЛЯ РИСУНКА\\
Контрольный объём смазочного слоя:\\
элемент $dx \times dz$, толщина $h$,\\
потоки $q_x$, $q_z$ через боковые грани,\\
выжимание $\partial h/\partial t$
\vspace{3cm}
}}
\caption{Контрольный объём смазочного слоя и баланс расходов}
\label{fig:control_volume}
\end{figure}

Подставляя выражения для $q_x$ и $q_z$:

\begin{equation}
\frac{\partial}{\partial x}\left[\frac{h}{2}(U_1 + U_2) - \frac{h^3}{12\mu}\frac{\partial p}{\partial x}\right] + \frac{\partial}{\partial z}\left[\frac{h}{2}(W_1 + W_2) - \frac{h^3}{12\mu}\frac{\partial p}{\partial z}\right] + \frac{\partial h}{\partial t} = 0.
\end{equation}

Раскрывая производные:

\begin{equation}
\frac{\partial}{\partial x}\left(\frac{h^3}{12\mu}\frac{\partial p}{\partial x}\right) + \frac{\partial}{\partial z}\left(\frac{h^3}{12\mu}\frac{\partial p}{\partial z}\right) = \frac{\partial}{\partial x}\left[\frac{h}{2}(U_1 + U_2)\right] + \frac{\partial}{\partial z}\left[\frac{h}{2}(W_1 + W_2)\right] + \frac{\partial h}{\partial t}.
\label{eq:reynolds_full}
\end{equation}

Уравнение~\eqref{eq:reynolds_full} является \textbf{обобщённым уравнением Рейнольдса} для случая нестационарного течения с подвижными границами произвольной формы.

Для стационарного случая ($\partial h/\partial t = 0$) и при условии $W_1 = W_2 = 0$ (движение только в направлении $x$) уравнение упрощается:

\begin{equation}
\frac{\partial}{\partial x}\left(\frac{h^3}{12\mu}\frac{\partial p}{\partial x}\right) + \frac{\partial}{\partial z}\left(\frac{h^3}{12\mu}\frac{\partial p}{\partial z}\right) = \frac{U_1 + U_2}{2}\frac{\partial h}{\partial x}.
\label{eq:reynolds_steady}
\end{equation}

Уравнение~\eqref{eq:reynolds_steady} представляет собой \textbf{классическое уравнение Рейнольдса} для гидродинамической смазки.

В случае одной движущейся поверхности (например, $U_1 = U$, $U_2 = 0$):

\begin{equation}
\frac{\partial}{\partial x}\left(\frac{h^3}{\mu}\frac{\partial p}{\partial x}\right) + \frac{\partial}{\partial z}\left(\frac{h^3}{\mu}\frac{\partial p}{\partial z}\right) = 6U\frac{\partial h}{\partial x}.
\label{eq:reynolds_one_surface}
\end{equation}

Полученное уравнение Рейнольдса~\eqref{eq:reynolds_one_surface} является основным уравнением теории гидродинамической смазки и связывает распределение давления $p(x, z)$ в смазочном слое с геометрией зазора $h(x, z)$ и скоростью движения поверхности $U$.

Следует отметить, что при решении уравнения Рейнольдса в областях расходящегося зазора могут возникать зоны отрицательного давления, не имеющие физического смысла. Для корректного описания таких областей необходима массо-сохраняющая модель кавитации, которая рассматривается в разделе~2.3 на основе подхода Jakobsson--Floberg--Olsson (JFO).


\subsection{Упорная постановка задачи}

% Упорный подшипник (thrust bearing)
% Полная постановка задачи: геометрия, кинематика, уравнение Рейнольдса,
% граничные условия, расходы, целевые функционалы

Рассмотрим упорный (осевой) подшипник скольжения -- узел, в котором осевая нагрузка воспринимается тонким слоем вязкой жидкости, заключённым между двумя поверхностями: вращающимся диском (ротором) и неподвижной опорной поверхностью (колодкой, или сектором). Несущая способность подшипника обусловлена гидродинамическим давлением, которое генерируется в сходящемся зазоре при относительном движении поверхностей.

Упорные подшипники скольжения находят широкое применение в инженерной практике. В турбинах гидро- и теплоэнергетических установок они воспринимают осевое усилие, создаваемое рабочим телом на рабочем колесе. В центробежных насосах и компрессорах упорные подшипники фиксируют осевое положение ротора, предотвращая контакт вращающихся и неподвижных элементов. В тяжёлых прокатных станах и вертикальных гидрогенераторах упорные подшипники несут вес ротора и всего рабочего оборудования.

Цель настоящей постановки состоит в следующем: для заданной геометрии зазора $h(r,\theta)$ и режима движения поверхностей получить распределение давления $p(r,\theta)$ в смазочном слое, а затем вычислить несущую способность $W$, расход смазки $Q$, момент трения $M$ и другие эксплуатационные характеристики. Все допущения, принятые при выводе уравнения Рейнольдса в подразделе~2.1.1, сохраняют силу: жидкость ньютоновская и несжимаемая, течение ламинарное и стационарное, давление постоянно по толщине смазочного слоя. Далее рассматривается один сектор (колодка) подшипника; полный подшипник состоит из $N$ одинаковых секторов, расположенных равномерно по окружности.


\subsubsection*{Геометрия и система координат}

Для описания упорного подшипника естественно использовать цилиндрические координаты $(r, \theta, y)$:
\begin{itemize}
\item $r$ -- радиальная координата в плоскости подшипника;
\item $\theta$ -- окружная координата, отсчитываемая по направлению вращения;
\item $y$ -- координата по толщине смазочного слоя, направленная перпендикулярно рабочим поверхностям (от $y = 0$ на неподвижной колодке до $y = h$ на вращающемся диске).
\end{itemize}

Область расчёта для одного сектора определяется следующим образом:
\begin{equation}
r \in [R_{\mathrm{in}},\; R_{\mathrm{out}}], \qquad \theta \in [\theta_0,\; \theta_1],
\label{eq:thrust_domain}
\end{equation}
где $R_{\mathrm{in}}$ и $R_{\mathrm{out}}$ -- внутренний и наружный радиусы рабочей зоны подшипника, $\theta_0$ и $\theta_1$ -- угловые границы сектора. Угловая протяжённость сектора обозначается
\begin{equation}
\beta = \theta_1 - \theta_0.
\label{eq:thrust_beta}
\end{equation}
Граница $\theta_0$ соответствует входной кромке (leading edge), а $\theta_1$ -- выходной кромке (trailing edge) по направлению вращения диска. Толщина смазочного слоя в каждой точке расчётной области составляет $y \in [0,\; h(r,\theta)]$.

Для подшипника из $N$ одинаковых секторов, расположенных равномерно по окружности, угловая протяжённость одного сектора определяется как $\beta = 2\pi/N$ за вычетом углового зазора между соседними колодками. Расчётная область одного сектора схематически изображена на рисунке~\ref{fig:thrust_bearing}.

% [РИСУНОК: Схема упорного подшипника — вид сверху на сектор]
% Должен показывать:
% - Вид сверху на один сектор подшипника
% - Оси r, θ, направление вращения ω
% - Границы R_in, R_out, θ_0, θ_1
% - Угловую протяжённость β
\begin{figure}[!ht]
\centering
\fbox{\parbox{0.8\textwidth}{\centering
\vspace{3cm}
ПЛЕЙСХОЛДЕР ДЛЯ РИСУНКА\\
Схема упорного подшипника -- вид сверху на сектор\\
Оси $r$, $\theta$, направление вращения $\omega$,\\
границы $R_{\mathrm{in}}$, $R_{\mathrm{out}}$, $\theta_0$, $\theta_1$
\vspace{3cm}
}}
\caption{Схема упорного подшипника: вид сверху на один сектор}
\label{fig:thrust_bearing}
\end{figure}


\subsubsection*{Кинематика движения поверхностей}

Нижняя поверхность ($y = 0$) представляет собой неподвижную колодку, верхняя поверхность ($y = h$) -- вращающийся диск с постоянной угловой скоростью $\omega$. Тангенциальная (окружная) скорость скольжения вращающегося диска в точке с радиальной координатой $r$ составляет:
\begin{equation}
U_\theta(r) = \omega\,r.
\label{eq:thrust_U_theta}
\end{equation}
Радиальные скорости обеих поверхностей равны нулю: $U_{r,1} = U_{r,2} = 0$. В стационарном режиме нормальные скорости поверхностей также обращаются в нуль: $V_1 = V_2 = 0$, что исключает эффект выжимания ($\partial h/\partial t = 0$).

Динамическая вязкость смазки полагается постоянной: $\mu = \mathrm{const}$ (допущение~6 из подраздела~2.1.1). Влияние температурной зависимости вязкости $\mu(T)$ на характеристики подшипника рассматривается отдельно в главе~4.

Вращение диска создаёт тангенциальный увлекающий поток (куэттовскую составляющую течения): жидкость, увлекаемая подвижной поверхностью за счёт условия прилипания, движется в окружном направлении. Когда такой поток проходит через участок зазора с уменьшающейся толщиной, условие неразрывности (сохранения массы) приводит к росту давления в смазочном слое -- именно этот механизм, называемый клиновым эффектом, обеспечивает несущую способность подшипника.


\subsubsection*{Закон толщины смазочного слоя}

В общем случае толщина смазочного слоя является заданной функцией координат:
\begin{equation}
h = h(r, \theta),
\label{eq:thrust_h_general}
\end{equation}
определяющей профиль зазора между колодкой и вращающимся диском. В разделе~2.2 функция $h(r,\theta)$ будет расширена для учёта детерминированных микроструктур рабочей поверхности.

Для тестирования численных методов и верификации результатов широко используется базовая геометрия -- линейный клин, в котором толщина зазора линейно изменяется по окружной координате:
\begin{equation}
h(r,\theta) = h_{\mathrm{in}} + (h_{\mathrm{out}} - h_{\mathrm{in}})\,\frac{\theta - \theta_0}{\theta_1 - \theta_0},
\label{eq:thrust_h_wedge}
\end{equation}
где $h_{\mathrm{in}} = h(\theta_0)$ -- толщина смазочного слоя на входной кромке сектора, $h_{\mathrm{out}} = h(\theta_1)$ -- толщина на выходной кромке. Необходимым условием генерации давления является сходящийся характер зазора по направлению движения, т.\,е.
\begin{equation}
h_{\mathrm{in}} > h_{\mathrm{out}}.
\label{eq:thrust_convergence_condition}
\end{equation}
В частном случае $h_{\mathrm{in}} = h_{\mathrm{out}}$ (плоскопараллельный зазор) правая часть уравнения Рейнольдса обращается в нуль и давление не генерируется.

В формуле~\eqref{eq:thrust_h_wedge} величины $h_{\mathrm{in}}$ и $h_{\mathrm{out}}$ приняты постоянными по радиальной координате $r$, что соответствует плоскому клину. Минимальная толщина смазочного слоя обозначается
\begin{equation}
h_{\min} = \min_{(r,\theta)} h(r,\theta);
\label{eq:thrust_h_min}
\end{equation}
для линейного клина $h_{\min} = h_{\mathrm{out}}$.

Для характеристики формы клина вводятся безразмерные параметры конвергенции:
\begin{equation}
K = \frac{h_{\mathrm{in}}}{h_{\mathrm{out}}}, \qquad \delta = \frac{h_{\mathrm{in}} - h_{\mathrm{out}}}{h_{\mathrm{out}}} = K - 1.
\label{eq:thrust_convergence}
\end{equation}
При $K = 1$ ($\delta = 0$) зазор является плоскопараллельным и давление не генерируется. Оптимальное значение параметра $K$, обеспечивающее максимальную несущую способность, зависит от соотношения размеров подшипника и определяется численно.

Помимо линейного клина, на практике применяются более сложные профили зазора: ступенчатый (профиль Рэлея), параболический, составной (с плоским участком и клиновой частью). Все перечисленные варианты описываются тем же уравнением Рейнольдса при соответствующем задании функции $h(r,\theta)$. Линейный клин~\eqref{eq:thrust_h_wedge} используется в настоящей работе как базовый верификационный профиль; в реальной колодке профиль зазора $h(r,\theta)$ определяется наклоном опоры и формой рабочей поверхности, что может приводить к зависимости $h$ от радиальной координаты~$r$. Продольный разрез клинового зазора при фиксированном $r$ представлен на рисунке~\ref{fig:thrust_wedge_profile}.

% [РИСУНОК: Продольный разрез клинового зазора]
% Должен показывать:
% - Сечение при фиксированном r по окружной координате θ
% - Неподвижную колодку (снизу), вращающийся диск (сверху)
% - Сходящийся зазор от h_in до h_out
% - Направление U_θ
% - Качественный профиль скорости u_θ(y) в 1–2 точках
% - Стрелку давления p
\begin{figure}[!ht]
\centering
\fbox{\parbox{0.8\textwidth}{\centering
\vspace{3cm}
ПЛЕЙСХОЛДЕР ДЛЯ РИСУНКА\\
Продольный разрез клинового зазора упорного подшипника\\
(сечение при фиксированном $r$):\\
неподвижная колодка (снизу), вращающийся диск (сверху),\\
сходящийся зазор $h_{\mathrm{in}} \to h_{\mathrm{out}}$, направление $U_\theta$,\\
качественные профили скорости $u_\theta(y)$, давление $p$
\vspace{3cm}
}}
\caption{Продольный разрез клинового зазора упорного подшипника при фиксированном $r$}
\label{fig:thrust_wedge_profile}
\end{figure}


\subsubsection*{Уравнение Рейнольдса в цилиндрических координатах}

Обобщённое уравнение Рейнольдса~\eqref{eq:reynolds_full}, полученное в подразделе~2.1.1 в декартовых координатах, необходимо переписать в цилиндрических координатах $(r, \theta)$, естественных для задачи об упорном подшипнике. Переход осуществляется введением дуговой координаты $s = r\theta$, вдоль которой производная принимает вид $\partial/\partial s = (1/r)\,\partial/\partial\theta$. Оператор дивергенции и градиента в плоскости $(r, \theta)$ приобретает стандартные множители $1/r$ и $1/r^2$, присутствующие в левой части уравнения Рейнольдса.

В рассматриваемой задаче скорость скольжения направлена по окружной координате~$\theta$: нижняя поверхность неподвижна ($U_{\theta,1} = 0$), верхняя движется со скоростью $U_{\theta,2} = \omega r$; радиальные скорости обеих поверхностей равны нулю. Подставляя эти условия в правую часть обобщённого уравнения Рейнольдса и учитывая, что окружная производная от $h$ содержит метрический множитель $1/r$, получаем:
\begin{equation}
\frac{U_{\theta,1} + U_{\theta,2}}{2}\,\frac{1}{r}\,\frac{\partial h}{\partial \theta} = \frac{\omega r}{2}\,\frac{1}{r}\,\frac{\partial h}{\partial \theta} = \frac{\omega}{2}\,\frac{\partial h}{\partial \theta}.
\label{eq:thrust_rhs_derivation}
\end{equation}

Стационарное уравнение Рейнольдса для несжимаемой ньютоновской жидкости в цилиндрических координатах принимает вид:
\begin{equation}
\frac{1}{r}\frac{\partial}{\partial r}\!\left[r\,\frac{h^3}{12\mu}\,\frac{\partial p}{\partial r}\right] + \frac{1}{r^2}\frac{\partial}{\partial \theta}\!\left[\frac{h^3}{12\mu}\,\frac{\partial p}{\partial \theta}\right] = \frac{\omega}{2}\,\frac{\partial h}{\partial \theta}.
\label{eq:thrust_reynolds}
\end{equation}
Множители $1/r$ и $1/r^2$ перед производными в левой части обусловлены дивергентной формой эллиптического оператора $\nabla \cdot (h^3 \nabla p)$ в цилиндрических координатах. Оператор Лапласа является его частным случаем при $h = \mathrm{const}$.

Умножая обе части уравнения~\eqref{eq:thrust_reynolds} на $12\mu$, получаем эквивалентную форму, удобную для численной реализации:
\begin{equation}
\frac{1}{r}\frac{\partial}{\partial r}\!\left[r\,h^3\,\frac{\partial p}{\partial r}\right] + \frac{1}{r^2}\frac{\partial}{\partial \theta}\!\left[h^3\,\frac{\partial p}{\partial \theta}\right] = 6\mu\omega\,\frac{\partial h}{\partial \theta}.
\label{eq:thrust_reynolds_num}
\end{equation}
Уравнение~\eqref{eq:thrust_reynolds_num} является основным расчётным уравнением для упорного подшипника в настоящей работе.

Левая часть уравнений~\eqref{eq:thrust_reynolds}--\eqref{eq:thrust_reynolds_num} представляет собой эллиптический оператор, описывающий перераспределение давления через вязкий смазочный слой: давление <<диффундирует>> как в радиальном, так и в окружном направлениях, причём <<проводимость>> пропорциональна кубу толщины зазора~$h^3$. Правая часть -- источниковый член, обусловленный изменением толщины зазора в направлении движения (клиновой эффект). При $\partial h/\partial \theta = 0$ (плоскопараллельный зазор) правая часть обращается в нуль, и при однородных граничных условиях уравнение допускает лишь тривиальное решение $p = 0$ -- давление не генерируется.

При наличии детерминированных микроструктур на рабочей поверхности функция $h(r,\theta)$ приобретает быстрые локальные вариации. Формально уравнение~\eqref{eq:thrust_reynolds_num} остаётся тем же, однако функция зазора существенно усложняется, что описывается в разделе~2.2.


\subsubsection*{Граничные условия}

Для замыкания уравнения Рейнольдса~\eqref{eq:thrust_reynolds_num} необходимо задать граничные условия на давление по всему контуру расчётной области~\eqref{eq:thrust_domain}. Рассмотрим три практически важных варианта.

\textit{Базовый вариант (атмосферные кромки).} На всех четырёх границах сектора давление принимается равным атмосферному:
\begin{equation}
\begin{aligned}
p(r,\,\theta_0) &= 0, \qquad & p(r,\,\theta_1) &= 0, \\
p(R_{\mathrm{in}},\,\theta) &= 0, \qquad & p(R_{\mathrm{out}},\,\theta) &= 0.
\end{aligned}
\label{eq:thrust_bc_base}
\end{equation}
Здесь давление отсчитывается от атмосферного ($p = p_{\mathrm{abs}} - p_{\mathrm{atm}}$). Условие~\eqref{eq:thrust_bc_base} соответствует изолированному сектору, кромки которого свободно сообщаются с окружающей средой, а принудительная подача смазки отсутствует. Данный вариант является наиболее распространённым при расчёте несамоустанавливающихся колодок.

\textit{Вариант с давлением подачи.} Если система смазки обеспечивает принудительную подачу жидкости через входную кромку сектора, граничное условие на $\theta = \theta_0$ модифицируется:
\begin{equation}
p(r,\,\theta_0) = p_{\mathrm{sup}}(r),
\label{eq:thrust_bc_supply}
\end{equation}
где $p_{\mathrm{sup}}(r)$ -- заданное давление подачи смазки; в простейшем случае $p_{\mathrm{sup}} = \mathrm{const}$. Остальные граничные условия остаются такими же, как в базовом варианте~\eqref{eq:thrust_bc_base}. Величина $p_{\mathrm{sup}}$ определяется конструкцией системы смазки и режимом работы насоса подачи.

\textit{Кавитация.} При решении уравнения Рейнольдса в областях расходящегося зазора расчётное давление может оказаться отрицательным. Отрицательное давление физически не реализуется: при снижении давления ниже давления насыщенных паров смазки жидкость разрывается и образуется кавитационная зона с парогазовой смесью. В простейшей постановке применяют ограничение $p \geq 0$ (условие Гюмбеля), при котором отрицательные давления обнуляются после решения. Такой подход, однако, нарушает закон сохранения массы на границе кавитационной зоны. Корректная массо-сохраняющая постановка, основанная на модели Jakobsson--Floberg--Olsson (JFO), вводится в разделе~2.3. Граница между зоной полного смазочного слоя и кавитационной зоной заранее неизвестна и определяется в процессе решения как часть задачи.

Следует отметить, что условия Дирихле $p = 0$ на всех четырёх границах сектора представляют лишь один из возможных вариантов замыкания задачи. В зависимости от конструкции узла и схемы смазки могут применяться и другие постановки: периодичность по $\theta$ (при моделировании полного кольцевого сегмента без разрывов), условие непротекания через радиальные границы $q_r = 0$, что эквивалентно $\partial p/\partial r = 0$ (при наличии уплотнений или перемычек), а также смешанные условия при комбинированной подаче смазки. Конкретный набор граничных условий определяется конструкцией узла и схемой подвода смазки.

Перечисленные варианты граничных условий схематически показаны на рисунке~\ref{fig:thrust_bc_schemes}.

% [РИСУНОК: Варианты граничных условий на секторе]
% Должен показывать три подрисунка (a), (b), (c):
% (a) Базовый: p = 0 на всех 4 границах сектора
% (b) С подпиткой: p = p_sup на θ = θ_0, остальные p = 0
% (c) Альтернативный: ∂p/∂r = 0 на радиальных границах (уплотнения)
%     или периодичность по θ
% На каждом подрисунке — контур сектора с обозначением ГУ на каждой стороне
\begin{figure}[!ht]
\centering
\fbox{\parbox{0.8\textwidth}{\centering
\vspace{3.5cm}
ПЛЕЙСХОЛДЕР ДЛЯ РИСУНКА\\
Варианты граничных условий на секторе:\\
(a) $p = 0$ на всех границах;\\
(b) $p = p_{\mathrm{sup}}$ на $\theta = \theta_0$, остальные $p = 0$;\\
(c) $\partial p/\partial r = 0$ на радиальных границах (уплотнения)
\vspace{3.5cm}
}}
\caption{Варианты граничных условий для упорного подшипника}
\label{fig:thrust_bc_schemes}
\end{figure}


\subsubsection*{Объёмные расходы}

Объёмные расходы жидкости через единицу длины определяются путём интегрирования профилей скорости по толщине смазочного слоя, аналогично выражениям~\eqref{eq:flow_rate_x}--\eqref{eq:flow_rate_z}, полученным в подразделе~2.1.1 в декартовых координатах. Ниже приведены формулы, адаптированные к цилиндрической системе координат.

Радиальный расход (на единицу длины по окружной координате):
\begin{equation}
q_r(r,\theta) = -\frac{h^3}{12\mu}\,\frac{\partial p}{\partial r}.
\label{eq:thrust_q_r}
\end{equation}
Поскольку радиальные скорости обеих поверхностей равны нулю, куэттовская составляющая в радиальном направлении отсутствует. Радиальный расход полностью определяется градиентом давления (пуазейлевская составляющая).

Окружной расход (на единицу длины по радиальной координате):
\begin{equation}
q_\theta(r,\theta) = \frac{h\,\omega\,r}{2} - \frac{h^3}{12\mu\,r}\,\frac{\partial p}{\partial \theta}.
\label{eq:thrust_q_theta}
\end{equation}
Первое слагаемое в~\eqref{eq:thrust_q_theta} представляет куэттовскую составляющую расхода, обусловленную увлечением жидкости вращающейся поверхностью. При $U_{\theta,1}=0$ и $U_{\theta,2}=\omega r$ средняя скорость потока составляет $\bar{U}_\theta = \omega r / 2$, а объёмный расход через единицу длины равен $h\,\bar{U}_\theta = h\omega r/2$. Второе слагаемое -- пуазейлевская составляющая; множитель $1/r$ в знаменателе возникает из-за метрического коэффициента при переходе от производной по дуге $\partial/\partial(r\theta)$ к производной по углу $\partial/\partial\theta$.

Суммарные объёмные расходы через радиальные границы подшипника получаются интегрированием удельных расходов по окружной кромке:
\begin{equation}
Q_{\mathrm{out}} = \int_{\theta_0}^{\theta_1} q_r(R_{\mathrm{out}}, \theta)\,R_{\mathrm{out}}\,d\theta, \qquad Q_{\mathrm{in}} = \int_{\theta_0}^{\theta_1} q_r(R_{\mathrm{in}}, \theta)\,R_{\mathrm{in}}\,d\theta.
\label{eq:thrust_Q_radial}
\end{equation}
Множитель $R_{\mathrm{out}}$ (соответственно, $R_{\mathrm{in}}$) перед $d\theta$ обусловлен элементом длины дуги $dl = r\,d\theta$ на окружности радиуса $r$.

Разность $Q_{\mathrm{out}} - Q_{\mathrm{in}}$ определяет утечку смазки через радиальные кромки сектора. Суммарный окружной расход через входную и выходную окружные кромки дополняет баланс массы: полный расход жидкости, поступающей в сектор, равен полному расходу, покидающему его.

Баланс массы в стационарном режиме без выжимания записывается в дифференциальной форме:
\begin{equation}
\frac{1}{r}\,\frac{\partial (r\,q_r)}{\partial r} + \frac{1}{r}\,\frac{\partial q_\theta}{\partial \theta} = 0.
\label{eq:thrust_mass_balance}
\end{equation}
Подстановка выражений~\eqref{eq:thrust_q_r} и~\eqref{eq:thrust_q_theta} в уравнение~\eqref{eq:thrust_mass_balance} приводит к уравнению Рейнольдса~\eqref{eq:thrust_reynolds}, что подтверждает внутреннюю согласованность полученных соотношений.


\subsubsection*{Целевые функционалы}

Результатом решения уравнения Рейнольдса является поле давления $p(r,\theta)$. На его основе вычисляются следующие интегральные характеристики подшипника.

\textit{(а) Несущая способность (осевая нагрузка).} Сила, воспринимаемая одним сектором подшипника:
\begin{equation}
W = \int_{\theta_0}^{\theta_1}\int_{R_{\mathrm{in}}}^{R_{\mathrm{out}}} p(r,\theta)\,r\,dr\,d\theta.
\label{eq:thrust_W}
\end{equation}
Множитель~$r$ в подынтегральном выражении обусловлен элементом площади в цилиндрических координатах $dA = r\,dr\,d\theta$. Для подшипника, состоящего из $N$ одинаковых секторов, полная воспринимаемая нагрузка составляет $W_{\mathrm{total}} = N\,W$.

\textit{(б) Координаты центра давления.} Точка приложения равнодействующей давления определяется соотношениями:
\begin{equation}
r_c = \frac{1}{W}\int_{\theta_0}^{\theta_1}\!\int_{R_{\mathrm{in}}}^{R_{\mathrm{out}}} p\,r^2\,dr\,d\theta, \qquad \theta_c = \frac{1}{W}\int_{\theta_0}^{\theta_1}\!\int_{R_{\mathrm{in}}}^{R_{\mathrm{out}}} p\,\theta\,r\,dr\,d\theta.
\label{eq:thrust_pressure_center}
\end{equation}
Формула для $\theta_c$ корректна для сектора с угловой протяжённостью $\beta \ll 2\pi$, когда угловая координата может рассматриваться как квазилинейная. Для широких секторов или полного кольца более строгим является определение через тригонометрические моменты с использованием $\mathrm{atan2}$.

Знание координат центра давления необходимо при проектировании самоустанавливающихся колодок, опорная точка которых должна совпадать с центром давления для обеспечения устойчивой работы.

Качественный вид распределения давления на секторе и положение равнодействующей показаны на рисунке~\ref{fig:thrust_pressure_field}.

% [РИСУНОК: Качественное поле давления на секторе упорного подшипника]
% Должен показывать:
% - Сектор (R_in, R_out, θ_0, θ_1) — вид сверху
% - Контурные линии (или псевдокарту) p(r,θ) с максимумом ближе к выходной кромке
% - Стрелку W (равнодействующая) и точку (r_c, θ_c)
% - Обозначения границ с p = 0
\begin{figure}[!ht]
\centering
\fbox{\parbox{0.8\textwidth}{\centering
\vspace{3cm}
ПЛЕЙСХОЛДЕР ДЛЯ РИСУНКА\\
Качественное распределение давления $p(r,\theta)$ на секторе:\\
изолинии давления, максимум ближе к выходной кромке,\\
равнодействующая $W$ и центр давления $(r_c, \theta_c)$
\vspace{3cm}
}}
\caption{Качественное распределение давления на секторе упорного подшипника}
\label{fig:thrust_pressure_field}
\end{figure}

\textit{(в) Профиль окружной скорости и сдвиговые напряжения.} Профиль скорости в окружном направлении записывается по аналогии с~\eqref{eq:velocity_profile_u}, с заменой $x \to r\theta$ и подстановкой $U_1 = 0$, $U_2 = \omega r$:
\begin{equation}
u_\theta(y) = \frac{y}{h}\,\omega r + \frac{1}{2\mu}\,\frac{1}{r}\,\frac{\partial p}{\partial \theta}\,y(y - h).
\label{eq:thrust_velocity_theta}
\end{equation}
Первый член описывает линейный (куэттовский) профиль, обусловленный движением верхней поверхности; второй -- параболическую (пуазейлевскую) добавку, определяемую градиентом давления.

Сдвиговое напряжение на вращающейся поверхности ($y = h$) определяется дифференцированием~\eqref{eq:thrust_velocity_theta}:
\begin{equation}
\tau_w(r,\theta) = \mu\,\frac{\partial u_\theta}{\partial y}\bigg|_{y=h} = \frac{\mu\,\omega\,r}{h} + \frac{h}{2r}\,\frac{\partial p}{\partial \theta}.
\label{eq:thrust_tau_w}
\end{equation}
В выражении~\eqref{eq:thrust_tau_w} первое слагаемое представляет вязкое (куэттовское) трение, пропорциональное скорости скольжения и обратно пропорциональное толщине зазора. Второе слагаемое -- вклад градиента давления (пуазейлевская составляющая), который может как увеличивать, так и уменьшать суммарное касательное напряжение в зависимости от знака $\partial p/\partial \theta$.

\textit{(г) Момент трения.} Суммарный момент сил трения относительно оси вращения подшипника:
\begin{equation}
M = \int_{\theta_0}^{\theta_1}\int_{R_{\mathrm{in}}}^{R_{\mathrm{out}}} \tau_w(r,\theta)\,r^2\,dr\,d\theta.
\label{eq:thrust_M}
\end{equation}
Множитель $r^2$ в подынтегральном выражении складывается из двух сомножителей: $r$ от элемента площади $dA = r\,dr\,d\theta$ и ещё один $r$ -- плечо касательной силы относительно оси вращения.

\textit{(д) Мощность потерь на трение:}
\begin{equation}
P_{\mathrm{loss}} = \omega\,M.
\label{eq:thrust_P_loss}
\end{equation}
Величина $P_{\mathrm{loss}}$ определяет тепловыделение в смазочном слое и является входным параметром для теплового расчёта подшипника.

\textit{(е) Коэффициент трения.} Суммарная тангенциальная сила трения на вращающейся поверхности:
\begin{equation}
F_t = \int_{\theta_0}^{\theta_1}\int_{R_{\mathrm{in}}}^{R_{\mathrm{out}}} \tau_w(r,\theta)\,r\,dr\,d\theta.
\label{eq:thrust_F_t}
\end{equation}
Коэффициент трения определяется как отношение силы трения к несущей способности:
\begin{equation}
f = \frac{F_t}{W}.
\label{eq:thrust_f_strict}
\end{equation}
В инженерных расчётах часто используют приближённое соотношение $F_t \approx M/R_m$, что приводит к оценке
\begin{equation}
f \approx \frac{M}{R_m\,W},
\label{eq:thrust_f}
\end{equation}
где $R_m = (R_{\mathrm{in}} + R_{\mathrm{out}})/2$ -- средний радиус рабочей зоны подшипника, принимаемый как характерный радиус приложения равнодействующей силы трения. Меньшие значения $f$ соответствуют более низким потерям на трение при заданной несущей способности.

\textit{(ж) Минимальная толщина смазочного слоя:}
\begin{equation}
h_{\min} = \min_{(r,\theta)} h(r,\theta).
\label{eq:thrust_h_min_target}
\end{equation}
Величина $h_{\min}$ определяет запас по толщине зазора и, следовательно, безопасность подшипника от контакта поверхностей. При проектировании $h_{\min}$ должна превышать суммарную шероховатость рабочих поверхностей с определённым коэффициентом запаса. Основные обозначения, введённые в настоящем подразделе, сведены в таблицу~\ref{tab:thrust_notations}.

\begin{table}[!ht]
\centering
\caption{Основные обозначения для упорного подшипника}
\label{tab:thrust_notations}
\begin{tabular}{|l|l|l|}
\hline
\textbf{Обозначение} & \textbf{Величина} & \textbf{Размерность} \\
\hline
$R_{\mathrm{in}}$, $R_{\mathrm{out}}$ & Внутренний и наружный радиусы & м \\
$\theta_0$, $\theta_1$, $\beta$ & Границы и угол сектора & рад \\
$\omega$ & Угловая скорость & рад/с \\
$\mu$ & Динамическая вязкость & Па$\cdot$с \\
$h(r,\theta)$ & Толщина смазочного слоя & м \\
$h_{\mathrm{in}}$, $h_{\mathrm{out}}$ & Толщина на входной и выходной кромках & м \\
$h_{\min}$ & Минимальная толщина зазора & м \\
$K$, $\delta$ & Параметры конвергенции клина & -- \\
$p(r,\theta)$ & Давление в смазочном слое & Па \\
$W$ & Несущая способность (нагрузка) & Н \\
$M$ & Момент трения & Н$\cdot$м \\
$P_{\mathrm{loss}}$ & Мощность потерь на трение & Вт \\
$Q_{\mathrm{in}}$, $Q_{\mathrm{out}}$ & Расходы через радиальные границы & м$^3$/с \\
$f$ & Коэффициент трения & -- \\
\hline
\end{tabular}
\end{table}

Сформулированная задача сводится к решению эллиптического уравнения Рейнольдса~\eqref{eq:thrust_reynolds_num} в области~\eqref{eq:thrust_domain} при заданном профиле зазора $h(r,\theta)$ и граничных условиях на давление. Структура уравнения (эллиптический оператор в левой части, источниковый член в правой) обеспечивает единственность решения при корректно поставленных граничных условиях Дирихле в отсутствие кавитации, т.\,е. для линейной постановки без ограничений на знак давления.

Численная дискретизация уравнения~\eqref{eq:thrust_reynolds_num} методом конечных разностей и алгоритм учёта кавитации на основе модели JFO рассматриваются в главе~3. При наличии микроструктур на рабочей поверхности функция зазора $h(r,\theta)$ приобретает быстрые локальные вариации, параметризация которых описывается в разделе~2.2.


\subsection{Радиальная постановка задачи}

% Радиальный подшипник (journal bearing)
%
% СОДЕРЖАНИЕ: аналогично 2.1.2

Рассмотрим радиальный подшипник...

% [РИСУНОК 2.3: Схема радиального подшипника]
\begin{figure}[!ht]
\centering
\fbox{\parbox{0.8\textwidth}{\centering
\vspace{3cm}
ПЛЕЙСХОЛДЕР ДЛЯ РИСУНКА 2.3\\
Схема радиального подшипника\\
Вал радиуса R, втулка, эксцентриситет e, координаты
\vspace{3cm}
}}
\caption{Схема радиального подшипника с координатами}
\label{fig:journal_bearing}
\end{figure}


\subsection{Возвратно-поступательная постановка задачи}

% Пара плунжер-цилиндр
%
% СОДЕРЖАНИЕ: аналогично 2.1.2

Рассмотрим пару плунжер-цилиндр...

% [РИСУНОК 2.4: Схема пары плунжер-цилиндр]
\begin{figure}[!ht]
\centering
\fbox{\parbox{0.8\textwidth}{\centering
\vspace{3cm}
ПЛЕЙСХОЛДЕР ДЛЯ РИСУНКА 2.4\\
Схема пары плунжер-цилиндр\\
Координаты, направление движения, зазор
\vspace{3cm}
}}
\caption{Схема пары плунжер-цилиндр}
\label{fig:plunger_cylinder}
\end{figure}


\subsection{Допущения и область применимости модели}

% СОДЕРЖАНИЕ:
% - Ньютоновская жидкость (μ = const)
% - Тонкий слой (h << L)
% - Ламинарность (Re << 1)
% - Изотермичность / неизотермичность
% - Несжимаемость (ρ = const)
% - Отсутствие инерции
% - Где модель ломается

Допущения, принятые при выводе уравнения Рейнольдса, перечислены в подразделе~2.1.1. Здесь обсудим границы применимости полученной модели и условия, при которых отдельные допущения могут нарушаться.

В рамках смазочной аппроксимации ключевым является малый параметр $\varepsilon = H/L$. При росте числа Рейнольдса или при больших уклонах поверхности точность модели снижается. Ниже перечислены основные факторы, требующие расширения модели.

% СОДЕРЖАНИЕ 2.1.5 (границы применимости):
% - При каких значениях Re·ε модель теряет точность (инерция)
% - Турбулентность: при каких Re ламинарность нарушается
% - Нагрев: когда нужна μ(T) и уравнение энергии
% - Неньютоновское поведение: полимерные/загущённые масла
% - Большие уклоны поверхности: |dh/dx| ~ 1
% - Упругое деформирование поверхностей (EHL)
% - Что из перечисленного снимается/расширяется далее в монографии

% TODO: Расписать границы применимости по пунктам выше


\subsection{Граничные условия и условия замыкания}

% СОДЕРЖАНИЕ:
% - Что на входе/выходе
% - Боковые границы
% - Отрицательное давление
% - Реализация кавитации (мост к разделу 2.3)

Для замыкания задачи необходимо задать граничные условия...

% TODO: Описать ГУ для каждой постановки


\subsection{Целевые функционалы и выходные величины}

% СОДЕРЖАНИЕ:
% - Нагрузка W
% - Расход Q
% - Потери/трение F
% - Коэффициент трения f
% - Минимальная толщина пленки h_min
% - Кавитационная зона

Основными выходными величинами являются...

% [ТАБЛИЦА 2.1: Безразмерные группы и масштабы]
\begin{table}[h]
\centering
\caption{Безразмерные группы и характерные масштабы}
\label{tab:dimensionless_groups}
\begin{tabular}{|l|l|l|}
\hline
\textbf{Величина} & \textbf{Размерная} & \textbf{Безразмерная} \\
\hline
Давление & $p$ & $\bar{p} = p h_0^2 / (\mu U L)$ \\
Координата & $x$ & $\bar{x} = x / L$ \\
Толщина & $h$ & $\bar{h} = h / h_0$ \\
% TODO: Добавить остальные & & \\
\hline
\end{tabular}
\end{table}


% =============================================================
\section{Модель с детерминированными микроструктурами}

\subsection{Параметризация микрорельефа: $h(x,z)$ и параметры}

% СОДЕРЖАНИЕ:
% - Как задается функция h(x,z)
% - Параметры: глубина h_d, радиусы a и b, шаг p, доля площади φ

Для учёта детерминированных микроструктур рабочую поверхность представляем в виде гладкой базовой поверхности с наложенными регулярными элементами. Суммарная толщина смазочного слоя записывается как суперпозиция:
\begin{equation}
h(x, z) = h_{\mathrm{base}}(x, z) + \Delta h(x, z),
\label{eq:h_superposition}
\end{equation}
где $h_{\mathrm{base}}$ -- толщина зазора для гладкой поверхности (клиновой профиль, определённый в подразделах~2.1.2--2.1.4), а $\Delta h$ -- локальная добавка, описывающая геометрию микроструктуры ($\Delta h \leq 0$ для углублений, $\Delta h \geq 0$ для выступов). Принцип суперпозиции проиллюстрирован на рисунке~\ref{fig:h_superposition}.

% [РИСУНОК: Суперпозиция зазора h = h_base + Δh]
% Должен показывать два вида:
% (1) Продольный разрез: базовый клин h_base + ямка Δh → суммарный h
% (2) Вид сверху: пятно одного элемента микроструктуры на фоне сектора
\begin{figure}[!ht]
\centering
\fbox{\parbox{0.8\textwidth}{\centering
\vspace{3.5cm}
ПЛЕЙСХОЛДЕР ДЛЯ РИСУНКА\\
Суперпозиция зазора $h = h_{\mathrm{base}} + \Delta h$:\\
(a) продольный разрез -- базовый клин + локальная ямка;\\
(b) вид сверху -- пятно элемента микроструктуры на секторе
\vspace{3.5cm}
}}
\caption{Суперпозиция толщины зазора: базовый профиль и локальная микроструктура}
\label{fig:h_superposition}
\end{figure}

% [РИСУНОК 2.5: Типовая микроструктура с параметрами]
\begin{figure}[!ht]
\centering
\fbox{\parbox{0.8\textwidth}{\centering
\vspace{3cm}
ПЛЕЙСХОЛДЕР ДЛЯ РИСУНКА 2.5\\
Типовая микроструктура (лунка/канавка)\\
Параметры: $a$, $b$, $h_d$, профиль в разрезе
\vspace{3cm}
}}
\caption{Типовая микроструктура с основными параметрами}
\label{fig:microtexture_params}
\end{figure}

% [ТАБЛИЦА 2.2: Параметры микроструктуры]
\begin{table}[h]
\centering
\caption{Параметры детерминированной микроструктуры}
\label{tab:texture_params}
\begin{tabular}{|l|l|l|}
\hline
\textbf{Параметр} & \textbf{Обозначение} & \textbf{Размерность} \\
\hline
Глубина элемента & $h_d$ & мкм \\
Радиус (полуось) & $a, b$ & мкм \\
Шаг раскладки & $p$ & мкм \\
Доля площади & $\phi$ & -- \\
% TODO: Добавить остальные & & \\
\hline
\end{tabular}
\end{table}


\subsection{Типы геометрии микроструктур}

% СОДЕРЖАНИЕ:
% - Лунки (круглые/эллиптические/эллипсоидные)
% - Канавки/борозды (продольные/поперечные)
% - Комбинированные элементы

Рассмотрим основные типы геометрии микроструктур...

% [РИСУНОК 2.6: Геометрия различных микроструктур]
\begin{figure}[!ht]
\centering
\fbox{\parbox{0.8\textwidth}{\centering
\vspace{3cm}
ПЛЕЙСХОЛДЕР ДЛЯ РИСУНКА 2.6\\
Типы микроструктур:\\
(a) круглая лунка, (b) эллиптическая лунка,\\
(c) продольная канавка, (d) поперечная канавка
\vspace{3cm}
}}
\caption{Основные типы геометрии микроструктур}
\label{fig:texture_geometries}
\end{figure}


\subsection{Типы раскладки микроструктур}

% СОДЕРЖАНИЕ:
% - Регулярная решетка (шахматная/прямоугольная/гексагональная)
% - Кольцевая/секторальная
% - Филлотаксис

Рассмотрим варианты пространственного расположения микроструктур...

% [РИСУНОК 2.7: Шахматная раскладка]
\begin{figure}[!ht]
\centering
\fbox{\parbox{0.6\textwidth}{\centering
\vspace{2cm}
ПЛЕЙСХОЛДЕР ДЛЯ РИСУНКА 2.7\\
Шахматная раскладка микролунок\\
(вид сверху, координаты центров)
\vspace{2cm}
}}
\caption{Шахматная раскладка микроструктур}
\label{fig:layout_staggered}
\end{figure}

% [РИСУНОК 2.8: Кольцевая раскладка]
\begin{figure}[!ht]
\centering
\fbox{\parbox{0.6\textwidth}{\centering
\vspace{2cm}
ПЛЕЙСХОЛДЕР ДЛЯ РИСУНКА 2.8\\
Кольцевая раскладка (для радиального подшипника)
\vspace{2cm}
}}
\caption{Кольцевая раскладка микроструктур}
\label{fig:layout_annular}
\end{figure}

% [РИСУНОК 2.9: Филлотаксис]
\begin{figure}[!ht]
\centering
\fbox{\parbox{0.6\textwidth}{\centering
\vspace{2cm}
ПЛЕЙСХОЛДЕР ДЛЯ РИСУНКА 2.9\\
Раскладка по принципу филлотаксиса
\vspace{2cm}
}}
\caption{Раскладка микроструктур по принципу филлотаксиса}
\label{fig:layout_phyllotaxis}
\end{figure}


\subsection{Безразмеризация и определяющие параметры}

% СОДЕРЖАНИЕ:
% - Переход к безразмерным переменным
% - Набор определяющих параметров
% - Упрощенные формы для конкретных геометрий

Переход к безразмерным переменным...

% TODO: Написать про безразмеризацию


% =============================================================
\section{Кавитация и уравнения JFO (массо-сохраняющая постановка)}

\subsection{Физика разрыва пленки и ограничения модели с обнулением давления}

% СОДЕРЖАНИЕ:
% - Что происходит при отрицательном давлении
% - Почему простое p=0 неверно (нарушение массового баланса)
% - Физическая картина кавитации

При решении уравнения Рейнольдса в ряде случаев получаются 
отрицательные значения давления...

% [РИСУНОК 2.10: Схема кавитации]
\begin{figure}[!ht]
\centering
\fbox{\parbox{0.8\textwidth}{\centering
\vspace{3cm}
ПЛЕЙСХОЛДЕР ДЛЯ РИСУНКА 2.10\\
Схема кавитации в смазочном слое:\\
зона полного заполнения (p > 0),\\
зона кавитации (p = 0, частичное заполнение)
\vspace{3cm}
}}
\caption{Схематическое представление кавитации в смазочном слое}
\label{fig:cavitation_scheme}
\end{figure}


\subsection{Уравнения Jakobsson-Floberg-Olsson}

% СОДЕРЖАНИЕ:
% - Полная формулировка JFO
% - Условия переключения
% - Форма, пригодная для численной реализации

Массо-сохраняющая модель кавитации, предложенная Jakobsson, Floberg и Olsson...

% TODO: Написать уравнения JFO

Принципиальное отличие модели JFO от простого обнуления давления проиллюстрировано на рисунке~\ref{fig:jfo_1d_profile}.

% [РИСУНОК: 1D-профиль давления и степени заполнения в модели JFO]
% Должен показывать:
% - Ось абсцисс: координата вдоль направления течения (θ или x)
% - Верхний график: p(θ) — давление; в зоне кавитации p = 0 (горизонтальная линия)
% - Нижний график (или наложенный): α(θ) или θ_f(θ) — степень заполнения;
%   α = 1 в зоне полного слоя, α < 1 в зоне кавитации
% - Вертикальные штриховые линии: граница rupture (разрыва) и reformation (восстановления)
% - Для сравнения: пунктиром показать "обнулённое" p < 0 (модель Гюмбеля)
\begin{figure}[!ht]
\centering
\fbox{\parbox{0.8\textwidth}{\centering
\vspace{3.5cm}
ПЛЕЙСХОЛДЕР ДЛЯ РИСУНКА\\
1D-профиль вдоль направления течения:\\
давление $p(\theta)$ и степень заполнения $\alpha(\theta)$;\\
границы разрыва (rupture) и восстановления (reformation);\\
пунктир -- решение с обнулением $p \geq 0$ (Гюмбель)
\vspace{3.5cm}
}}
\caption{Сравнение моделей кавитации: обнуление давления (Гюмбель) и массо-сохраняющая постановка (JFO)}
\label{fig:jfo_1d_profile}
\end{figure}


\subsection{Связь с численной реализацией}

% СОДЕРЖАНИЕ:
% - Что именно будет реализовано в главе 3
% - Как JFO встраивается в численную схему
% - Алгоритм переключения между зонами

В главе 3 будет представлена численная реализация описанной модели...

% [РИСУНОК 2.11: Блок-схема модели]
\begin{figure}[!ht]
\centering
\fbox{\parbox{0.8\textwidth}{\centering
\vspace{3cm}
ПЛЕЙСХОЛДЕР ДЛЯ РИСУНКА 2.11\\
Блок-схема:\\
Входные параметры → Уравнения Рейнольдса + JFO →\\
→ Численное решение → Выходные величины
\vspace{3cm}
}}
\caption{Блок-схема математической модели}
\label{fig:model_flowchart}
\end{figure}


% =============================================================
\section*{Выводы по главе 2}
\addcontentsline{toc}{section}{Выводы по главе 2}

% СОДЕРЖАНИЕ (0.5-1.5 страницы):
% - 5-10 пунктов о принятой модели
% - Какие допущения
% - Какие безразмерные формы получены
% - Какие граничные условия/замыкание (JFO)
% - Какие параметры микроструктуры будут варьироваться

\begin{enumerate}
\item В главе представлена математическая постановка задачи гидродинамической 
смазки с учетом детерминированных микроструктур рабочей поверхности.

\item Получено уравнение Рейнольдса для описания распределения давления 
в тонком смазочном слое...

% TODO: Добавить остальные выводы (всего 5-10 пунктов)

\end{enumerate}