\chapter{Постановка и численное моделирование гидродинамических задач смазочного слоя с детерминированными микроструктурами рабочей поверхности}

% =============================================================
% ГЛАВА 2
% Структура с плейсхолдерами для рисунков и таблиц
% =============================================================

\section{Классическая модель для гладкой поверхности}

\subsection{Ключевые допущения и вывод уравнения Рейнольдса}

Рассмотрим движение вязкой несжимаемой ньютоновской жидкости в тонком зазоре между двумя твёрдыми поверхностями. Введём декартову систему координат $(x, y, z)$, где плоскость $xz$ совпадает с нижней поверхностью, а ось $y$ направлена перпендикулярно поверхностям (рисунок~\ref{fig:lubrication_scheme}). Толщина смазочного слоя обозначается как $h(x, z, t)$, где $t$ -- время.

% [РИСУНОК 2.1: Схема смазочного слоя]
% Должен показывать:
% - Две поверхности (верхняя и нижняя)
% - Координатные оси x, y, z
% - Толщина слоя h(x,z,t)
% - Скорости поверхностей U₁, U₂
% - Профиль скорости u(y) внутри слоя
% - Давление p(x,z,t)
\begin{figure}[h]
\centering
\fbox{\parbox{0.8\textwidth}{\centering
\vspace{3cm}
ПЛЕЙСХОЛДЕР ДЛЯ РИСУНКА 2.1\\
Схема смазочного слоя между двумя поверхностями\\
(найти рисунок "Reynolds lubrication schematic")
\vspace{3cm}
}}
\caption{Схема смазочного слоя между двумя поверхностями}
\label{fig:lubrication_scheme}
\end{figure}

\subsubsection*{Основные допущения}

Вывод уравнения Рейнольдса основан на следующих физических допущениях:

\begin{enumerate}
\item Жидкость ньютоновская: касательные напряжения пропорциональны скоростям деформации, в частности $\tau_{xy} = \mu\,\partial u/\partial y$, $\tau_{yz} = \mu\,\partial w/\partial y$.

\item Жидкость несжимаемая: $\rho = \mathrm{const}$.

\item Тонкий слой: $H \ll L$, где $H$ --- характерный (порядковый) масштаб толщины смазочного слоя $h(x,z,t)$; малые уклоны поверхностей $|\partial h/\partial x| \ll 1$, $|\partial h/\partial z| \ll 1$.

\item Течение ламинарное, инерционные эффекты пренебрежимо малы: $\mathrm{Re}\cdot\varepsilon \ll 1$, где $\varepsilon = H/L$.

\item Давление постоянно по толщине слоя: $\partial p/\partial y \approx 0$ (следствие смазочной аппроксимации).

\item Динамическая вязкость постоянна: $\mu = \mathrm{const}$ (температурная зависимость рассматривается отдельно при необходимости).

\item На границах жидкости и твёрдых поверхностей выполняется условие прилипания (no-slip).

\item Объёмные силы (гравитация и др.) малы по сравнению с градиентом давления и не учитываются.
\end{enumerate}

\subsubsection*{Исходные уравнения}

Движение жидкости описывается уравнениями Навье--Стокса для несжимаемой жидкости:

\begin{equation}
\rho\left(\frac{\partial \mathbf{v}}{\partial t} + (\mathbf{v} \cdot \nabla)\mathbf{v}\right) = -\nabla p + \mu \nabla^2 \mathbf{v},
\label{eq:navier_stokes}
\end{equation}

и уравнением неразрывности:

\begin{equation}
\nabla \cdot \mathbf{v} = 0,
\label{eq:continuity}
\end{equation}

где $\mathbf{v} = (u, v, w)$ -- вектор скорости;
\hphantom{где }$\rho$ -- плотность жидкости;
\hphantom{где }$p$ -- давление;
\hphantom{где }$\mu$ -- динамическая вязкость.

В компонентной форме уравнения движения приведены в приложении~А. Уравнение неразрывности~\eqref{eq:continuity} в компонентной записи имеет вид:

\begin{equation}
\frac{\partial u}{\partial x} + \frac{\partial v}{\partial y} + \frac{\partial w}{\partial z} = 0.
\label{eq:continuity_comp}
\end{equation}

Далее, на основе анализа порядков величин, будет получена смазочная аппроксимация и выведено уравнение Рейнольдса.

\subsubsection*{Масштабирование и оценка порядков величин}

Для анализа уравнений введём характерные масштабы задачи:

$L$ -- характерный размер в направлениях $x$ и $z$ (длина подшипника);

$H$ -- характерная толщина смазочного слоя;

$U$ -- характерная скорость движения поверхностей;

$P$ -- характерное давление.

Безразмерные переменные вводятся следующим образом:

\begin{equation}
\bar{x} = \frac{x}{L}, \quad \bar{y} = \frac{y}{H}, \quad \bar{z} = \frac{z}{L}, \quad \bar{u} = \frac{u}{U}, \quad \bar{v} = \frac{v}{V}, \quad \bar{w} = \frac{w}{U}, \quad \bar{p} = \frac{p}{P},
\end{equation}

где $V$ -- характерная скорость в направлении $y$, которую нужно определить из уравнения неразрывности.

Из уравнения неразрывности следует оценка:

\begin{equation}
\frac{U}{L} \sim \frac{V}{H} \quad \Rightarrow \quad V \sim U\frac{H}{L}.
\end{equation}

Ключевым параметром в теории смазки является отношение:

\begin{equation}
\varepsilon = \frac{H}{L} \ll 1,
\label{eq:epsilon}
\end{equation}

которое характеризует малость толщины слоя по сравнению с его протяжённостью.

Подставляя безразмерные переменные в компонентные уравнения движения, получаем оценки для различных членов. Рассмотрим $y$-компоненту уравнения движения:

\begin{equation}
\rho\frac{U^2}{L}\varepsilon \sim \frac{P}{H} + \mu\frac{U}{H^2}.
\end{equation}

Определяем характерный масштаб давления из баланса вязких сил и градиента давления:

\begin{equation}
P \sim \mu U\frac{L}{H^2}.
\label{eq:pressure_scale}
\end{equation}

Характерное давление определяется балансом вязкого сопротивления потоку в тонком зазоре и продольного градиента давления.

Вводим число Рейнольдса для смазочного слоя:

\begin{equation}
\text{Re} = \frac{\rho U H}{\mu}.
\end{equation}

При условии $\text{Re} \cdot \varepsilon \ll 1$ (что выполняется в большинстве практических случаев) инерционные члены в уравнениях движения пренебрежимо малы по сравнению с вязкими и градиентом давления. Малость этого параметра означает, что инерция жидкости подавлена тонкостью смазочного слоя ($\varepsilon = H/L \ll 1$), и течение полностью определяется балансом вязких сил и градиента давления.

Анализируя порядки производных с учётом $\varepsilon \ll 1$, получаем:

\begin{equation}
\frac{\partial^2}{\partial y^2} \gg \frac{\partial^2}{\partial x^2}, \quad \frac{\partial^2}{\partial y^2} \gg \frac{\partial^2}{\partial z^2}.
\end{equation}

Таким образом, в $x$- и $z$-компонентах уравнения движения доминируют члены:

\begin{equation}
\frac{\partial p}{\partial x} \sim \mu\frac{\partial^2 u}{\partial y^2}, \quad \frac{\partial p}{\partial z} \sim \mu\frac{\partial^2 w}{\partial y^2}.
\end{equation}

Из $y$-компоненты уравнения движения следует, что $\partial p/\partial y \sim \varepsilon^2 \ll 1$, откуда:

\begin{equation}
\frac{\partial p}{\partial y} \approx 0.
\label{eq:dp_dy_zero}
\end{equation}

Это означает, что давление в смазочном слое не зависит от координаты $y$: $p = p(x, z, t)$.

\subsubsection*{Упрощённые уравнения движения}

С учётом сделанных допущений уравнения движения принимают вид:

\begin{align}
\frac{\partial p}{\partial x} &= \mu\frac{\partial^2 u}{\partial y^2}, \label{eq:simplified_x} \\
\frac{\partial p}{\partial z} &= \mu\frac{\partial^2 w}{\partial y^2}. \label{eq:simplified_z}
\end{align}

Уравнение неразрывности сохраняет форму~\eqref{eq:continuity_comp}.

\subsubsection*{Граничные условия}

На поверхностях выполняются условия прилипания (no-slip):

\begin{equation}
\begin{aligned}
&y = 0: \quad u = U_1(x, z, t), \quad w = W_1(x, z, t), \quad v = V_1(x, z, t), \\
&y = h: \quad u = U_2(x, z, t), \quad w = W_2(x, z, t), \quad v = V_2(x, z, t),
\end{aligned}
\label{eq:boundary_conditions}
\end{equation}

где $U_i$, $W_i$, $V_i$ -- компоненты скорости соответствующих поверхностей.

\subsubsection*{Профиль скорости}

Интегрируя уравнение~\eqref{eq:simplified_x} дважды по $y$ и используя граничные условия~\eqref{eq:boundary_conditions}, получаем профиль скорости в направлении $x$:

\begin{equation}
u(x, y, z, t) = U_1 + \frac{y}{h}(U_2 - U_1) + \frac{1}{2\mu}\frac{\partial p}{\partial x}y(y - h).
\label{eq:velocity_profile_u}
\end{equation}

Аналогично для направления $z$:

\begin{equation}
w(x, y, z, t) = W_1 + \frac{y}{h}(W_2 - W_1) + \frac{1}{2\mu}\frac{\partial p}{\partial z}y(y - h).
\label{eq:velocity_profile_w}
\end{equation}

Профиль скорости~\eqref{eq:velocity_profile_u} состоит из двух частей: линейной (куэттовский поток, обусловленный движением поверхностей) и параболической (пуазейлевский поток, обусловленный градиентом давления).

\subsubsection*{Объёмный расход}

Объёмный расход жидкости через единицу длины перпендикулярно направлению $x$ определяется интегрированием скорости по толщине слоя:

\begin{equation}
q_x = \int_0^h u \, dy.
\end{equation}

Подставляя выражение~\eqref{eq:velocity_profile_u}:

\begin{equation}
q_x = \int_0^h \left[U_1 + \frac{y}{h}(U_2 - U_1) + \frac{1}{2\mu}\frac{\partial p}{\partial x}y(y - h)\right] dy.
\end{equation}

Вычисляем интегралы:

\begin{equation}
\begin{aligned}
\int_0^h U_1 \, dy &= U_1 h, \\
\int_0^h \frac{y}{h}(U_2 - U_1) \, dy &= \frac{U_2 - U_1}{h} \cdot \frac{h^2}{2} = \frac{h}{2}(U_2 - U_1), \\
\int_0^h \frac{1}{2\mu}\frac{\partial p}{\partial x}y(y - h) \, dy &= \frac{1}{2\mu}\frac{\partial p}{\partial x}\left(\frac{h^3}{3} - \frac{h^3}{2}\right) = -\frac{h^3}{12\mu}\frac{\partial p}{\partial x}.
\end{aligned}
\end{equation}

Окончательно для расхода в направлении $x$:

\begin{equation}
q_x = \frac{h}{2}(U_1 + U_2) - \frac{h^3}{12\mu}\frac{\partial p}{\partial x}.
\label{eq:flow_rate_x}
\end{equation}

Аналогично для направления $z$:

\begin{equation}
q_z = \frac{h}{2}(W_1 + W_2) - \frac{h^3}{12\mu}\frac{\partial p}{\partial z}.
\label{eq:flow_rate_z}
\end{equation}

\subsubsection*{Вывод уравнения Рейнольдса}

Из уравнения неразрывности~\eqref{eq:continuity}, интегрируя по толщине слоя от $y = 0$ до $y = h$:

\begin{equation}
\int_0^h \frac{\partial u}{\partial x} \, dy + \int_0^h \frac{\partial v}{\partial y} \, dy + \int_0^h \frac{\partial w}{\partial z} \, dy = 0.
\end{equation}

Используя правило Лейбница для дифференцирования интеграла:

\begin{equation}
\frac{\partial}{\partial x}\int_0^h u \, dy = \int_0^h \frac{\partial u}{\partial x} \, dy + u|_{y=h}\frac{\partial h}{\partial x},
\end{equation}

и аналогично для $z$.

Для второго интеграла:

\begin{equation}
\int_0^h \frac{\partial v}{\partial y} \, dy = v|_{y=h} - v|_{y=0} = V_2 - V_1.
\end{equation}

Кинематическое условие на верхней поверхности $y = h(x,z,t)$:

\begin{equation}
V_2 = \frac{\partial h}{\partial t} + U_2\frac{\partial h}{\partial x} + W_2\frac{\partial h}{\partial z}.
\label{eq:kinematic_upper}
\end{equation}

Нижняя поверхность полагается плоской и неподвижной в направлении $y$, откуда $V_1 = 0$.

Подставляя в уравнение неразрывности и используя определения расходов~\eqref{eq:flow_rate_x} и~\eqref{eq:flow_rate_z}:

\begin{equation}
\frac{\partial q_x}{\partial x} + \frac{\partial q_z}{\partial z} + \frac{\partial h}{\partial t} = 0.
\label{eq:continuity_integrated}
\end{equation}

Подставляя выражения для $q_x$ и $q_z$:

\begin{equation}
\frac{\partial}{\partial x}\left[\frac{h}{2}(U_1 + U_2) - \frac{h^3}{12\mu}\frac{\partial p}{\partial x}\right] + \frac{\partial}{\partial z}\left[\frac{h}{2}(W_1 + W_2) - \frac{h^3}{12\mu}\frac{\partial p}{\partial z}\right] + \frac{\partial h}{\partial t} = 0.
\end{equation}

Раскрывая производные:

\begin{equation}
\frac{\partial}{\partial x}\left(\frac{h^3}{12\mu}\frac{\partial p}{\partial x}\right) + \frac{\partial}{\partial z}\left(\frac{h^3}{12\mu}\frac{\partial p}{\partial z}\right) = \frac{\partial}{\partial x}\left[\frac{h}{2}(U_1 + U_2)\right] + \frac{\partial}{\partial z}\left[\frac{h}{2}(W_1 + W_2)\right] + \frac{\partial h}{\partial t}.
\label{eq:reynolds_full}
\end{equation}

Уравнение~\eqref{eq:reynolds_full} является \textbf{обобщённым уравнением Рейнольдса} для случая нестационарного течения с подвижными границами произвольной формы.

Для стационарного случая ($\partial h/\partial t = 0$) и при условии $W_1 = W_2 = 0$ (движение только в направлении $x$) уравнение упрощается:

\begin{equation}
\frac{\partial}{\partial x}\left(\frac{h^3}{12\mu}\frac{\partial p}{\partial x}\right) + \frac{\partial}{\partial z}\left(\frac{h^3}{12\mu}\frac{\partial p}{\partial z}\right) = \frac{U_1 + U_2}{2}\frac{\partial h}{\partial x}.
\label{eq:reynolds_steady}
\end{equation}

Уравнение~\eqref{eq:reynolds_steady} представляет собой \textbf{классическое уравнение Рейнольдса} для гидродинамической смазки.

В случае одной движущейся поверхности (например, $U_1 = U$, $U_2 = 0$):

\begin{equation}
\frac{\partial}{\partial x}\left(\frac{h^3}{\mu}\frac{\partial p}{\partial x}\right) + \frac{\partial}{\partial z}\left(\frac{h^3}{\mu}\frac{\partial p}{\partial z}\right) = 6U\frac{\partial h}{\partial x}.
\label{eq:reynolds_one_surface}
\end{equation}

Полученное уравнение Рейнольдса~\eqref{eq:reynolds_one_surface} является основным уравнением теории гидродинамической смазки и связывает распределение давления $p(x, z)$ в смазочном слое с геометрией зазора $h(x, z)$ и скоростью движения поверхности $U$.

Следует отметить, что при решении уравнения Рейнольдса в областях расходящегося зазора могут возникать зоны отрицательного давления, не имеющие физического смысла. Для корректного описания таких областей необходима массо-сохраняющая модель кавитации, которая рассматривается в разделе~2.3 на основе подхода Jakobsson--Floberg--Olsson (JFO).


\subsection{Упорная постановка задачи}

% Упорный подшипник (thrust bearing)
%
% СОДЕРЖАНИЕ:
% - Геометрия и координаты
% - Кинематика (скорость поверхностей)
% - Закон толщины пленки h(x,z)
% - Граничные условия
% - Что ищем (давление, зона кавитации)
% - Выражения для нагрузки W, трения F, расхода Q

Рассмотрим упорный подшипник...

% [РИСУНОК 2.2: Схема упорного подшипника]
\begin{figure}[h]
\centering
\fbox{\parbox{0.8\textwidth}{\centering
\vspace{3cm}
ПЛЕЙСХОЛДЕР ДЛЯ РИСУНКА 2.2\\
Схема упорного подшипника\\
Координаты ($r$, $\theta$, $y$), границы, направление вращения
\vspace{3cm}
}}
\caption{Схема упорного подшипника с координатами}
\label{fig:thrust_bearing}
\end{figure}


\subsection{Радиальная постановка задачи}

% Радиальный подшипник (journal bearing)
%
% СОДЕРЖАНИЕ: аналогично 2.1.2

Рассмотрим радиальный подшипник...

% [РИСУНОК 2.3: Схема радиального подшипника]
\begin{figure}[h]
\centering
\fbox{\parbox{0.8\textwidth}{\centering
\vspace{3cm}
ПЛЕЙСХОЛДЕР ДЛЯ РИСУНКА 2.3\\
Схема радиального подшипника\\
Вал радиуса R, втулка, эксцентриситет e, координаты
\vspace{3cm}
}}
\caption{Схема радиального подшипника с координатами}
\label{fig:journal_bearing}
\end{figure}


\subsection{Возвратно-поступательная постановка задачи}

% Пара плунжер-цилиндр
%
% СОДЕРЖАНИЕ: аналогично 2.1.2

Рассмотрим пару плунжер-цилиндр...

% [РИСУНОК 2.4: Схема пары плунжер-цилиндр]
\begin{figure}[h]
\centering
\fbox{\parbox{0.8\textwidth}{\centering
\vspace{3cm}
ПЛЕЙСХОЛДЕР ДЛЯ РИСУНКА 2.4\\
Схема пары плунжер-цилиндр\\
Координаты, направление движения, зазор
\vspace{3cm}
}}
\caption{Схема пары плунжер-цилиндр}
\label{fig:plunger_cylinder}
\end{figure}


\subsection{Допущения и область применимости модели}

% СОДЕРЖАНИЕ:
% - Ньютоновская жидкость (μ = const)
% - Тонкий слой (h << L)
% - Ламинарность (Re << 1)
% - Изотермичность / неизотермичность
% - Несжимаемость (ρ = const)
% - Отсутствие инерции
% - Где модель ломается

Допущения, принятые при выводе уравнения Рейнольдса, перечислены в подразделе~2.1.1. Здесь обсудим границы применимости полученной модели и условия, при которых отдельные допущения могут нарушаться.

В рамках смазочной аппроксимации ключевым является малый параметр $\varepsilon = H/L$. При росте числа Рейнольдса или при больших уклонах поверхности точность модели снижается. Ниже перечислены основные факторы, требующие расширения модели.

% СОДЕРЖАНИЕ 2.1.5 (границы применимости):
% - При каких значениях Re·ε модель теряет точность (инерция)
% - Турбулентность: при каких Re ламинарность нарушается
% - Нагрев: когда нужна μ(T) и уравнение энергии
% - Неньютоновское поведение: полимерные/загущённые масла
% - Большие уклоны поверхности: |dh/dx| ~ 1
% - Упругое деформирование поверхностей (EHL)
% - Что из перечисленного снимается/расширяется далее в монографии

% TODO: Расписать границы применимости по пунктам выше


\subsection{Граничные условия и условия замыкания}

% СОДЕРЖАНИЕ:
% - Что на входе/выходе
% - Боковые границы
% - Отрицательное давление
% - Реализация кавитации (мост к разделу 2.3)

Для замыкания задачи необходимо задать граничные условия...

% TODO: Описать ГУ для каждой постановки


\subsection{Целевые функционалы и выходные величины}

% СОДЕРЖАНИЕ:
% - Нагрузка W
% - Расход Q
% - Потери/трение F
% - Коэффициент трения f
% - Минимальная толщина пленки h_min
% - Кавитационная зона

Основными выходными величинами являются...

% [ТАБЛИЦА 2.1: Безразмерные группы и масштабы]
\begin{table}[h]
\centering
\caption{Безразмерные группы и характерные масштабы}
\label{tab:dimensionless_groups}
\begin{tabular}{|l|l|l|}
\hline
\textbf{Величина} & \textbf{Размерная} & \textbf{Безразмерная} \\
\hline
Давление & $p$ & $\bar{p} = p h_0^2 / (\mu U L)$ \\
Координата & $x$ & $\bar{x} = x / L$ \\
Толщина & $h$ & $\bar{h} = h / h_0$ \\
% TODO: Добавить остальные & & \\
\hline
\end{tabular}
\end{table}


% =============================================================
\section{Модель с детерминированными микроструктурами}

\subsection{Параметризация микрорельефа: $h(x,z)$ и параметры}

% СОДЕРЖАНИЕ:
% - Как задается функция h(x,z)
% - Параметры: глубина h_d, радиусы a и b, шаг p, доля площади φ

Для учета детерминированных микроструктур рабочую поверхность представляем 
в виде гладкой базовой поверхности с наложенными регулярными элементами...

% [РИСУНОК 2.5: Типовая микроструктура с параметрами]
\begin{figure}[h]
\centering
\fbox{\parbox{0.8\textwidth}{\centering
\vspace{3cm}
ПЛЕЙСХОЛДЕР ДЛЯ РИСУНКА 2.5\\
Типовая микроструктура (лунка/канавка)\\
Параметры: $a$, $b$, $h_d$, профиль в разрезе
\vspace{3cm}
}}
\caption{Типовая микроструктура с основными параметрами}
\label{fig:microtexture_params}
\end{figure}

% [ТАБЛИЦА 2.2: Параметры микроструктуры]
\begin{table}[h]
\centering
\caption{Параметры детерминированной микроструктуры}
\label{tab:texture_params}
\begin{tabular}{|l|l|l|}
\hline
\textbf{Параметр} & \textbf{Обозначение} & \textbf{Размерность} \\
\hline
Глубина элемента & $h_d$ & мкм \\
Радиус (полуось) & $a, b$ & мкм \\
Шаг раскладки & $p$ & мкм \\
Доля площади & $\phi$ & -- \\
% TODO: Добавить остальные & & \\
\hline
\end{tabular}
\end{table}


\subsection{Типы геометрии микроструктур}

% СОДЕРЖАНИЕ:
% - Лунки (круглые/эллиптические/эллипсоидные)
% - Канавки/борозды (продольные/поперечные)
% - Комбинированные элементы

Рассмотрим основные типы геометрии микроструктур...

% [РИСУНОК 2.6: Геометрия различных микроструктур]
\begin{figure}[h]
\centering
\fbox{\parbox{0.8\textwidth}{\centering
\vspace{3cm}
ПЛЕЙСХОЛДЕР ДЛЯ РИСУНКА 2.6\\
Типы микроструктур:\\
(a) круглая лунка, (b) эллиптическая лунка,\\
(c) продольная канавка, (d) поперечная канавка
\vspace{3cm}
}}
\caption{Основные типы геометрии микроструктур}
\label{fig:texture_geometries}
\end{figure}


\subsection{Типы раскладки микроструктур}

% СОДЕРЖАНИЕ:
% - Регулярная решетка (шахматная/прямоугольная/гексагональная)
% - Кольцевая/секторальная
% - Филлотаксис

Рассмотрим варианты пространственного расположения микроструктур...

% [РИСУНОК 2.7: Шахматная раскладка]
\begin{figure}[h]
\centering
\fbox{\parbox{0.6\textwidth}{\centering
\vspace{2cm}
ПЛЕЙСХОЛДЕР ДЛЯ РИСУНКА 2.7\\
Шахматная раскладка микролунок\\
(вид сверху, координаты центров)
\vspace{2cm}
}}
\caption{Шахматная раскладка микроструктур}
\label{fig:layout_staggered}
\end{figure}

% [РИСУНОК 2.8: Кольцевая раскладка]
\begin{figure}[h]
\centering
\fbox{\parbox{0.6\textwidth}{\centering
\vspace{2cm}
ПЛЕЙСХОЛДЕР ДЛЯ РИСУНКА 2.8\\
Кольцевая раскладка (для радиального подшипника)
\vspace{2cm}
}}
\caption{Кольцевая раскладка микроструктур}
\label{fig:layout_annular}
\end{figure}

% [РИСУНОК 2.9: Филлотаксис]
\begin{figure}[h]
\centering
\fbox{\parbox{0.6\textwidth}{\centering
\vspace{2cm}
ПЛЕЙСХОЛДЕР ДЛЯ РИСУНКА 2.9\\
Раскладка по принципу филлотаксиса
\vspace{2cm}
}}
\caption{Раскладка микроструктур по принципу филлотаксиса}
\label{fig:layout_phyllotaxis}
\end{figure}


\subsection{Безразмеризация и определяющие параметры}

% СОДЕРЖАНИЕ:
% - Переход к безразмерным переменным
% - Набор определяющих параметров
% - Упрощенные формы для конкретных геометрий

Переход к безразмерным переменным...

% TODO: Написать про безразмеризацию


% =============================================================
\section{Кавитация и уравнения JFO (массо-сохраняющая постановка)}

\subsection{Физика разрыва пленки и ограничения модели с обнулением давления}

% СОДЕРЖАНИЕ:
% - Что происходит при отрицательном давлении
% - Почему простое p=0 неверно (нарушение массового баланса)
% - Физическая картина кавитации

При решении уравнения Рейнольдса в ряде случаев получаются 
отрицательные значения давления...

% [РИСУНОК 2.10: Схема кавитации]
\begin{figure}[h]
\centering
\fbox{\parbox{0.8\textwidth}{\centering
\vspace{3cm}
ПЛЕЙСХОЛДЕР ДЛЯ РИСУНКА 2.10\\
Схема кавитации в смазочном слое:\\
зона полного заполнения (p > 0),\\
зона кавитации (p = 0, частичное заполнение)
\vspace{3cm}
}}
\caption{Схематическое представление кавитации в смазочном слое}
\label{fig:cavitation_scheme}
\end{figure}


\subsection{Уравнения Jakobsson-Floberg-Olsson}

% СОДЕРЖАНИЕ:
% - Полная формулировка JFO
% - Условия переключения
% - Форма, пригодная для численной реализации

Массо-сохраняющая модель кавитации, предложенная Jakobsson, Floberg и Olsson...

% TODO: Написать уравнения JFO


\subsection{Связь с численной реализацией}

% СОДЕРЖАНИЕ:
% - Что именно будет реализовано в главе 3
% - Как JFO встраивается в численную схему
% - Алгоритм переключения между зонами

В главе 3 будет представлена численная реализация описанной модели...

% [РИСУНОК 2.11: Блок-схема модели]
\begin{figure}[h]
\centering
\fbox{\parbox{0.8\textwidth}{\centering
\vspace{3cm}
ПЛЕЙСХОЛДЕР ДЛЯ РИСУНКА 2.11\\
Блок-схема:\\
Входные параметры → Уравнения Рейнольдса + JFO →\\
→ Численное решение → Выходные величины
\vspace{3cm}
}}
\caption{Блок-схема математической модели}
\label{fig:model_flowchart}
\end{figure}


% =============================================================
\section*{Выводы по главе 2}
\addcontentsline{toc}{section}{Выводы по главе 2}

% СОДЕРЖАНИЕ (0.5-1.5 страницы):
% - 5-10 пунктов о принятой модели
% - Какие допущения
% - Какие безразмерные формы получены
% - Какие граничные условия/замыкание (JFO)
% - Какие параметры микроструктуры будут варьироваться

\begin{enumerate}
\item В главе представлена математическая постановка задачи гидродинамической 
смазки с учетом детерминированных микроструктур рабочей поверхности.

\item Получено уравнение Рейнольдса для описания распределения давления 
в тонком смазочном слое...

% TODO: Добавить остальные выводы (всего 5-10 пунктов)

\end{enumerate}