% =============================================================
% Оборот титула (УДК, ББК, аннотация)
% =============================================================
% НЕ используем \thispagestyle{empty} — номер страницы должен быть (это страница 2)

\noindent УДК ХХХ.ХХ\\
ББК ХХ.ХХХ\\
Х-000

\vspace{1cm}

\noindent\textbf{Авторы:}\\[0.3em]
\textit{<Должность, степень, ФИО автора 1>}\\
\textit{<Должность, степень, ФИО автора 2>}

\vspace{0.5cm}

\noindent\textbf{Рецензенты:}\\[0.3em]
доктор технических наук, профессор \textit{<ФИО>}\\
доктор технических наук, профессор \textit{<ФИО>}

\vspace{1cm}

Монография посвящена исследованию детерминированных микроструктур~--- регулярных рельефных элементов, целенаправленно наносимых на рабочую поверхность подшипников скольжения для улучшения их трибологических, динамических и~шумовых характеристик. Рассмотрены геометрия микрорельефа и~варианты расположения элементов (кольцевое, шахматное, филлотаксис). Представлены постановки гидродинамических задач на~основе уравнения Рейнольдса и~численные методы их решения с~учётом кавитации, термовязкостных эффектов и~шероховатости. Приведены результаты анализа подшипниковых узлов центробежного насоса, шарошечного долота и~устойчивости системы <<ротор~--- подшипник>>. Предназначена для научных работников, аспирантов и~инженеров в~области машиноведения и~трибологии.

\vfill

\noindent ISBN ХХХ-Х-ХХХХХ-ХХХ-Х

\vspace{0.3cm}

\noindent\hfill\copyright~<Авторы>, \the\year\\
\hfill\copyright~СФУ, \the\year

\newpage